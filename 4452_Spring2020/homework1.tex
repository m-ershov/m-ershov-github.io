\documentclass[12pt]{amsart}

\usepackage{amsmath}
\usepackage{amssymb}
\usepackage{amsthm}
%\usepackage{psfig}

\begin{document}
\baselineskip=16pt
\textheight=8.5in
%\parindent=0pt 
\def\sk {\hskip .5cm}
\def\skv {\vskip .08cm}
\def\cos {\mbox{cos}}
\def\sin {\mbox{sin}}
\def\tan {\mbox{tan}}
\def\intl{\int\limits}
\def\lm{\lim\limits}
\newcommand{\frc}{\displaystyle\frac}
\def\xbf{{\mathbf x}}
\def\fbf{{\mathbf f}}
\def\gbf{{\mathbf g}}

\def\dbA{{\mathbb A}}
\def\dbB{{\mathbb B}}
\def\dbC{{\mathbb C}}
\def\dbD{{\mathbb D}}
\def\dbE{{\mathbb E}}
\def\dbF{{\mathbb F}}
\def\dbG{{\mathbb G}}
\def\dbH{{\mathbb H}}
\def\dbI{{\mathbb I}}
\def\dbJ{{\mathbb J}}
\def\dbK{{\mathbb K}}
\def\dbL{{\mathbb L}}
\def\dbM{{\mathbb M}}
\def\dbN{{\mathbb N}}
\def\dbO{{\mathbb O}}
\def\dbP{{\mathbb P}}
\def\dbQ{{\mathbb Q}}
\def\dbR{{\mathbb R}}
\def\dbS{{\mathbb S}}
\def\dbT{{\mathbb T}}
\def\dbU{{\mathbb U}}
\def\dbV{{\mathbb V}}
\def\dbW{{\mathbb W}}
\def\dbX{{\mathbb X}}
\def\dbY{{\mathbb Y}}
\def\dbZ{{\mathbb Z}}

\def\la{{\langle}}
\def\ra{{\rangle}}
\def\summ{{\sum\limits}}

\bf\centerline{Homework \#1. Due Thursday, January 23rd, by 2pm in my mailbox}\rm
\vskip .1cm
All reading assignments and references to exercises, definitions etc. are from our main book `Coding Theory: A First Course' by Ling and Xing 
\vskip .1cm

\bf\centerline{Reading: }\rm
\skv
1. For this homework assignment: Chapters 1 and 2
\skv
\skv
2. Before the class on Wed, Jan 22: 3.1 and 4.1
\skv

\skv
\bf\centerline{Problems: }\rm
\skv
{\bf 1.} Recall that the parity-check code of length $n$, denoted below by $PCC_n$, is defined by
$$PCC_n=\{x_1\ldots x_n\in \{0,1\}^n: \sum_{i=1}^n x_i\mbox{ is even}\}.$$
\begin{itemize}
\item[(a)] Prove formally that $PCC_n$ is $1$-error detecting in the sense of Definition~2.5.4 (page 12).
\item[(b)] As observed in Lecture~1, $|PCC_n|=2^{n-1}$. Prove that $PCC_n$ is the largest possible $1$-error detecting binary code of length $n$, that is,
prove that if $C\subseteq \{0,1\}^n$ is any binary code of length $n$ which is $1$-error detecting, then $|C|\leq 2^{n-1}$.  
\end{itemize}
\skv



{\bf 2.} Given an integer $n\geq 2$, let $\dbZ_{n}=\{0,1,\ldots, n-1\}$ (here we are thinking of elements of $\dbZ_{n}$ as integers, not as congruence classes mod $n$, the latter being a typical convention in MATH 3354). Recall that the ISBN-10 code $I_{10}$ is defined by
$$I_{10}=\{x_1x_2\ldots x_{10}\in (\dbZ_{11})^{10}\mbox{ s.t. } 11 | (10x_1+9x_2+\ldots+2x_9+x_{10})\}$$
(The notation $a | b$ means that $a$ divides $b$, that is, $b=ac$ for some integer $c$).

The ISBN-13 code $C$, which replaced the ISBN-10 code in 2007, is defined by
$$I_{13}=\{x_1x_2\ldots x_{13}\in (\dbZ_{10})^{13}\mbox{ s.t. } 10\mid (x_1+3x_2+x_3+3x_4+\ldots+3x_{12}+x_{13})\}$$
(the coefficients are alternating between $1$ and $3$). Thus, the ISBN-13 code has larger length (13 instead of 10), but uses smaller alphabet (10 symbols instead of 11).
\begin{itemize}
\item[(a)] Prove that $I_{10}$ and $I_{13}$ are both $1$-error detecting.
\item[(b)] Prove that $I_{10}$ detects any transposition error, that is, any error where two different symbols in the original word are swapped (e.g.
$1357924687$ is sent and $1327954687$ is received).
\item[(c)] Prove that $I_{13}$ does not necessarily detect transposition errors. Does it detect some transposition errors? If yes, which ones?
\item[(d)] How would the properties of $I_{13}$ change if the weights $1$ and $3$ in the definition were replaced by another pair of integers?
 \end{itemize}
\skv

{\bf 3.} Let $C$ be a code of Hamming distance $d$. Suppose that a codeword $v_0\in C$ was transmitted, and let $k$ be the number of errors that occurred during the transmission, that is, $k=d_{Hamm}(v_0,w)$ where $w$ is the received. In class we proved (Theorem~2.2) that if $k\leq \frac{d-1}{2}$, then NND (nearest neighbor decoding) rule works correctly, that is, $v_0=c(w)$ where $c(w)$ is the result of applying NND to $w$.

Now assume that $d$ is even and we only know that $k\leq \frac{d}{2}$. Prove that while NND rule may not work correctly in this case, it still correctly determines the number of transmission errors, that is, $$d_{Hamm}(v_0,w)=d_{Hamm}(c(w),w).$$  
\skv
{\bf 4.} Problem 2.7, page 15. Replace IMLD by INND in the instructions for this problem.
\skv
{\bf 5.} Problem 2.8, page 15. Make sure to prove your answer.
\skv
{\bf 6.}
\begin{itemize} 
\item[(a)] Problem 2.3, page 15.
\item[(b)] Give an example of a word $w$ showing that if the memoryless binary channel from Problem~2.3 is used for transmission and $w$ is the received word, 
then both the complete NND rule and the complete MLD rule apply to $w$ without getting to the ``random choice'' stage, but yield different answers. 
\end{itemize}
\end{document}



