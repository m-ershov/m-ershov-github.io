\documentclass[12pt]{amsart}

\usepackage{amsmath}
\usepackage{amssymb}
\usepackage{amsthm}
%\usepackage{psfig}

\begin{document}
\baselineskip=16pt
\textheight=8.5in
%\parindent=0pt 
\def\sk {\hskip .5cm}
\def\skv {\vskip .08cm}
\def\cos {\mbox{cos}}
\def\sin {\mbox{sin}}
\def\tan {\mbox{tan}}
\def\intl{\int\limits}
\def\lm{\lim\limits}
\newcommand{\frc}{\displaystyle\frac}
\def\xbf{{\mathbf x}}
\def\fbf{{\mathbf f}}
\def\gbf{{\mathbf g}}

\def\dbA{{\mathbb A}}
\def\dbB{{\mathbb B}}
\def\dbC{{\mathbb C}}
\def\dbD{{\mathbb D}}
\def\dbE{{\mathbb E}}
\def\dbF{{\mathbb F}}
\def\dbG{{\mathbb G}}
\def\dbH{{\mathbb H}}
\def\dbI{{\mathbb I}}
\def\dbJ{{\mathbb J}}
\def\dbK{{\mathbb K}}
\def\dbL{{\mathbb L}}
\def\dbM{{\mathbb M}}
\def\dbN{{\mathbb N}}
\def\dbO{{\mathbb O}}
\def\dbP{{\mathbb P}}
\def\dbQ{{\mathbb Q}}
\def\dbR{{\mathbb R}}
\def\dbS{{\mathbb S}}
\def\dbT{{\mathbb T}}
\def\dbU{{\mathbb U}}
\def\dbV{{\mathbb V}}
\def\dbW{{\mathbb W}}
\def\dbX{{\mathbb X}}
\def\dbY{{\mathbb Y}}
\def\dbZ{{\mathbb Z}}

\def\la{{\langle}}
\def\ra{{\rangle}}
\def\summ{{\sum\limits}}

\bf\centerline{Homework \#2. Due Wednesday, January 29th, in class}\rm
\vskip .1cm
All reading assignments and references to exercises, definitions etc. are from our main book `Coding Theory: A First Course' by Ling and Xing 
\vskip .1cm

\bf\centerline{Reading: }\rm
\skv
1. For this homework assignment: 3.1, 3.2 and 4.1
\skv
\skv
2. For the classes next week: 4.2-4.6. It is very unlikely that we will cover all the material from those sections in 2 classes, but I might touch on at least one topic from each section.
\skv

\skv
\bf\centerline{Problems: }\rm
\skv
{\bf 1.} Problem 3.1, page 36

{\bf 2.} Problem 3.2, page 36

{\bf 3.} 
\begin{itemize}
\item[(a)] Use the Euclidean algorithm to find integers $u$ and $v$ such that $127u+35v=1$. If you have not studied this before, see Lecture~4 of my 3354 notes.
\item[(b)] Use your answer in (a) to compute $35^{-1}$ in $\dbZ_{127}$. Make sure to explain your logic.
\end{itemize}

{\bf 4.} Problem 3.4, page 36. {\bf Hint:} (a) can be proved directly from the standard formula for binomial coefficients and
basic divisibility properties. Then use (a) to solve both (b) and (c).

{\bf 5.} Problem 3.5, page 36

{\bf 6.} Let $F$ be a finite field, and let $Q$ be the set of all nonzero squares in $F$, that is, all nonzero elements of $F$ representable as $a^2$ for some $a\in F$.
\begin{itemize}
\item[(a)] Assume that $F$ has odd characteristic (for the definition of characteristic see Definition 3.1.10 on page 21). Prove that $|Q|=\frac{|F|-1}{2}$.
{\bf Hint:} Prove that for every nonzero $b\in F$ the equation $x^2=b$ has either no solutions (for $x$) in $F$ or exactly two solutions. 
\item[(b)] Now assume that $F$ has characteristic $2$. Prove that $|Q|=|F|-1$, that is, every nonzero element of $F$ is a square. {\bf Hint:}
Since $F$ is finite, it suffices to prove that the map $x\mapsto x^2$ from $F\setminus\{0\}$ to $F\setminus\{0\}$ is injective (one-to-one). The latter is not hard to prove directly (using the assumption ${\rm char} F=2$). 
\end{itemize}
For both parts you will likely need to use the fact that fields have no zero divisors (Lemma~3.1.3(ii) from the book).


{\bf 7.} Problem 4.2, page 66. Ignore the question about the number of bases, but make sure to prove your answer whether the given set is a subspace or not. If the set in question is a subspace, compute its dimension (also with proof).
\end{document}



