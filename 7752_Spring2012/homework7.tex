\documentclass[12pt]{article}

\usepackage{amsmath}
\usepackage{amssymb}
\usepackage{amsthm}
%\usepackage{psfig}

\begin{document}
\baselineskip=15pt
\textheight=8.7in
\parindent=0pt
\def\sk {\hskip .5cm}
\def\skv {\vskip .07cm}
\def\cos {\mbox{cos}}
\def\sin {\mbox{sin}}
\def\tan {\mbox{tan}}
\def\intl{\int\limits}
\def\lm{\lim\limits}
\newcommand{\frc}{\displaystyle\frac}
\def\xbf{{\mathbf x}}
\def\fbf{{\mathbf f}}
\def\gbf{{\mathbf g}}

\def\Ker{{\rm Ker\,}}
\def\phi{\varphi}

\def\dbA{{\mathbb A}}
\def\dbB{{\mathbb B}}
\def\dbC{{\mathbb C}}
\def\dbD{{\mathbb D}}
\def\dbE{{\mathbb E}}
\def\dbF{{\mathbb F}}
\def\dbG{{\mathbb G}}
\def\dbH{{\mathbb H}}
\def\dbI{{\mathbb I}}
\def\dbJ{{\mathbb J}}
\def\dbK{{\mathbb K}}
\def\dbL{{\mathbb L}}
\def\dbM{{\mathbb M}}
\def\dbN{{\mathbb N}}
\def\dbO{{\mathbb O}}
\def\dbP{{\mathbb P}}
\def\dbQ{{\mathbb Q}}
\def\dbR{{\mathbb R}}
\def\dbS{{\mathbb S}}
\def\dbT{{\mathbb T}}
\def\dbU{{\mathbb U}}
\def\dbV{{\mathbb V}}
\def\dbW{{\mathbb W}}
\def\dbX{{\mathbb X}}
\def\dbY{{\mathbb Y}}
\def\dbZ{{\mathbb Z}}

\def\Aut{{\rm Aut}}

\def\la{{\langle}}
\def\ra{{\rangle}}

\bf\centerline{Homework Assignment \# 7. }\rm
\skv
{\bf Plan for next week}: start Galois theory (online lecture 19, DF 14.1 and parts of 14.2)
\skv
\bf\centerline{Problems, to be submitted by Thursday, March 22nd. }\rm
\skv

{\bf Problem 1:} \rm Let $F$ be a field and $\Omega$ a subset of $F[x]$.
Use the existence and uniqueness of algebraic closures to prove that
there exists a unique splitting field for $\Omega$ over $F$ up to
$F$-isomorphism. {\bf Hint:} First show that any splitting field 
$K$ for $\Omega$ lies in some algebraic closure of $F$.
\skv

{\bf Problem 2:} \rm Before doing this problem read \S~13.6 in DF.
Let $p$ be a prime, $n\geq 2$ an integer, $f(x)=x^n-p$, and let
$K\subset \dbC$ be the splitting field for $f(x)$ over $\dbQ$.
As we proved in class, $K=\dbQ(\sqrt[n]{p},\omega_n)$ where
$\omega_n=e^{2\pi i/n}$.
\begin{itemize}
\item[(a)] Prove that $[K:\dbQ]\leq n\phi(n)$ where $\phi$
is the Euler function.
\item[(b)] Assume that $n$ is prime. Prove that inequality in (a)
is equality.
\item[(c)] Let $p=3$ and $n=12$. Prove that inequality in (a)
is strict and find $[K:\dbQ]$. {\bf Hint:} Compute $\omega_{12}$
explicitly.
\end{itemize}
\skv
{\bf Problem 3 (practice):} (a) Prove that if  $K/\dbQ$ is any field extension,
then any automorphism of $K$ must fix $\dbQ$ elementwise.

(b) Prove that the automorphism group $\Aut(\dbR/\dbQ)$ is trivial.
See [DF, Problem~7, p.567] for a sketch of the proof.

\skv
{\bf Problem 4:} Let $K/F$ be an algebraic extension. Prove that $K/F$
is normal if and only if for any algebraic extension $L/K$ and
any $F$-automorphism $\sigma\in \Aut_F(L)$ we have $\sigma(K)=K$.

\skv
{\bf Problem 5:} Let $K/F$ be a field extension, and let $K_1$ and $K_2$
be subfields of $K$ containing $F$ such that the extensions
$K_1/F$ and $K_2/F$ are normal. Prove that the extensions
$K_1 K_2/F$ and $K_1\cap K_2/F$ are also normal.
\skv
{\bf Problem 6:} Problems 10 and 11 in [DF, p.556]
\skv
{\bf Problem 7:} Let $p_1,\ldots,p_k$ be distinct primes and
$K=\dbQ(\sqrt{p_1},\ldots,\sqrt{p_k})$. Find a primitive element
of $K/\dbQ$ (and justify that your element is indeed primitive).
\skv
{\bf Problem 8 (practice):} Problem 13 in [DF, p.556]. 
\end{document}
