\documentclass[12pt]{article}

\usepackage{amsmath}
\usepackage{amssymb}
\usepackage{amsthm}
%\usepackage{psfig}

\begin{document}
\baselineskip=15pt
\textheight=8in
\parindent=0pt
\def\sk {\hskip .5cm}
\def\skv {\vskip .12cm}
\def\cos {\mbox{cos}}
\def\sin {\mbox{sin}}
\def\tan {\mbox{tan}}
\def\intl{\int\limits}
\def\lm{\lim\limits}
\newcommand{\frc}{\displaystyle\frac}
\def\xbf{{\mathbf x}}
\def\fbf{{\mathbf f}}
\def\gbf{{\mathbf g}}

\def\Ker{{\rm Ker\,}}
\def\phi{\varphi}

\def\dbA{{\mathbb A}}
\def\dbB{{\mathbb B}}
\def\dbC{{\mathbb C}}
\def\dbD{{\mathbb D}}
\def\dbE{{\mathbb E}}
\def\dbF{{\mathbb F}}
\def\dbG{{\mathbb G}}
\def\dbH{{\mathbb H}}
\def\dbI{{\mathbb I}}
\def\dbJ{{\mathbb J}}
\def\dbK{{\mathbb K}}
\def\dbL{{\mathbb L}}
\def\dbM{{\mathbb M}}
\def\dbN{{\mathbb N}}
\def\dbO{{\mathbb O}}
\def\dbP{{\mathbb P}}
\def\dbQ{{\mathbb Q}}
\def\dbR{{\mathbb R}}
\def\dbS{{\mathbb S}}
\def\dbT{{\mathbb T}}
\def\dbU{{\mathbb U}}
\def\dbV{{\mathbb V}}
\def\dbW{{\mathbb W}}
\def\dbX{{\mathbb X}}
\def\dbY{{\mathbb Y}}
\def\dbZ{{\mathbb Z}}

\def\Aut{{\rm Aut}}
\def\Im{{\rm Im}}


\def\la{{\langle}}
\def\ra{{\rangle}}

\bf\centerline{Homework Assignment \# 3. }\rm
\vskip .1cm
{\bf Plan for next week:} Modules over PID continued (12.1, online lectures 7-9).
\vskip .1cm

\bf\centerline{Problems, to be submitted by Thu, February 9th. }\rm
\vskip .1cm

\bf{Problem 1. }\rm The main goal of this problem is to classify
$2$-dimensional $\dbR$-algebras ($\dbR$=reals), that is,
$\dbR$-algebras which are 2-dimensional as vector spaces over $\dbR$.
Note that the idea of solution was discussed in the last lecture of Algebra-I.
\skv
Let $F$ be a field with ${\rm char}(F)\neq 2$, and
let $A$ be a $2$-dimensional $F$-algebra (as always, with $1$).
\begin{itemize}
\item[(a)] Let $u\in A$ be any element which is not an $F$-multiple of $1$.
Prove that
\begin{itemize}
\item[(i)] $u$ generates $A$ as an $F$-algebra, that is, the minimal $F$-subalgebra of $A$ containing $u$ and $1$ is $A$ itself.
\item[(ii)] $u$ satisfies a quadratic equation $au^2+bu+c=0$
for some $a,b,c\in F$ with $a\neq 0$.
\end{itemize}
\item[(b)] Show that there exists $v\in A$ such that $v^2\in F$.
{\bf Hint:} take any $u$ as in (a), and look for $v$ of the form
$u+\beta$ with $\beta\in F$.
\item[(c)] Deduce from (b) that $A$ is isomorphic as an $F$-algebra
to  $F[x]/(x^2-c)$ for some $c\in F$. 
\item[(d)] Prove that if $c=d^2$ for some $d\in F$, then $F[x]/(x^2-c)\cong F\times F$.
\item[(e)] Now let $F=\dbR$ (real numbers). Prove that in (c) one can choose
$c=0,1$ or $-1$. Then prove that the algebras $\dbR[x]/(x^2+1)$, $\dbR[x]/(x^2-1)$ and $\dbR[x]/x^2$ 
are pairwise non-isomorphic. {\bf Hint:} the algebras can be distinguished from
each other by simple abstract properties.
\end{itemize}

{\bf Problem 2.} Let $R$ be a commutative ring with $1$. A left $R$-module $M$ is called {\it Noetherian} if it satisfies the ascending chain condition on submodules and {\it Artinian}
if it satisfies the descending chain condition on submodules. Assume that an $R$-module $M$
is both Artinian and Noetherian. (For example, $R$ might be a field, and $M$ might be a 
finite-dimensional vector space over $R$). Let $T:M\to M$ be an $R$-module homomorphism.
\begin{itemize}
\item[(a)] Prove that there exists $k\in\dbN$ s.t. $\Ker(T^k)=\Ker(T^{2k})$ and $\Im(T^k)=\Im(T^{2k})$.
\item[(b)] Prove that if $k$ is as in part (a), then $M=\Ker(T^k)\oplus \Im(T^k)$
\item[(c)] Deduce from (a) and (b) that there exist submodules $M_0$ and $M_1$ of $M$
s.t. $M=M_0\oplus M_1$, $T_{| M_0}$ is nilpotent and $T_{|M_1}$ is invertible (as a map from $M_1$ to $M_1$).
\end{itemize}



{\bf Problem 3. } Let $V$ and $W$ be finite dimensional vector spaces
over a field $F$, let $\{v_1,\ldots, v_n\}$ be a basis of $V$
and $\{w_1,\ldots, w_m\}$ a basis of $W$.

Let  $\phi: V\otimes_F W\to Mat_{n\times m}(F)$ be the
$F$-linear transformation such that $\phi(v_i\otimes w_j)=e_{ij}$
where $e_{ij}$ is the matrix whose $(i,j)$-entry is equal to $1$ and all
other entries are equal to $0$ (note that such transformation
exists and is unique because $\{v_i\otimes w_j : 1\leq i\leq n, 1\leq j\leq m\}$
is a basis for $V\otimes_F W$; furthermore, $\phi$ is an isomorphism
since matrices $\{e_{ij}\}$ form a basis of $Mat_{n\times m}(F)$).

Prove that for a matrix $A\in Mat_{n\times m}(F)$ the following are equivalent:
\begin{itemize}
\item[(a)] $A=\phi(v\otimes w)$ for some $v\in V, w\in W$ (note: $v$ and $w$
need not be elements of the above bases)

\item[(b)] $rk(A)\leq 1$.
\end{itemize}
Note that this problem yields a one-line solution to Problem~5 from HW\#2.
\skv

{\bf Problem 4 (practice).} Let $R=\oplus_{n=0}^{\infty} R_n$ be a graded ring.
An element $r\in R$ is called \emph{homogeneous} if $r\in R_n$
for some $n$. 

\sk Any $r\in R$ can be uniquely written as $r=\sum_{n=0}^{\infty} r_n$
where $r_n\in R_n$ and only finitely many $r_n$'s are nonzero. The elements
$\{r_n\}$ are called the \emph{homogeneous components of $r$}.

(a) Let $I$ be an ideal of $R$. Prove that the following are equivalent:
\begin{itemize}
\item[(i)] $I$ is a graded ideal, that is, $I=\oplus_{n=0}^{\infty} I\cap R_n$  
\item[(ii)] For each $r\in I$ all homogeneous components of $r$ also lie in $I$
\end{itemize}

(b) Let $I$ be an ideal of $R$ generated by homogeneous elements
(possibly of different degrees). Prove that $I$ is graded.

\skv
{\bf Problem 5.} $\empty$
\begin{itemize}
\item[(a)] Let $R$ be a commutative ring with $1$ and $M$ an $R-module$.
Let $m_1,\ldots, m_k$ be elements of $M$ and $\sigma\in S_k$ a permutation.
Prove that $m_{\sigma(1)}\wedge\ldots\wedge m_{\sigma(k)}=(-1)^{\sigma}
m_1\ldots m_k$.
\item[(b)] Use (a) to prove Proposition~6.5 from the online version of Lecture~6.
\end{itemize}
\end{document}
