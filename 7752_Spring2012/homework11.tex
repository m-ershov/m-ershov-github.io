\documentclass[12pt]{article}

\usepackage{amsmath}
\usepackage{amssymb}
\usepackage{amsthm}
%\usepackage{psfig}

\begin{document}
\baselineskip=15pt
\textheight=8.7in
\parindent=0pt
\def\sk {\hskip .5cm}
\def\skv {\vskip .07cm}
\def\cos {\mbox{cos}}
\def\sin {\mbox{sin}}
\def\tan {\mbox{tan}}
\def\intl{\int\limits}
\def\lm{\lim\limits}
\newcommand{\frc}{\displaystyle\frac}
\def\xbf{{\mathbf x}}
\def\fbf{{\mathbf f}}
\def\gbf{{\mathbf g}}

\def\Ker{{\rm Ker\,}}
\def\Gal{{\rm Gal\,}}
\def\phi{\varphi}

\def\dbA{{\mathbb A}}
\def\dbB{{\mathbb B}}
\def\dbC{{\mathbb C}}
\def\dbD{{\mathbb D}}
\def\dbE{{\mathbb E}}
\def\dbF{{\mathbb F}}
\def\dbG{{\mathbb G}}
\def\dbH{{\mathbb H}}
\def\dbI{{\mathbb I}}
\def\dbJ{{\mathbb J}}
\def\dbK{{\mathbb K}}
\def\dbL{{\mathbb L}}
\def\dbM{{\mathbb M}}
\def\dbN{{\mathbb N}}
\def\dbO{{\mathbb O}}
\def\dbP{{\mathbb P}}
\def\dbQ{{\mathbb Q}}
\def\dbR{{\mathbb R}}
\def\dbS{{\mathbb S}}
\def\dbT{{\mathbb T}}
\def\dbU{{\mathbb U}}
\def\dbV{{\mathbb V}}
\def\dbW{{\mathbb W}}
\def\dbX{{\mathbb X}}
\def\dbY{{\mathbb Y}}
\def\dbZ{{\mathbb Z}}

\def\Aut{{\rm Aut}}
\def\deg{{\rm deg}}

\def\la{{\langle}}
\def\ra{{\rangle}}

\bf\centerline{Homework Assignment \# 11. }\rm

{\bf Plan for the next week:} Dimension theory of affine varieties (continued) and
integral extensions. Good references on commutative algebra and algebraic geometry freely 
available online are notes by J. Milne

\centerline{http://www.jmilne.org/math/xnotes/CA.pdf} 

and 

\centerline{http://www.jmilne.org/math/CourseNotes/AG.pdf}

\skv
\skv
\bf\centerline{Problems, to be submitted by Thu, April 26th. }\rm
\skv
{\bf Note:} Bonus problems should not be discussed with other people,
can be submitted separately any time before the final exam on May 5th,
and will be counted towards your final exam score (the exact weights
will be determined later).
\skv
{\bf Problem 1:} Complete the proof of Lemma 24.2: if $k$ is an algebraically closed
field, $n\geq 1$ an integer and $f\in k[x_1,\ldots,x_n]$ a non-constant polynomial,
then $\dim (Z(f))=n-1$. Recall that the following remains to be done:
\begin{itemize}
\item[(i)] Show that if $f$ is irreducible and $E$ is the field of fractions of
$k[x_1,\ldots,x_n]/(f)$, then $E$ has transcendence degree $n-1$.
\item[(ii)] Assuming that the statement of the problem holds for irreducible $f$,
prove it in the general case.
\end{itemize}
\skv
{\bf Problem 2:} Let $k$ be an algebraically closed field. Prove that any
nonzero prime ideal  of $k[x,y]$ is equal to $(f)$ for some irreducible $f\in k[x,y]$
or $(x-a,y-b)$ for some $a,b\in k$. {\bf Hint:} Reformulate the problem in terms
of algebraic sets. You may use Proposition~2.26 from Milne's algebraic geometry notes.
\skv
\skv 
 
In Problems 3 and 4 we identify the set $Mat_n(k)$ of $n\times n$ matrices over a field $k$
with $k^{n^2}$ and thus can talk about Zariski topology on $Mat_n(k)$.
\skv
{\bf Problem 3:} Let $k$ be an infinite field.
\begin{itemize}
\item[(a)] Prove that $SL_n(k)=\{A\in Mat_n(k): \det(A)=1\}$ is Zariski closed
(that is, closed in Zariski topology) and find its dimension.

\item[(b)] Fix $1\leq d\leq n$, and let $R_d(n,k)$  be the set of all matrices in $Mat_n(k)$
which have rank $\leq d$. Prove that $R_d(n,k)$ is Zariski closed, guess its dimension and
give a heuristic argument.

\item[(c)] (BONUS) Now prove your guess for $\dim R_d(n,k)$. You may want to use the following result:
if a field extension $E/k$ is generated by a set $X$, then some subset of $X$ is a transcendence
basis. (We did not explicitly prove this in class, but the argument we used to prove the
existence of a transcendence basis carries over without any changes).
\end{itemize}

\skv
{\bf Problem 4:} Let $k$ be an arbitrary field. If $Y$ is a subset of $k^n$, we will
denote by $\overline Y$ the {\it Zariski closure} of $Y$, that is, the closure of $Y$ in the 
Zariski topology. 
\begin{itemize}
\item[(a)] Prove that $\overline Y=Z(I(Y))$ for any $Y\subseteq k^n$. 
\item[(b)] Let $A$ be a commutative subset of $Mat_n(k)$, that is, $ab=ba$ for all $a,b\in A$.
Prove that $\overline A$ is also commutative. {\bf Hint:} First show that for any $a\in Mat_n(k)$,
the centralizer of $a$ in $Mat_n(k)$ is Zariski closed. Then show that $ab=ba$ for all $a\in A$
and $b\in \overline A$ and finally deduce the assertion of the problem.
\end{itemize}
\skv
{\bf Problem 5:} In this problem $k$ is an infinite field.
Let $Y$ be a subset of $k^n$. Recall that in the notations from class 
$K[Y]=k[x_1,\ldots, x_n]/I(Y)$ is the set of polynomial functions from $Y$ to $k$. Let $O(Y)$
be the set of all everywhere defined rational functions on $Y$, that is,
all functions $f:Y\to k$ f for which there exist polynomials $p,q\in k[x_1,\ldots, x_n]$
s.t. $q$ does not vanish at any point of $Y$ and $f=p/q$ as a function on $Y$. Clearly,
$K[Y]\subseteq O(Y)$.
\begin{itemize}
\item[(a)] Prove that if $Y$ is an algebraic set, then $O(Y)=K[Y]$. {\bf Hint:} Use weak
Nullstellensatz.
\item[(b)] Let $Y=k^1\setminus\{0\}$, the affine line with $0$ removed. Prove that
$K[Y]=k[x]$ (polynomials in one variable) while $O(Y)=k[x,1/x]$.
\item[(c)] Find an algebraic subset $Z$ of $k^2$ such that $k[Z]\cong k[x,1/x]$.
How is $Z$ related to $Y$? (No formal answer is expected).
\item[(d)] Find a non-algebraic subset $W$ of $k^2$ for which $O(W)=k[W]\cong k[x_1,x_2]$.
\end{itemize}

\skv
{\bf Problem 6:} (practice) DF, Problem 18 on pp. 530--531.
\skv
{\bf Problem 7:} Before doing this Problem read (carefully) the Theorem and the subsequent 
example on pp. 647--648 of DF. Let $p$ be a prime, $F=\dbF_p$, the finite field of order $p$,
and $K=F(t)$, the field of rational functions over $F$ in one variable $t$.
Let $G=\Aut(K)=\Aut(K/F)$. As explained on page 647 of DF, $G$ is precisely
the group of fractional linear transformations $t\mapsto \frac{at+b}{ct+d}$, with
$a,b,c,d\in F$ and $ad-bc\neq 0$ and therefore isomorphic to $PGL_2(F)$.
\begin{itemize}
\item[(a)] Let $H$ be the cyclic subgroup of $G$ generated by the map $t\mapsto t+1$
and $L=K^H$, the fixed field of $H$. Find $u\in L$ such that $L=F(u)$.
Initially your $u$ may be expressed using sums and/or products, but the final
answer should be in ``closed form''.
\item[(b)] (BONUS) Now let $M=K^G$. Find $v\in M$ such that $M=F(v)$. Note
that for $p=2$, (one possible such) $v$ is given on page 648 of DF, but without
any explanation of where it came from. If you realize what the formula for that
$v$ really means, you should be able to find the corresponding $v$ for arbitrary $p$.
\item[(c)] (BONUS) Now find $v$ satisfying (b) without any ``guessing''. It is probably
difficult to put your $v$ in a closed form, but you can describe it in terms
of sums and products. It is fairly easy to ensure that $v\in M$; the trickier
part is to show that $M=F(v)$ without having $v$ in a closed form.


\end{itemize}

{\bf Problem 8:} (practice) Let $F$ be a field of characteristic not equal to $2$, and let
$$K=F(x,\sqrt{x^3-x})=F(x)[y]/(y^2-(x^3-x)).$$
Prove that the field extension $K/F$ cannot be generated by one element.

{\bf Hint:} Let $\bar x$ and $\bar y$ be the images of $x$ and $y$ in $K$, respectively.
Show that $[K:F(\bar x)]=3$ and $[K:F(\bar y)]=2$. Now suppose that
$L=F(t)$ for some $t$ (of course, $t$ must be transcendental, so we can think of $L$ as
the field of rational functions in the formal variable $t$). Then we must have
$\bar x=\frac{a(t)}{b(t)}$ and $\bar y=\frac{c(t)}{d(t)}$ for some polynomials
$a(t),b(t),c(t)$ and $d(t)$ such that
$$\left(\frac{c(t)}{d(t)}\right)^2=\left(\frac{a(t)}{b(t)}\right)^3-\frac{a(t)}{b(t)}.
\eqno(***)$$
Use Problem~6 to get restrictions on the degrees of $a,b,c,d$ and then
show that (***) has no solutions satisfying those restrictions.
\end{document} 