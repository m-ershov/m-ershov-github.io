\documentclass[12pt]{article}

\usepackage{amsmath}
\usepackage{amssymb}
\usepackage{amsthm}
%\usepackage{psfig}

\begin{document}
\baselineskip=15pt
\textheight=9in
\parindent=0pt 
\def\sk {\hskip .5cm}
\def\skv {\vskip .07cm}
\def\cos {\mbox{cos}}
\def\sin {\mbox{sin}}
\def\tan {\mbox{tan}}
\def\intl{\int\limits}
\def\lm{\lim\limits}
\newcommand{\frc}{\displaystyle\frac}
\def\xbf{{\mathbf x}}
\def\fbf{{\mathbf f}}
\def\gbf{{\mathbf g}}

\def\Ker{{\rm Ker\,}}
\def\phi{\varphi}

\def\dbA{{\mathbb A}}
\def\dbB{{\mathbb B}}
\def\dbC{{\mathbb C}}
\def\dbD{{\mathbb D}}
\def\dbE{{\mathbb E}}
\def\dbF{{\mathbb F}}
\def\dbG{{\mathbb G}}
\def\dbH{{\mathbb H}}
\def\dbI{{\mathbb I}}
\def\dbJ{{\mathbb J}}
\def\dbK{{\mathbb K}}
\def\dbL{{\mathbb L}}
\def\dbM{{\mathbb M}}
\def\dbN{{\mathbb N}}
\def\dbO{{\mathbb O}}
\def\dbP{{\mathbb P}}
\def\dbQ{{\mathbb Q}}
\def\dbR{{\mathbb R}}
\def\dbS{{\mathbb S}}
\def\dbT{{\mathbb T}}
\def\dbU{{\mathbb U}}
\def\dbV{{\mathbb V}}
\def\dbW{{\mathbb W}}
\def\dbX{{\mathbb X}}
\def\dbY{{\mathbb Y}}
\def\dbZ{{\mathbb Z}}

\def\Aut{{\rm Aut}}

\def\la{{\langle}}
\def\ra{{\rangle}}
\def\rk{{\rm rk}}

\bf\centerline{Homework Assignment \# 5. }\rm
\vskip .5cm
{\bf Plan for next week:} Jordan canonical form (12.3 and online
lecture 12+) and field extensions (13.1, 13.2 and online lecture 14).
\vskip .1cm

\bf\centerline{Problems, to be submitted by Thu, February 23rd. }\rm
\vskip .1cm
{\bf Problem 1:} Let $R$ be a commutative ring (with 1).
\skv
(a) Let $C$ be an $R$-algebra and let $A$ and $B$ be $R$-subalgebras of $C$
which commute with each other, that is, $ab=ba$ for any $a\in A, b\in B$
(note that $A$ and $B$ themselves do not have to be commutative).
Prove that there is an \underline{$R$-algebra homomorphism} $\phi:A\otimes_R B\to C$
such that $\phi(a\otimes b)=ab$ for each $a\in A$ and $b\in B$.
\skv
(b) Prove that $\dbZ[i]\otimes_{\dbZ} \dbR\cong \dbC$ as rings (as usual $\dbR$
is real numbers and $\dbC$ are complex numbers).
\skv
(c) Now assume that $R$ is a field, and let $A$ be a finite-dimensional
$R$-algebra. Prove that the algebra $A\otimes_R A$ cannot be a field
unless $\dim_R A=1$. {\bf Hint:} use (a).


\vskip .2cm
{\bf Problem 2: } \rm DF, Problem 6, page 488. 
\vskip .2cm
{\bf Problem 3: } \rm DF, Problem 9, page 489, first two matrices
are for practice. 
\vskip .2cm

{\bf Problem 4: } \rm Recall that for a matrix $A$ we denoted by
$\chi_A(x)$ and $\mu_A(x)$ its characteristic and minimal polynomials, respectively. Determine the number of possible RCFs of $8\times 8$ matrices $A$ over $\dbQ$ with 
$\chi_A(x)=x^8-x^4$. Explain your argument in detail.

\skv
{\bf Problem 5: } \rm (a) (practice) Prove that two $3\times 3$ matrices
over some field $F$ are similar if and only if they have the same minimal and characteristic 
polynomials. Give an example showing that this does not hold for $4\times 4$ matrices.

(b) A matrix $A$ is called idempotent if $A^2=A$. Prove that two idempotent $n\times n$ matrices are similar if and only if they have they same rank. {\bf Hint:} What is the minimal polynomial of an idempotent matrix? How does rank relate to eigenvalue $0$?

\skv
{\bf Problem 6: } Find the number of distinct conjugacy classes in the group $GL_3(\dbZ/2\dbZ)$
and specify one element in each conjugacy class.
\end{document}