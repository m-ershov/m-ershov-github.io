\documentclass[12pt]{article}

\usepackage{amsmath}
\usepackage{amssymb}
\usepackage{amsthm}
%\usepackage{psfig}

\begin{document}
\baselineskip=15pt
\textheight=8in
\parindent=0pt
\def\sk {\hskip .5cm}
\def\skv {\vskip .12cm}
\def\cos {\mbox{cos}}
\def\sin {\mbox{sin}}
\def\tan {\mbox{tan}}
\def\intl{\int\limits}
\def\lm{\lim\limits}
\newcommand{\frc}{\displaystyle\frac}
\def\xbf{{\mathbf x}}
\def\fbf{{\mathbf f}}
\def\gbf{{\mathbf g}}

\def\Ker{{\rm Ker\,}}
\def\phi{\varphi}

\def\dbA{{\mathbb A}}
\def\dbB{{\mathbb B}}
\def\dbC{{\mathbb C}}
\def\dbD{{\mathbb D}}
\def\dbE{{\mathbb E}}
\def\dbF{{\mathbb F}}
\def\dbG{{\mathbb G}}
\def\dbH{{\mathbb H}}
\def\dbI{{\mathbb I}}
\def\dbJ{{\mathbb J}}
\def\dbK{{\mathbb K}}
\def\dbL{{\mathbb L}}
\def\dbM{{\mathbb M}}
\def\dbN{{\mathbb N}}
\def\dbO{{\mathbb O}}
\def\dbP{{\mathbb P}}
\def\dbQ{{\mathbb Q}}
\def\dbR{{\mathbb R}}
\def\dbS{{\mathbb S}}
\def\dbT{{\mathbb T}}
\def\dbU{{\mathbb U}}
\def\dbV{{\mathbb V}}
\def\dbW{{\mathbb W}}
\def\dbX{{\mathbb X}}
\def\dbY{{\mathbb Y}}
\def\dbZ{{\mathbb Z}}

\def\Aut{{\rm Aut}}
\def\Im{{\rm Im}}


\def\la{{\langle}}
\def\ra{{\rangle}}
\def\rk{{\rm rk}}


\bf\centerline{Homework Assignment \# 4. }\rm
\vskip .1cm
{\bf Plan for next week:} Rational Canonical Form (12.2, online lectures 10-11).
\vskip .1cm

\bf\centerline{Problems, to be submitted by Thu, February 16th. }\rm
\vskip .1cm

{\bf Problem 1.} (a) Let $R$ be a commutative ring, let $M$ be an $R$-module
and $N$ its submodule. Prove that $M$ is Noetherian $\iff$
$N$ and $M/N$ are both Noetherian.

{\bf Hint:} The forward direction is easy. For the backwards direction,
observe that if $\{P_i\}$ is an ascending chain of submodules of $M$,
then $\{P_i\cap N\}$ is an ascending chain of submodules of $N$
and $\{(P_i+N)/N\}$ is an ascending chain of submodules of $M/N$.

(b) Let $R$ be a commutative Noetherian ring. Use (a) to prove that
$R^n$ is a Noetherian module for any $n\in\dbN$. 

(c) Use (a) and (b) to prove Lemma from class: if $R$ is Noetherian,
then every submodule of a finitely generated $R$-module is finitely generated.
\skv

{\bf Problem 2.} Let $A$ be a ring (with 1). A subring $B$ of $A$ is called a
\emph{retract} if there exists a surjective ring homomorphism
$\phi: A\to B$ such that $\phi_{| B}=id_B$, that is, $\phi(b)=b$ for all $b\in B$.

Now let $M$ and $N$ be two $R$-modules. Prove that the tensor algebra
$T(M)$ is (naturally isomorphic to) a subalgebra of $T(M\oplus N)$ and that this subalgebra is a retract. Also prove the analogous statement about the symmetric algebras.

\skv
{\bf Problem 3.} (This is the first half of the practice problem 2.7) Let $I$ and $J$ be ideals of a (commutative) ring $R$, and let $\pi_I:R\to R/I$ and $\pi_J:R\to R/J$ be canonical projections.
\begin{itemize}
\item[(a)] Prove that every element of $R/I\otimes_R R/J$ can be written as
a simple tensor $\pi_I(1)\otimes \pi_J(r)$ for some $r\in R$
and also as $\pi_I(r')\otimes \pi_J(1)$ for some $r'\in R$.
\item[(b)] Use (a) to prove that $R/I\otimes_R R/J\cong R/(I+J)$ (as $R$-modules).
\end{itemize}

{\bf Problem 4:} \rm Let $R$ be a PID. For an $R$-module $M$ define
$\rk(M)$ to be the minimal size of a generating set of $M$.
\begin{itemize}
\item[(a)] Let $M$ be a finitely generated $R$-module and
$R/a_1R\oplus \ldots \oplus R/a_mR\oplus R^s$ its invariant factor
decomposition, that is, $a_1,\ldots, a_m$ are nonzero 
non-units and $a_1\mid a_2\mid\ldots\mid a_m$. Prove that
$\rk(M)=m+s$. {\bf Warning:} It is not true in general that
$\rk(P\oplus Q)=\rk(P)+\rk(Q)$. {\bf Hint:} Let $p$ be a prime
dividing $a_1$. How is $M$ related to $M'=(R/pR)^{m+s}$
and what is $\rk(M')$ (and why)?

\item[(b)] Let $F$ be a free $R$-module of rank $n$ with basis 
$e_1,\ldots, e_n$, let $N$ be the submodule of $F$
generated by some elements $v_1,\ldots, v_n\in F$,
and let $A\in Mat_n(F)$ be the matrix such that
$$\left(\begin{array}{c} v_1 \\ \vdots \\ v_n\end{array}\right)=
A \left(\begin{array}{c} e_1 \\ \vdots \\ e_n\end{array}\right)$$ 
Find a simple condition on the entries of $A$ which holds
if and only if $\rk(F/N)=n$.
\end{itemize}

{\bf Problem 5}. Let $R=\mathbb R[x]$, $F=R^3$ (the standard 3-dimensional $R$-module)
and $N$ the $R$-submodule of $F$
generated by $(1-x,1,0)$, $(-2,4-x,0)$ and $(1,-5,-x)$.
\begin{itemize}
\item[(a)] Find compatible bases for $F$ and $N$, that is, bases satisfying the conclusion
of the submodule structure theorem.
\item[(b)] Describe the quotient module $F/N$ in IF and ED forms.
\end{itemize}
\end{document}