\documentclass[12pt]{amsart}

\usepackage{amsmath}
\usepackage{amssymb}
\usepackage{amsthm}
%\usepackage{psfig}

\newtheorem* {Definition}    {Definition}
\newtheorem* {Theorem}    {Theorem}
\newtheorem* {Lemma}    {Lemma}


\begin{document}
 \pagenumbering{gobble}
\baselineskip=16pt
\textheight=8.5in
\textwidth=6in
%\parindent=0pt 
\def\sk {\hskip .5cm}
\def\skv {\vskip .08cm}
\def\cos {\mbox{cos}}
\def\sin {\mbox{sin}}
\def\tan {\mbox{tan}}
\def\intl{\int\limits}
\def\lm{\lim\limits}
\newcommand{\frc}{\displaystyle\frac}
\def\xbf{{\mathbf x}}
\def\fbf{{\mathbf f}}
\def\gbf{{\mathbf g}}

\def\dbA{{\mathbb A}}
\def\dbB{{\mathbb B}}
\def\dbC{{\mathbb C}}
\def\dbD{{\mathbb D}}
\def\dbE{{\mathbb E}}
\def\dbF{{\mathbb F}}
\def\dbG{{\mathbb G}}
\def\dbH{{\mathbb H}}
\def\dbI{{\mathbb I}}
\def\dbJ{{\mathbb J}}
\def\dbK{{\mathbb K}}
\def\dbL{{\mathbb L}}
\def\dbM{{\mathbb M}}
\def\dbN{{\mathbb N}}
\def\dbO{{\mathbb O}}
\def\dbP{{\mathbb P}}
\def\dbQ{{\mathbb Q}}
\def\dbR{{\mathbb R}}
\def\dbS{{\mathbb S}}
\def\dbT{{\mathbb T}}
\def\dbU{{\mathbb U}}
\def\dbV{{\mathbb V}}
\def\dbW{{\mathbb W}}
\def\dbX{{\mathbb X}}
\def\dbY{{\mathbb Y}}
\def\dbZ{{\mathbb Z}}

\def\la{{\langle}}
\def\ra{{\rangle}}
\def\Ker{{\rm Ker}}
\def\rk{{\rm rk}}
\def\summ{{\sum\limits}}
\def\lra{\longrightarrow}
\def\str{\stackrel}

\def\Gal{{\rm Gal\,}}
\def\phi{{\varphi}}

\bf\centerline{Math 8851. Homework \#4. To be completed by 5pm on Fri, Oct 20}\rm
\vskip .1cm
Below [DDMS] refers to the book `Analytic pro-$p$ groups', 2nd edition by Dixon, du Sautoy, Mann and Segal.
\skv
Before stating Problem~1 we introduce several definitions.
\skv
\begin{Definition}\rm A {\it supernatural number} if a formal product $\prod\limits_{p}p^{a_p}$ where $p$ ranges over all primes and
each $a_p$ is either a non-negative integer or infinity.
\end{Definition}
\skv
Supernatural numbers form a monoid with respect to multiplication given by $$\prod\limits_{p}p^{a_p}\cdot \prod\limits_{p}p^{b_p}=
\prod\limits_{p}p^{a_p+b_p}$$ where as usual we set $\infty+x=x+\infty=\infty$ for any $x\in \dbZ_{\geq 0}\sqcup \{\infty\}$. 

It is not hard to show that for any non-empty set $S$ of supernatural numbers there are unique greatest common divisor $gcd(S)$ (which is a multiple of any common divisor of the elements of $S$) and least common multiple $LCM(S)$ (which divides any common multiple of the elements of $S$), and
morever both $gcd(S)$ and $LCM(S)$ are given by the standard formulas: if $S=\{s_i\}_{i\in I}$ where $s_i=\prod\limits_{p}p^{a_{i,p}}$, then $gcd(S)=\prod\limits_{p}p^{m_p}$ and $LCM(S)=\prod\limits_{p}p^{M_p}$ where $m_p=\inf\{a_{i,p}: i\in I\}$ and $M_p=\sup\{a_{i,p}: i\in I\}$.
\skv
If $G$ is a profinite group, the order of $G$ is the supernatural number $|G|$ defined by 
$$|G|=LCM(\{|G/N|: N \mbox{ is an open normal subgroup of }G\}).$$ 
Note that $G$ is pro-$p$ for some prime $p$ $\iff$ $|G|=p^a$ for some $a\in \dbZ_{\geq 0}\sqcup\{\infty\}$.

If $H$ is a closed subgroup of $G$, we define the index $[G:H]$ by $[G:H]=LCM(\{[G:U]\})$ where $U$ ranges over all open subgroups of $G$
containing $H$.

\begin{Definition}\rm Let $G$ be a profinite group and $p$ a prime dividing $|G|$. A closed subgroup $H$ of $G$ is called a 
{\it Sylow pro-$p$ subgroup} if $H$ is a pro-$p$ subgroup and $[G:H]$ is coprime to $p$.
\end{Definition}
One can show that Sylow pro-$p$ subgroups always exist and any two Sylow pro-$p$ subgroups of $G$ are conjugate (see Problems~1.11 and 1.12 in [DDMS]), but this is not part of this homeowrk.


{\bf 1.} 
\begin{itemize}
\item[(a)] Prove that if $G$ is a profinite group and $H$ is a closed normal subgroup of $G$, then $|G|=|G/H|\cdot |H|$.
\item[(b)] Let $G=SL_n(\dbZ_p)$ (where as usual $\dbZ_p$ is $p$-adic integers). Describe explicitly a Sylow pro-$p$ subgroup of $G$
and prove your answer. {\bf Hint:} Problem~5 from HW\#1 is relevant here.
\end{itemize}
\skv

{\bf 2.} We start with some definitions. Let $A$ be an associative ring with $1$ and $M$ a right $R$-module. A map $f:A\to M$
is called a {\it derivation} if
\begin{itemize}  
\item[(1)] $f(a+b)=f(a)+f(b)$ for all $a,b\in A$;
\item[(2)] $f(ab)=f(a).b+f(b)$ for all $a,b\in A$.
\end{itemize}
The set of all derivations from $A$ to $M$ (which is clearly an abelian group with respect to pointwise addition) will be denoted
by $Der(A,M)$.

If $G$ is a group and $M$ is a right $G$-module, a derivation from $G$ to $M$ is a map $G\to M$ satisfying (2) above (for all $a,b\in G$).
Again we denote by $Der(G,M)$ the set of all derivations from $G$ to $M$, which is still an abelian group. 
Recall that $Der(G,M)$ appeared in class in the course of the explicit description of the first cohomology,
namely $$H^1(G,M)\cong Der(G,M)/IDer(G,M)$$ where $IDer(G,M)$ is the subgroup of inner derivations (maps of the form $g\mapsto m-m.g$
for some fixed $m\in M$); however, this is not directly related to this problem. The main point of this problem is to give an important example of a derivation in the case of a non-trivial action (which actually arises in some proofs that I am going to discuss in class).
\skv
Now the actual problem begins
\begin{itemize}
\item[(a)] Let $G$ be a group and $M$ a right $G$-module. Prove that the restriction map $Der(\dbZ[G],M)\to Der(G,M)$ is an isomorphism of
abelian groups. 
\item[(b)] Again let $G$ be a group and $\omega_G$ be the augmentation ideal of $\dbZ[G]$ 
(the ideal generated by all elements of the form $g-1$, $g\in G$). Prove that if $X$ generates $G$ as a group, then the
set $\{x-1: x\in X\}$ generates $\omega_G$ as a right $G$-module (equivalently, $\dbZ[G]$-module).
\item[(c)] Now assume that $G$ is a free group and $X$ is a free generating set for $G$. Then one can show (this is not part of the problem) that
$\omega_G$ is a free right $\dbZ[G]$-module, freely generated by $\{x-1: x\in X\}$, that is, for any $f\in \omega_G$ there
exist unique elements $\{D_x(f)\}_{x\in X}$ such that $$f=\sum\limits_{x\in X} (x-1)D_x(f)$$ (if $X$ is infinite, we implicitly require that only finitely many $D_x(f)$ are nonzero). Prove that for any $x\in X$ the map $\frac{\partial}{\partial x}: G\to \dbZ[G]$
given by $\frac{\partial }{\partial x}(g)=D_x(g-1)$ is a derivation. It is called the (right) {Fox derivative} with respect to $x$.
\end{itemize}
\skv

{\bf 3.} Let $X$ and $Y$ be topological spaces and $C(X,Y)$ the space of continuous maps from $X$ to $Y$. The {\it compact-open} topology
on $C(X,Y)$ is the topology with subbase $\{U_{K,O}\}$ where $K\subseteq X$ is compact, $O\subseteq Y$ is open and 
$U_{K,O}=\{f\in C(X,Y): f(K)\subseteq U\}$.

Now let $W/F$ be a Galois extension and consider $\Gal(W/F)$ as a subset of $C(W,W)$ where $W$ is endowed with the discrete topology. Prove that
the Krull topology on $\Gal(W/F)$ coincides with the compact-open topology (that is, the topology induced from the compact-open topology on
$C(W,W)$).
\skv

{\bf 4.} Let $W/F$ be a Galois extension and $L$ a subfield of $W/F$.
\begin{itemize}
\item[(a)] Prove that the Krull topology on $\Gal(W/L)$ is induced from the Krull topology on $\Gal(W/F)$.
\item[(b)] Assume now that $L/F$ is Galois, so that $\Gal(W/L)$ is normal in $\Gal(W/F)$ and $\Gal(W/F)/\Gal(W/L)$ is canonically isomorphic to
$\Gal(L/F)$. Prove that under this isomorphism, the Krull topology on $\Gal(L/F)$ corresponds to the quotient topology on $\Gal(W/F)/\Gal(W/L)$.  
\end{itemize}
\skv

{\bf 5.} Let $\{d_n\}_{n\in\dbN}$ be a sequence of pairwise coprime integers and $K=\dbQ(\sqrt{d_1},\sqrt{d_2},\ldots)$. Define
the map $\iota: \Gal(K/\dbQ)\to\dbF_2^{\infty}$ by $\iota(\phi)=(a_1,a_2,\ldots)$ where $a_i=0$ if $\phi(\sqrt{d_i})=\sqrt{d_i}$ and
$a_i=1$ if $\phi(\sqrt{d_i})=-\sqrt{d_i}$. Prove that $\iota$ is a group isomorphism.
\skv


{\bf 6.} In each part of this problem we are given a Galois extension $W/F$ and a closed subgroup $H$ of $G=\Gal(W/F)$. Find (with proof)
the fixed $L$ of $H$ (equivalently, find the unique field $L$ such that $\Gal(W/L)=H$). In each part we also fix a prime $p$.
\begin{itemize}
\item[(a)] $F$ is a finite field, $W=\overline F$ and $H=\prod\limits_{q\neq p}\dbZ_q$. (Recall that in this case
$G$ is canonically isomorphic to $\widehat{\dbZ}=\prod\limits_{q} \dbZ_q$.
\item[(b)] $F=\dbQ$, $W=\dbQ(\{\zeta_{n}: n\in\dbN\})$ where $\zeta_{n}$ is a primitive $n^{\rm th}$ root of unity and 
$H=\prod\limits_{q\neq p}\dbZ_q^{\times}$. (Recall that in this case $G$ is canonically isomorphic to $\widehat{\dbZ}^{\times}=\prod\limits_q \dbZ_q^{\times})$ 
\item[(c)] Let $F$ and $W$ be as in (b), and let $H$ be the product of $\prod\limits_{q\neq p}\dbZ_q^{\times}$ (the subgroup from (b))
and the subgroup $(\dbZ_p^{\times})^2$ consisting of all squares in $\dbZ_p^{\times}$. (As stated in class, if $p$ is odd, then
$\dbZ_p^{\times}\cong \dbZ/(p-1)\dbZ\times \dbZ_p$, so $(\dbZ_p^{\times})^2$ has index $2$ in $\dbZ_p^{\times}$ and
$\dbZ_2^{\times}\cong \dbZ/2\dbZ\times \dbZ_2$, so $(\dbZ_2^{\times})^2$ has index $4$ in $\dbZ_2^{\times}$).
\end{itemize} 
{\bf Hint:} Analyzing the proofs of the isomorhisms $\Gal(W/F)\cong \widehat{\dbZ}$ in (a) and $\Gal(W/F)\cong \widehat{\dbZ}^{\times}$ in (b) and (c) will probably be helpful for all parts. In (c) you may be you need to use some facts not discussed in Algebra-II to rigorously prove the answer.
\end{document}
