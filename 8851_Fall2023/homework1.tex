\documentclass[12pt]{amsart}

\usepackage{amsmath}
\usepackage{amssymb}
\usepackage{amsthm}
%\usepackage{psfig}

\newtheorem* {Theorem}    {Theorem}
\newtheorem* {Lemma}    {Lemma}


\begin{document}
 \pagenumbering{gobble}
\baselineskip=16pt
\textheight=8.5in
\textwidth=6in
%\parindent=0pt 
\def\sk {\hskip .5cm}
\def\skv {\vskip .08cm}
\def\cos {\mbox{cos}}
\def\sin {\mbox{sin}}
\def\tan {\mbox{tan}}
\def\intl{\int\limits}
\def\lm{\lim\limits}
\newcommand{\frc}{\displaystyle\frac}
\def\xbf{{\mathbf x}}
\def\fbf{{\mathbf f}}
\def\gbf{{\mathbf g}}

\def\dbA{{\mathbb A}}
\def\dbB{{\mathbb B}}
\def\dbC{{\mathbb C}}
\def\dbD{{\mathbb D}}
\def\dbE{{\mathbb E}}
\def\dbF{{\mathbb F}}
\def\dbG{{\mathbb G}}
\def\dbH{{\mathbb H}}
\def\dbI{{\mathbb I}}
\def\dbJ{{\mathbb J}}
\def\dbK{{\mathbb K}}
\def\dbL{{\mathbb L}}
\def\dbM{{\mathbb M}}
\def\dbN{{\mathbb N}}
\def\dbO{{\mathbb O}}
\def\dbP{{\mathbb P}}
\def\dbQ{{\mathbb Q}}
\def\dbR{{\mathbb R}}
\def\dbS{{\mathbb S}}
\def\dbT{{\mathbb T}}
\def\dbU{{\mathbb U}}
\def\dbV{{\mathbb V}}
\def\dbW{{\mathbb W}}
\def\dbX{{\mathbb X}}
\def\dbY{{\mathbb Y}}
\def\dbZ{{\mathbb Z}}

\def\la{{\langle}}
\def\ra{{\rangle}}
\def\Ker{{\rm Ker}}
\def\rk{{\rm rk}}
\def\summ{{\sum\limits}}

\bf\centerline{Math 8851. Homework \#1. To be completed by 5pm on Fri, Sep 8}\rm
\vskip .1cm
Below [DDMS] refers to the book `Analytic pro-$p$ groups', 2nd edition by Dixon, du Sautoy, Mann and Segal.
\skv
{\bf 1.} Prove the implication (P2) $\Rightarrow$ (P1) in the statement about the equivalence of 4 definitions of a profinite group.
That is, prove that if $G=\projlim\limits_{i\in I}F_i$ for some inverse system of finite groups $\{F_i\}_{i\in I}$, then 
$G$ is a closed subgroup of $\projlim\limits_{i\in I}F_i$. 
\skv
{\bf 2.} (Problem 3(ii) after Chapter 1 in [DDMS], page 31). Prove the universal property of the profinite completion:
Let $H$ be an abstract group, $\widehat H$ its profinite completion and $j:H\to \widehat H$ the canonical map. 
Prove that for any profinite group $G$ and any homomorphism of abstract groups $\theta:H\to G$, there exists a unique continuous homomorphism $\widehat\theta: \widehat H\to G$ such that $\widehat\theta j=\theta$.
\skv
{\bf Hint:} As is typical for a proof of a universal property, it is the existence part that requires more work.
First prove this for finite $G$, in which case one can describe $\widehat\theta$ quite explicitly. Then use the universal property of inverse limits to extend the result to the general case.
\skv

{\bf 3.} Let $G$ be a profinite group and $p$ a prime number. Prove that the following 4 conditions on $G$ are equivalent:
\begin{itemize}
\item[(i)] $G\cong \projlim_{i\in I}P_i$ for some inverse system of finite $p$-groups $\{P_i\}_{i\in I}$
\item[(ii)] $G$ has a base of neighborhoods of $1$ consisting of open subgroups of $p$-power index
\item[(iii)] Every open subgroup of $G$ has $p$-power index
\item[(iv)] Every continuous finite quotient of $G$ is a $p$-group. Here by a continuous quotient of $G$ we mean the image of a continuous homomorphism from $G$ to another topological group, and finite groups are considered as discrete groups.
\end{itemize}
\skv
{\bf 4.} Let $p$ be a prime, and let $\dbZ_p$ be the ring of $p$-adic integers, as defined in class, that is, 
$\dbZ_p=\projlim\limits_{k\in \dbZ}\dbZ/p^k\dbZ$. Prove that
\begin{itemize}
\item[(a)] $\dbZ_p$ is a local ring whose unique maximal ideal is $p\dbZ_p$;
\item[(b)] Every nonzero ideal of $\dbZ_p$ is equal to $p^k\dbZ_p$ for some $k\in \dbN$ and $\dbZ_p/p^k\dbZ_p\cong \dbZ/p^k\dbZ$.
\end{itemize}
You may use the representation of $\dbZ_p$ as ``power series in $p$'' discussed in class.
\skv
{\bf 5.} Let $p$ be a prime and $n\in\dbN$. Let $G=SL_n(\dbZ_p)$ and $H=SL_n^1(\dbZ_p)$, the first congruence subgroup of $SL_n(\dbZ_p)$,
defined as the kernel of the natural projection map $SL_n(\dbZ_p)\to SL_n(\dbZ_p/p\dbZ_p)$, that is,
$$SL_n^1(\dbZ_p)=\{g\in G : g\equiv 1\mod p\dbZ_p\}.$$ Prove that $G$ is a profinite group and $H$ is a pro-$p$ group.
\skv
{\bf Note:} The topology on $G$ and $H$ is induced from $Mat_{n}(\dbZ_p)$ where we identify $Mat_{n}(\dbZ_p)$ with $\dbZ_p^{n^2}$
endowed with the product topology.
\skv
{\bf 6.} Let $\widehat\dbZ$ be the profinite completion of $\dbZ$ considered as a ring. Prove the isomorphism
$$\dbZ\cong \prod_{p}\dbZ_p$$
where $p$ ranges over all primes.
\skv
{\bf 7.} Let $G$ be an abstract group and $\Lambda$ a family of normal finite index subgroups of $G$ closed under finite intersection.
Recall that in this case there exists a topology on $G$ (called the $\Lambda$-topology) which turns $G$ into a topological group and where
$\Lambda$ is a base of neighborhoods of identity.

Assume now that the $\Lambda$-topology admits a countable base of neighborhoods of $1$. Then there exists such a base $\{U_k\}_{k=1}^{\infty}$
where $U_1\supseteq U_2\supseteq \cdots$ and each $U_i\in\Lambda$. Define the pseudometric $d$ on $G$ as follows:
$$d(g,h)=\left\{
\begin{array}{ll}
0 & \mbox{ if }g^{-1}h\in U_k \mbox{ for all } k\in\dbN\\
\frac{1}{k} &  \mbox{ if }g^{-1}h\not \in U_k \mbox{ and } k \mbox{ is smallest with this property.}
\end{array}
\right.
$$
\begin{itemize}
\item[(a)] Prove that $d$ is indeed a pseudo-metric.
\item[(b)] Let $\widetilde G$ be the completion of $G$ with respect to $d$ (the elements of $\widetilde G$ are equivalence classes
of the Cauchy sequences of elements of $G$). Prove that if we define a binary operation $\cdot$ on $\widetilde G$
by $[x_n]\cdot[y_n]=[x_n y_n]$ (here $[a_n]$ is the equivalence class of a Cauchy sequence $\{a_n\}$), then $\cdot$ is well defined
and $(\widetilde G,\cdot)$ is a topological group.
\item[(c)] Recall that the $\Lambda$-completion of $G$, denoted by $\widehat G_{\Lambda}$, is defined by
${\widehat G}_{\Lambda}=\projlim\limits_{U\in \Lambda} G/U$. Prove that ${\widehat G}_{\Lambda}\cong (\widetilde G,\cdot)$ as topological groups.

\noindent{\bf Hint:} It is probably helpful to first show that ${\widehat G}_{\Lambda}\cong \projlim\limits_{k\in\dbN} G/U_k$.
\end{itemize}
\end{document}



