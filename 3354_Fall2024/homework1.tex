\documentclass[11pt]{amsart}

\usepackage{amsmath}
\usepackage{amssymb}
\usepackage{amsthm}
\usepackage{url}
\usepackage{hyperref}
%\usepackage{psfig}

\begin{document}
\baselineskip=16pt
\textheight=8.5in
\parindent=0pt 
\def\sk {\hskip .5cm}
\def\skv {\vskip .08cm}
\def\cos {\mbox{cos}}
\def\sin {\mbox{sin}}
\def\tan {\mbox{tan}}
\def\intl{\int\limits}
\def\lm{\lim\limits}
\newcommand{\frc}{\displaystyle\frac}
\def\xbf{{\mathbf x}}
\def\fbf{{\mathbf f}}
\def\gbf{{\mathbf g}}

\def\dbA{{\mathbb A}}
\def\dbB{{\mathbb B}}
\def\dbC{{\mathbb C}}
\def\dbD{{\mathbb D}}
\def\dbE{{\mathbb E}}
\def\dbF{{\mathbb F}}
\def\dbG{{\mathbb G}}
\def\dbH{{\mathbb H}}
\def\dbI{{\mathbb I}}
\def\dbJ{{\mathbb J}}
\def\dbK{{\mathbb K}}
\def\dbL{{\mathbb L}}
\def\dbM{{\mathbb M}}
\def\dbN{{\mathbb N}}
\def\dbO{{\mathbb O}}
\def\dbP{{\mathbb P}}
\def\dbQ{{\mathbb Q}}
\def\dbR{{\mathbb R}}
\def\dbS{{\mathbb S}}
\def\dbT{{\mathbb T}}
\def\dbU{{\mathbb U}}
\def\dbV{{\mathbb V}}
\def\dbW{{\mathbb W}}
\def\dbX{{\mathbb X}}
\def\dbY{{\mathbb Y}}
\def\dbZ{{\mathbb Z}}

\def\la{{\langle}}
\def\ra{{\rangle}}
\def\summ{{\sum\limits}}


\bf\centerline{Homework \#1. Due on Friday, September 6th, 11:59pm on Canvas}\rm
\vskip .1cm

\bf\centerline{Reading: }\rm
\skv
1. For this assignment: Online lectures: 1, the beginning of 2 (Example 2.1) and the beginning of 3 (subsection 3.1). From Hungerford:
the beginning of Appendix B (pp. 509-512) and the beginning of Appendix C.

2. For next week's classes: Online lectures 3 and 4. I am planning to skip the content of Lecture 2 for now, but it would not hurt to go over it as well. From Hungerford: Appendix C, 1.1 and 1.2.
\skv
Online lectures are currently posted on the Spring 2016 webpage
\skv

\skv
\centerline{\url{https://m-ershov.github.io/3354_Spring2016/}}
\skv


\skv
\bf\centerline{Problems: }\rm
\skv
\bf{Problem 1: }\rm Let $R$ be a commutative ring with $1$.
Prove the following equalities using only the ring axioms and 
results proved  in class or online lectures.
\begin{itemize}
\item[(a)] $-(xy)=(-x)y$ for all $x,y\in R$ 
\item[(b)] $(-1)(-1)=1$ 
\item[(c)] $(-x)(-y)=xy$ for all $x,y\in R$
\item[(d)] $x(y-z)=xy-xz$ for all $x,y,z\in R$ 
\end{itemize}
\bf{Hint: }\rm Additive cancellation law (proved in lecture 1) can be used to solve
many questions of this type as follows. Suppose that we want to prove inequality of
the from $a=b$. By additive cancellation law, if we prove that $a+c=b+c$ for some $c\in R$,
we can conclude that $a=b$. Note that the implication would work for any $c$, so $c$
is for us to choose. The idea is to choose $c$ in such a way that both expressions $a+c$
and $b+c$ can be simplified (using ring axioms) so that after simplification it becomes
easy to prove that $a+c=b+c$.
\skv
Recall that by definition $x-y=x+(-y)$.
\skv
\bf{Problem 2: }\rm Let $F$ be a field, and suppose that
$xy=0$ for some $x,y\in F$. Prove that $x=0$ or $y=0$.
\bf{Hint: }\rm Consider two cases: $x=0$ (in this case there is nothing
to prove) and $x\neq 0$. Recall that in a field every nonzero
element has multiplicative inverse.
\skv
\skv
\bf{Problem 3: }\rm Let $X$ be any set, and let $R=\mathcal P(X)$ (the power set of $X$), that is,
$R$ is the of all subsets of $X$. As in online lecture 2, define addition $+$ and multiplication $\cdot$ on $R$ by setting $A\cdot B=A\cap B$ (intersection) 
and $A+B=(A\cup B)\setminus (A\cap B)=(A\setminus B)\cup (B\setminus A)$ (symmetric difference =`exclusive or') 
for arbitrary $A,B\in R$ (that is, for arbitrary  $A,B\subseteq X$).
Prove that $R$ with these operations is a commutative ring with $1$. 

{\bf Note:} Multiplication axioms (M0)-(M3) are checked in online lecture 3, so you only need
to check the addition axioms (A0)-(A4) and distributivity. You may want to read the beginning of Appendix~B in Hungerford before
doing this problem.


{\bf Hint:} To check associativity of addition ($(A+B)+C=A+(B+C)$), take an arbitrary element $x\in X$,
and consider $8$ cases: case 1 ($x\in A$, $x\in B$, $x\in C$), case 2 ($x\in A$, $x\in B$, $x\not \in C$)
etc. In each case show that $x$ belongs to both $(A+B)+C$ and $A+(B+C)$ or does not belong to either of
those sets. 
\skv
\skv
\skv
\bf{Problem 4: }\rm Prove by induction that the following equalities hold for any $n\in\dbN$:
\begin{itemize}
\item[(a)] $1^2+2^2+\ldots+ n^2=\frac{n(n+1)(2n+1)}{6}$ 
\item[(b)] $a+ar+ar^2+\ldots+ar^{n-1}=a\frac{1-r^n}{1-r}$ where $a,r\in\dbR$ and $r\neq 1$
\end{itemize}

\skv
\bf{Problem 5: }\rm Consider the following ``proof'' by induction:
For each $n\in\dbN$ let $P(n)$ be the statement
$$\sum_{i=0}^n 2^i= 2^{n+1}. \eqno (***)$$
{\bf Claim: } $P(n)$ is true for all $n\in\dbN$.
\skv
\skv{\it Proof: } ``$P(n-1)\Rightarrow P(n)$.'' Assume that $P(n-1)$ is true
for some $n\in\dbN$. Then $\summ_{i=0}^{n-1} 2^i = 2^n.$ Adding $2^n$
to both sides, we get $\summ_{i=0}^{n-1} 2^i + 2^n =2^n+2^n,$
whence $\summ_{i=0}^{n} 2^i=2^{n+1}$, which is precisely $P(n)$.
Thus, $P(n)$ is true.
\skv
By the principle of mathematical induction, $P(n)$ is true for all $n$. $\square$
\begin{itemize}
\item[(a)] Show that the statement $P(n)$ is false (it is actually false for any $n$).
\item[(b)] Explain why the above ``proof'' does not contradict the principle of
mathematical induction, that is, find a mistake in the above ``proof''
(Hint: the mistake is in the general logic).
\end{itemize}

\skv
\bf{Problem 6: }\rm In online lecture~3 it is proved that for every $n\in\dbN$
there exist $a_n,b_n\in\dbZ$ such that $(1+\sqrt{2})^n=a_n+b_n\sqrt{2}$. Moreover,
it is shown that such $a_n$ and $b_n$ satisfy the following recursive relations:
$a_1=b_1=1$ and $a_{n+1}=a_n+2b_n$, $b_{n+1}=a_n+b_n$ for all $n\in\dbN$.
\begin{itemize}
\item[(a)] Use the above recursive formulas and mathematical induction to prove that
$a_n^2-2b_n^2=(-1)^n$ for all $n\in\dbN$.
\item[(b)] Prove that for all $n\in\dbN$ there exist $c_n,d_n\in\dbZ$ such that
$(1+\sqrt{3})^n=c_n+d_n\sqrt{3}$.
\item[(c)] (bonus) Find a simple formula relating $c_n$ and $d_n$ (similar to the one in (a))
and prove it. 
\end{itemize}
\end{document}
