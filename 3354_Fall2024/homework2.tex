\documentclass[11pt]{amsart}

\usepackage{amsmath}
\usepackage{amssymb}
\usepackage{amsthm}
\usepackage{url}
\usepackage{hyperref}
%\usepackage{psfig}

\begin{document}
\baselineskip=16pt
\textheight=8.5in
\parindent=0pt 
\def\sk {\hskip .5cm}
\def\skv {\vskip .08cm}
\def\cos {\mbox{cos}}
\def\sin {\mbox{sin}}
\def\tan {\mbox{tan}}
\def\intl{\int\limits}
\def\lm{\lim\limits}
\newcommand{\frc}{\displaystyle\frac}
\def\xbf{{\mathbf x}}
\def\fbf{{\mathbf f}}
\def\gbf{{\mathbf g}}

\def\dbA{{\mathbb A}}
\def\dbB{{\mathbb B}}
\def\dbC{{\mathbb C}}
\def\dbD{{\mathbb D}}
\def\dbE{{\mathbb E}}
\def\dbF{{\mathbb F}}
\def\dbG{{\mathbb G}}
\def\dbH{{\mathbb H}}
\def\dbI{{\mathbb I}}
\def\dbJ{{\mathbb J}}
\def\dbK{{\mathbb K}}
\def\dbL{{\mathbb L}}
\def\dbM{{\mathbb M}}
\def\dbN{{\mathbb N}}
\def\dbO{{\mathbb O}}
\def\dbP{{\mathbb P}}
\def\dbQ{{\mathbb Q}}
\def\dbR{{\mathbb R}}
\def\dbS{{\mathbb S}}
\def\dbT{{\mathbb T}}
\def\dbU{{\mathbb U}}
\def\dbV{{\mathbb V}}
\def\dbW{{\mathbb W}}
\def\dbX{{\mathbb X}}
\def\dbY{{\mathbb Y}}
\def\dbZ{{\mathbb Z}}

\def\la{{\langle}}
\def\ra{{\rangle}}
\def\summ{{\sum\limits}}


\bf\centerline{Homework \#2. Due on Thursday, September 12th, 11:59pm on Canvas}\rm
\vskip .1cm

\bf\centerline{Reading: }\rm
\skv
1. For this assignment: Online lectures 2 and 3. From Hungerford: Appendix C, 1.1 and 1.2.

2. For next week's classes: Online lectures 4, 5 and beginning of 6. From Hungerford: 1.2, 1.3 and the beginning of 2.1.
\skv
Online lectures are currently posted on the Spring 2016 webpage
\skv

\skv
\centerline{\url{https://m-ershov.github.io/3354_Spring2016/}}
\skv


\skv
\bf\centerline{Problems: }\rm
\skv
\skv
\bf{Problem 1: }\rm
Given $n,k\in\mathbb Z$ with
$0\leq k\leq n$, define the binomial coefficient
$n\choose k$ by $${n\choose k}=\frac{n!}{k!(n-k)!}$$
(recall that $0!=1$).
\begin{itemize}
\item[(a)] Prove that ${n\choose k}={{n-1}\choose k} + {{n-1}\choose {k-1}}$
for any $1\leq k< n$ (direct computation).

\item[(b)] Now prove the binomial theorem: for every $a,b\in\mathbb R$
and $n\in\mathbb N$,
$$(a+b)^n=\sum_{k=0}^n {n\choose k} a^{n-k} b^{k}=
{n\choose 0} a^n +{n\choose 1} a^{n-1}b+\ldots+{n\choose {n-1}} a b^{n-1}+{n\choose n} b^n.$$
\bf{Hint: }\rm Use induction on $n$. For the induction step write\newline
$(a+b)^{n+1}=(a+b)^{n}\cdot (a+b)$ and use part (a).
\end{itemize}
\skv
{\bf Note:} In Problems 2 and 3 below we assume that $R$ is an ordered ring (see definition in online Lecture~2) and $R_{>0}$ is its set of positive elements. Recall that if $R$ is an ordered ring, given $x,y\in R$, we write $x>y$ if $x-y\in R_{>0}$. Also, by definition $x<y$ is the same as $y>x$, $x\geq y$ means
$x>y$ or $x=y$ and $x\leq y$ means $x<y$ or $x=y$. In online Lecture~2 (Example 2.4) it is proved that $>$ is a transitive relation:
if $x>y$ and $y>z$, then $x>z$.

\skv
{\bf Problem 2:} Let $R$ be an ordered ring. Prove the following basic properties of the relation $>$:
\begin{itemize}
\item[(a)] ($<$ is anti-symmetric): there are no elements $x,y\in R$ such that $x<y$ and $y<x$.
\item[(b)] If $x>y$, then $x+z>y+z$ for all $z\in R$, that is, one can add a fixed element to both sides of an inequality. Recall
that we argued in Lecture~1 that the corresponding property for equalites holds simply because $+$ is a binary operation (and does not
use any axioms of addition). Explain why the situation is different for inequalities.
\item[(c)] If $x>y$ and $z>0$, then $xz>yz$.
\end{itemize}

\skv
{\bf Problem 3:}
\begin{itemize}
\item[(a)] Let $R$ be an ordered ring. Prove that $x^2>0$ for every nonzero $x\in R$.
{\bf Hint:} Consider two cases.
\item[(b)] Use (a) to prove that $\dbC$ (complex numbers) is not an ordered ring (no matter how we try
to define the set of positive elements).
\end{itemize}
\skv
{\bf Problem 4:} 
\begin{itemize}
\item[(a)] 
Use the induction axiom (called the Induction Property on page 4 of online Lecture~3) to
prove (formally) that $n\geq 1$ for all $n\in\dbZ_{>0}$
\item[(b)] Let $x,y\in\dbZ$ with $x>y$. Use (a) to prove that $x\geq y+1$.
\end{itemize}
\skv
{\bf Problem 5 (bonus)}: Deduce the well-ordering principle (axiom (Z3) on page 3 of online Lecture~3) from the induction axiom.

{\bf Hint:} Let $Q(n)$ be some property of subsets of $\dbZ_{>0}$, and let $P(n)$ be the statement
``Every subset of $\dbZ_{>0}$ satisfying $Q(n)$ has the smallest element.'' You are free to define
$Q(n)$ in any way you like. Your goal is to define $Q(n)$ in such a way that
\begin{itemize}
\item[(i)] Every non-empty subset of $\dbZ_{>0}$ satisfies $Q(n)$ for some $n$.
\item[(ii)] You can prove $P(1)$ directly.
\item[(iii)] You can prove the implication $P(n)\Rightarrow P(n+1)$ for all $n$.
\end{itemize}

\skv
\bf{Problem 6: }\rm Let $a,b,c\in\dbZ$ such that $c\mid a$ and $c\mid b$. Prove {\it directly
from definition of divisibility} that $c\mid (ma+nb)$ for any $m,n\in\dbZ$ (do not refer
to any divisibility properties proved in class).
\skv
\bf{Problem 7: }\rm Let $a,b,c\in\dbZ$ such that $c\mid ab$. Is it always true that $c\mid a$ or $c\mid b$?
If the statement is true for all possible values of $a,b,c$, prove it; otherwise give a counterexample.
\end{document}
