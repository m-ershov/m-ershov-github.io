\documentclass[11pt]{amsart}

\usepackage{amsmath}
\usepackage{amssymb}
\usepackage{amsthm}
\usepackage{url}
\usepackage{hyperref}
%\usepackage{psfig}

\begin{document}
\baselineskip=16pt
\textheight=8.5in
%\parindent=0pt 
\def\sk {\hskip .5cm}
\def\skv {\vskip .08cm}
\def\cos {\mbox{cos}}
\def\sin {\mbox{sin}}
\def\tan {\mbox{tan}}
\def\intl{\int\limits}
\def\lm{\lim\limits}
\newcommand{\frc}{\displaystyle\frac}
\def\xbf{{\mathbf x}}
\def\fbf{{\mathbf f}}
\def\gbf{{\mathbf g}}

\def\dbA{{\mathbb A}}
\def\dbB{{\mathbb B}}
\def\dbC{{\mathbb C}}
\def\dbD{{\mathbb D}}
\def\dbE{{\mathbb E}}
\def\dbF{{\mathbb F}}
\def\dbG{{\mathbb G}}
\def\dbH{{\mathbb H}}
\def\dbI{{\mathbb I}}
\def\dbJ{{\mathbb J}}
\def\dbK{{\mathbb K}}
\def\dbL{{\mathbb L}}
\def\dbM{{\mathbb M}}
\def\dbN{{\mathbb N}}
\def\dbO{{\mathbb O}}
\def\dbP{{\mathbb P}}
\def\dbQ{{\mathbb Q}}
\def\dbR{{\mathbb R}}
\def\dbS{{\mathbb S}}
\def\dbT{{\mathbb T}}
\def\dbU{{\mathbb U}}
\def\dbV{{\mathbb V}}
\def\dbW{{\mathbb W}}
\def\dbX{{\mathbb X}}
\def\dbY{{\mathbb Y}}
\def\dbZ{{\mathbb Z}}

\def\la{{\langle}}
\def\ra{{\rangle}}
\def\summ{{\sum\limits}}


\bf\centerline{Homework \#5. Due on Thursday, October 3rd, 11:59pm on Canvas}\rm
\vskip .1cm

\bf\centerline{Reading: }\rm
\skv
1. For this assignment: Online lecture 8 and 9. 
From Hungerford: 2.3, 4.1 and 4.2.

2. For next week's classes: Online lecture 10. From Hungerford: 4.2 and parts of 4.3, 5.1 and 7.1. 
\skv
Online lectures are currently posted on the Spring 2016 webpage
\skv

\skv
\centerline{\url{https://m-ershov.github.io/3354_Spring2016/}}
\skv


\skv
\bf\centerline{Problems: }\rm
\skv
\bf{Preface to problem 1: }\rm
Recall from the previous homework that for 
$n,k\in\mathbb Z$ with $0\leq k\leq n$, the binomial coefficient
$n\choose k$ is defined by ${n\choose k}=\frac{n!}{k!(n-k)!}$
(where $0!=1$). Also recall the binomial theorem:
for every $a,b\in\mathbb R$
and $n\in\mathbb N$,
$$(a+b)^n=\sum_{k=0}^n {n\choose k} a^{n-k} b^{k}=
{n\choose 0} a^n +{n\choose 1} a^{n-1}b+\ldots+{n\choose {n-1}} a b^{n-1}+{n\choose n} b^n.$$
Note that $n\choose k$ is always an integer -- this is not obvious from definition,
but it is (almost) obvious from the binomial theorem.
\skv
\bf{Problem 1: }\rm Suppose that $p$ is prime and $0<k<p$. Prove that
$p\mid {p\choose k}$. \bf{Hint: }\rm First prove the following lemma:
Suppose that $n,m\in\mathbb Z$, $p$ is prime, $m\mid n$, $p\mid n$ and
$p\nmid m$. Then $p\mid \frac{n}{m}$ (this follows from Euclid's lemma). 
\skv
\bf{Problem 2: }\rm In both parts of this problem $p$ is a prime number.
\begin{itemize}
\item[(a)] Prove the little Fermat's theorem: $n^p\equiv n\mod p$ for any $n\in\mathbb N$.
\item[(b)] Reformulate (a) as an equality in $\mathbb Z_p$. Your reformulation should be of the form
``for all $x\in\mathbb Z_p$ we have $f(x)=0$ where $f:\mathbb Z_p\to\mathbb Z_p$ is a certain explicit function''
 \end{itemize}
\bf{Hint for (a): }\rm Fix $p$ and use induction on $n$. For the  induction
step use the result of Problem~1.
\skv
\bf{Problem 3: }\rm Let $X=\mathbb R^2$ (the set of ordered pairs of real numbers) and define a relation $\sim$ on $X$ by
$$(x_1,y_1)\sim (x_2,y_2)\iff x_1+y_1=x_2+y_2.$$
\begin{itemize}
\item[(a)] Prove that $\sim$ is an equivalence relation. 
\item[(b)] Describe equivalence classes with respect to $\sim$. {\bf Hint:} there is a very easy geometric description if you think of elements of $X$ as points on the Euclidean plane.
\end{itemize}
 \skv
\bf{Problem 4: }\rm Define a relation $\sim$ on $\dbZ$ by
$$x\sim y\iff x^3\equiv y^3\mod 4.$$
\begin{itemize}
\item[(a)] Prove that $\sim$ is an equivalence relation. 
\item[(b)] Find the number of equivalence classes with respect to $\sim$ and describe (explicitly) each class.
\end{itemize}
{\bf Hint for (b):} The equivalence classes with respect to $\sim$ are closely related to congruence classes mod $4$. Once you figure out the relationship (and why it holds), it is fairly easy to finish the problem.
\skv
\bf{Problem 5: }\rm Let $R$ be a commutative ring with $1$ and $R[x]$ the ring of polynomials with coefficients in $R$. Prove the following properties of the degree function:
\begin{itemize}
\item[(a)] $\deg(f+g)\leq \max\{\deg(f),\deg(g)\}$ for all $f,g\in R[x]$
\item[(b)] $\deg(f+g) = \max\{\deg(f),\deg(g)\}$ for all $f,g\in R[x]$ such that $\deg(f)\neq \deg(g)$
\item[(c)] $\deg(fg)\leq \deg(f)+\deg(g)$
\item[(d)] If $R$ has no zero divisors (such $R$ is called a domain), then $\deg(fg)=\deg(f)+\deg(g)$
\item[(e)] Find a specific $n$ and $f\in \dbZ_n[x]$ such that $f$ is non-constant (that is, $\deg(f)>0$), but $f$ is invertible.
{\bf Hint:} Part (d) (and what we proved earlier) yields certain restrictions on the possible values of $n$.
 \end{itemize}
\skv
\bf{Problem 6: }\rm Let $f(x)=x^4-1$ and $g(x)=x^2+3x+1$, and consider $f$ and $g$ either as polynomials in $\mathbb Z_5[x]$ or 
$\mathbb Z_7[x]$ (in both cases the coefficients of $f$ and $g$ should be interpreted as congruence classes mod $5$ and mod $7$,
respectively). In both cases do the following:
\begin{itemize}
\item[(a)] divide $f$ by $g$ with remainder
\item[(b)] Compute $gcd(f,g)$
\item[(c)] Find explicit polynomials $u(x)$ and $v(x)$ such that $gcd(f(x),g(x))=f(x)u(x)+g(x)v(x)$. 
 \end{itemize}
{\bf Note:} We will talk about gcd in class on Monday, Sep 30 (see also 4.2 in Hungerford), but the following information should be sufficient
to solve this problem. If $F$ is a field and $f(x),g(x)\in F[x]$, not both $0$, $gcd(f(x),g(x))$ is defined to be the MONIC polynomial
of largest possible degree which divides both $f(x)$ and $g(x)$ (a polynomial is called monic if its leading coefficient is $1$).
We will prove that $gcd(f(x),g(x))$ always exists and is unique.

Similarly to $\mathbb Z$, one can prove that $gcd(f(x),g(x))$ is the unique monic polynomial of smallest possible degree representable
as $f(x)u(x)+g(x)v(x)$ for some $u(x),v(x)\in F[x]$. Moreover, one can find the gcd itself as well as $u$ and $v$ from the above representation
using the Euclidean algorithm essentially in the same way as for $\mathbb Z$. The only real difference is that while in the case of $\mathbb Z$,
gcd is the last nonzero remainder in the first part of the Euclidean algorithm, in the polynomial case the last nonzero remainder is a constant multiple of the gcd (to get the actual gcd, one just needs to divide that remainder by its leading coefficient).
\end{document}


