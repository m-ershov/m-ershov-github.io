\documentclass[12pt]{amsart}

\usepackage{amsmath}
\usepackage{amssymb}
\usepackage{amsthm}
\usepackage{url}
\usepackage{hyperref}

%\usepackage{psfig}

\begin{document}
\baselineskip=16pt
%\textheight=9.6in
%\parindent=0pt
\def\sk {\hskip .5cm}
\def\skv {\vskip .12cm}
\def\cos {\mbox{cos}}
\def\sin {\mbox{sin}}
\def\tan {\mbox{tan}}
\def\intl{\int\limits}
\def\lm{\lim\limits}
\newcommand{\frc}{\displaystyle\frac}
\def\xbf{{\mathbf x}}
\def\fbf{{\mathbf f}}
\def\gbf{{\mathbf g}}

\def\dbA{{\mathbb A}}
\def\dbB{{\mathbb B}}
\def\dbC{{\mathbb C}}
\def\dbD{{\mathbb D}}
\def\dbE{{\mathbb E}}
\def\dbF{{\mathbb F}}
\def\dbG{{\mathbb G}}
\def\dbH{{\mathbb H}}
\def\dbI{{\mathbb I}}
\def\dbJ{{\mathbb J}}
\def\dbK{{\mathbb K}}
\def\dbL{{\mathbb L}}
\def\dbM{{\mathbb M}}
\def\dbN{{\mathbb N}}
\def\dbO{{\mathbb O}}
\def\dbP{{\mathbb P}}
\def\dbQ{{\mathbb Q}}
\def\dbR{{\mathbb R}}
\def\dbS{{\mathbb S}}
\def\dbT{{\mathbb T}}
\def\dbU{{\mathbb U}}
\def\dbV{{\mathbb V}}
\def\dbW{{\mathbb W}}
\def\dbX{{\mathbb X}}
\def\dbY{{\mathbb Y}}
\def\dbZ{{\mathbb Z}}

\def\la{{\langle}}
\def\ra{{\rangle}}

\def\Aut{{\rm Aut}}
\def\End{{\rm End}}
\def\Inn{{\rm Inn}}
\def\Gal{{\rm Gal}}
\def\Ker{{\rm Ker}}
\def\Im{{\rm Im\,}}
\def\phi{{\varphi}}

\bf\centerline{Algebra-II, Spring 2021. Midterm \#2}\rm
\skv
\bf\centerline{due by 11:59pm on Friday Apr 22nd}\rm
\vskip .3cm
{\bf Directions: } Provide complete arguments
(do not skip steps). State clearly any result you are referring to. Partial credit for
incorrect solutions, containing steps in the right direction, may be given.
\vskip .1cm

{\bf Rules: } You are not allowed to discuss midterm problems with each other.
You may ask me any questions about the problems (e.g. if the formulation is unclear),
but as a rule I will only provide minor hints. You may freely use the following resources:
\begin{itemize}
\item[(i)] the book by Dummit and Foote
\item[(ii)] your class notes (including notes from 7751)
\item[(iii)] your previous assignments (homeworks and midterms)
\item[(iv)] any materials posted on the Math 7751/7752 collab sites and any materials posted on \url{http://people.virginia.edu/~mve2x/}
\end{itemize}

The use of any other resources is prohibited and will be considered a violation of the UVA honor code.


\skv
{\bf Scoring:} The exam contains 5 problems, all of which will count towards your score. The first 4 problems are worth 10 points, and the last problem
is worth 12 points. Thus, the maximal possible total is 52, but the score of 50 will count as 100\%.


\skv

{\bf Problem 1:} Let $F$ be a field.
\begin{itemize}
\item[(a)] Let $f(x)\in F[x]$ be a nonzero polynomial and $K/F$ a field extension.
Prove that $$F[x]/(f(x))\otimes_F K\cong K[x]/(f(x))$$ as $F$-algebras.
\item[(b)] Let $L/F$ be a finite separable extension. Prove that
there exists a finite extension $K/F$ such that 
$L\otimes_F K\cong \underbrace{K\times\ldots\times K}_{n \mbox{ times }}$
for some $n$.
\end{itemize}
\skv

{\bf Problem 2:} DF, Problem~6 on page 582. Make sure to include all the details.
\skv
\newpage

{\bf Problem 3:} Let $F$ be a field and $f(x)= x^4+ 1\in F[x]$.
\begin{itemize}
\item[(a)] Determine for which characteristic of $F$ $f(x)$ is separable.
\item[(b)]  Assume that $f(x)$ is separable and irreducible over $F$, and let $K$ be the splitting field of $f(x)$ over $F$.  
Determine the Galois group $\Gal(K/F)$ (the answer should be of the form ``$\Gal(K/F)$ is isomorphic to $G$'' where $G$ is a familiar group).
\item[(c)] Suppose that $f(x)$ is irreducible over $F$. Prove first that $F$ is infinite and then that $F$ has characteristic $0$.
\end{itemize}

\skv
{\bf Problem 4:} \rm Let $S=\{n_1,\ldots, n_k\}$ be a finite set of positive integers $\geq 2$
none of which is a perfect square, and let $K=\dbQ(\sqrt{n_1},\ldots, \sqrt{n_k})$. You are NOT allowed to refer to HW\#6.2.
\begin{itemize}
\item[(a)] Prove that $K/\dbQ$ is a Galois extension and $Gal(K/\dbQ)\cong \dbZ_2^m$ for some $m\leq k$
\item[(b)] Now assume that $n_1,\ldots, n_k$ are pairwise coprime. Prove that $K$ contains at least $2^k$
distinct subfields $L$ with $[L:\dbQ]=2$. 
\item[(c)] Keep the extra assumption from (b). Use (b) to prove that $[K:\dbQ]=2^k$.  
\end{itemize}

\skv
{\bf Problem 5:} Let $p$ be an odd prime, $\omega=e^{2\pi i/p}$, $K=\dbQ(\omega)$
and $M$ the unique subfield of $K$ with $[M:\dbQ]=2$. Let $m$ be a generator of $(\dbZ/p\dbZ)^{\times}$
and $\zeta=\sum_{i=0}^{(p-3)/2} \omega^{m^{2i}}$.
At the end of Lecture 18 notes posted on collab it is proved
that $\zeta\in M$ (we did not get to this part during class; read it before starting on this problem).
\begin{itemize}
\item[(a)] Prove that $\zeta\not\in\dbQ$ and deduce that
$M=\dbQ(\zeta)$. {\bf Hint:} Assume that $\zeta\not\in\dbQ$ and deduce that $\deg_{\dbQ}(\omega)<p-1$, thereby
reaching a contradiction.
\item[(b)] Prove by direct computation that if $p=5$, then $M=\dbQ(\sqrt{5})$.
\item[(c)] Let $S$ be the set of all elements of $(\dbZ/p\dbZ)^{\times}$ representable 
as squares. Prove that $\zeta=\sum_{s\in S}\omega^s$.
\item[(d)] Prove that $-1\in S$ if and only if $p\equiv 1\mod 4$
\item[(e)] Prove that $\overline\zeta=\zeta$ if $p\equiv 1\mod 4$ and $\overline\zeta=-1-\zeta$
if $p\equiv 3\mod 4$. Deduce that $M\subset \dbR$ if and only if $p\equiv 1\mod 4$.
\item[(f)] Let $L$ be the unique subfield of $K$ with $[K:L]=2$. As proved in class,
$L=K\cap \dbR$. Now prove that $M\subset \dbR$ if and only if $p\equiv 1\mod 4$ just 
by using this fact and Galois correspondence (do not use an explicit description of $M$).
{\bf Hint:} What is the relationship between the subgroups of $\Gal(K/\dbQ)$ corresponding
to $L$ and $M$ depending on $p$ mod $4$?
\end{itemize} 

\end{document}
