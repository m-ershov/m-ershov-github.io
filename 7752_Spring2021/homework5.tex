\documentclass[12pt]{amsart}

\usepackage{amsmath}
\usepackage{amssymb}
\usepackage{amsthm}
\usepackage{url}
\usepackage{hyperref}
%\usepackage{psfig}

\begin{document}
\baselineskip=15pt
\textheight=8.4in
\parindent=0pt
\def\sk {\hskip .5cm}
\def\skv {\vskip .12cm}
\def\cos {\mbox{cos}}
\def\sin {\mbox{sin}}
\def\tan {\mbox{tan}}
\def\intl{\int\limits}
\def\lm{\lim\limits}
\newcommand{\frc}{\displaystyle\frac}
\def\xbf{{\mathbf x}}
\def\fbf{{\mathbf f}}
\def\gbf{{\mathbf g}}

\def\Ker{{\rm Ker\,}}
\def\phi{\varphi}

\def\dbA{{\mathbb A}}
\def\dbB{{\mathbb B}}
\def\dbC{{\mathbb C}}
\def\dbD{{\mathbb D}}
\def\dbE{{\mathbb E}}
\def\dbF{{\mathbb F}}
\def\dbG{{\mathbb G}}
\def\dbH{{\mathbb H}}
\def\dbI{{\mathbb I}}
\def\dbJ{{\mathbb J}}
\def\dbK{{\mathbb K}}
\def\dbL{{\mathbb L}}
\def\dbM{{\mathbb M}}
\def\dbN{{\mathbb N}}
\def\dbO{{\mathbb O}}
\def\dbP{{\mathbb P}}
\def\dbQ{{\mathbb Q}}
\def\dbR{{\mathbb R}}
\def\dbS{{\mathbb S}}
\def\dbT{{\mathbb T}}
\def\dbU{{\mathbb U}}
\def\dbV{{\mathbb V}}
\def\dbW{{\mathbb W}}
\def\dbX{{\mathbb X}}
\def\dbY{{\mathbb Y}}
\def\dbZ{{\mathbb Z}}

\def\Aut{{\rm Aut}}
\def\exp{{\rm exp}}
\def\det{{\rm det}}
\def\tr{{\rm tr}}

\def\la{{\langle}}
\def\ra{{\rangle}}
\def\rk{{\rm rk}}


\bf\centerline{Homework Assignment \# 5}\rm
\vskip .2cm
{\bf Plan for the week of Mar 1:} Field extensions and algebraic closures (13.1, 13.2, 13.4 and online lectures 14, 15).
\vskip .1cm
Here and in all future assignments ``online'' refers to Algebra-II lectures posted on my Spring 2010 Algebra-II webpage
\vskip .1cm
\centerline{\url{http://people.virginia.edu/~mve2x/7752_Spring2010/}}
\vskip .1cm

%{\bf Note on hints:} All hints are given at the end of the %assignment, each on a separate page.
%Problems (or parts of problems) for which hint is available are %marked with *.

\vskip .3cm

\bf\centerline{Problems, due by 11:59pm on Friday, March 5th.}\rm
\vskip .1cm
\skv
 {\bf Problem 1: } Find the number of distinct conjugacy classes in the group $GL_3(\dbF_2)$ (where $\dbF_2$ is the field with
$2$ elements) and specify one element in each conjugacy class.
\skv
{\bf Problem 2: } \rm 
\begin{itemize}
\item[(a)] Prove that two $3\times 3$ matrices
over some field $F$ are similar if and only if they have the same minimal and characteristic 
polynomials. Give an example showing that this does not hold for $4\times 4$ matrices.

\item[(b)] A matrix $A$ is called idempotent if $A^2=A$. Prove that two idempotent $n\times n$ matrices are similar if and only if they have they same rank. {\bf Hint:} What is the minimal polynomial of an idempotent matrix? How does rank relate to eigenvalue $0$?
\end{itemize}
\skv
{\bf Problem 3: } \rm
Prove that there is no matrix $A\in Mat_{10}(\dbQ)$ satisfying $A^4=-Id$.
\skv
{\bf Problem 4: }\rm DF, Problem~15 on page 500.
\skv
{\bf Problem 5: }\rm DF, Problem~20 on page 501. Also find an explicit $P\in GL_n(F)$ such that $A=PBP^{-1}$ where $A$ and
$B$ are the two matrices in the form (you may choose which one
is $A$ and which one is $B$).
\skv
{\bf Problem 6:} Let $V$ be an $n$-dimensional vector space over some field $F$,
and let $T:V\to V$ be a {\bf nilpotent} $F$-linear map. Prove that $T^n=0$ in two different ways:
\begin{itemize}
\item[(a)] using JCF
\item[(b)] without using JCF or RCF, but instead looking at the sequence of kernels $\{\Ker(T^k)\}_{k=1}^{\infty}$. The idea here
is similar to Problem~2 in HW\#4 of Algebra~1.
\end{itemize}
\skv
{\bf Problem 7:} Given $A\in Mat_n(\dbC)$, define its exponential
$\exp(A)$ as in DF, Problem~41(b) on page 503 (you do not have to prove convergence of the series defining $\exp(A)$).
\begin{itemize}
\item[(a)] Prove that if $A=PBP^{-1}$ for some $P\in GL_n(\dbC)$,
then $\exp(A)=P\,\exp(B)\,P^{-1}$
\item[(b)] Use (a) to prove that $\det(\exp(A))=\exp(\tr(A))$.
 \end{itemize}
\end{document}
