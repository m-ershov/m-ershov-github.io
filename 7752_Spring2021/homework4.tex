\documentclass[12pt]{amsart}

\usepackage{amsmath}
\usepackage{amssymb}
\usepackage{amsthm}
\usepackage{url}
\usepackage{hyperref}
%\usepackage{psfig}

\begin{document}
\baselineskip=15pt
\textheight=8.4in
\parindent=0pt
\def\sk {\hskip .5cm}
\def\skv {\vskip .12cm}
\def\cos {\mbox{cos}}
\def\sin {\mbox{sin}}
\def\tan {\mbox{tan}}
\def\intl{\int\limits}
\def\lm{\lim\limits}
\newcommand{\frc}{\displaystyle\frac}
\def\xbf{{\mathbf x}}
\def\fbf{{\mathbf f}}
\def\gbf{{\mathbf g}}

\def\Ker{{\rm Ker\,}}
\def\phi{\varphi}

\def\dbA{{\mathbb A}}
\def\dbB{{\mathbb B}}
\def\dbC{{\mathbb C}}
\def\dbD{{\mathbb D}}
\def\dbE{{\mathbb E}}
\def\dbF{{\mathbb F}}
\def\dbG{{\mathbb G}}
\def\dbH{{\mathbb H}}
\def\dbI{{\mathbb I}}
\def\dbJ{{\mathbb J}}
\def\dbK{{\mathbb K}}
\def\dbL{{\mathbb L}}
\def\dbM{{\mathbb M}}
\def\dbN{{\mathbb N}}
\def\dbO{{\mathbb O}}
\def\dbP{{\mathbb P}}
\def\dbQ{{\mathbb Q}}
\def\dbR{{\mathbb R}}
\def\dbS{{\mathbb S}}
\def\dbT{{\mathbb T}}
\def\dbU{{\mathbb U}}
\def\dbV{{\mathbb V}}
\def\dbW{{\mathbb W}}
\def\dbX{{\mathbb X}}
\def\dbY{{\mathbb Y}}
\def\dbZ{{\mathbb Z}}

\def\Aut{{\rm Aut}}

\def\la{{\langle}}
\def\ra{{\rangle}}
\def\rk{{\rm rk}}


\bf\centerline{Homework Assignment \# 4. Preliminary version}\rm
\vskip .2cm
{\bf Plan for the week of Feb 22:} Finish Rational Canonical Form (12.2, online lectures 10-11), Jordan Canonical Form (12.3, online lecture 12).
\vskip .1cm
Here and in all future assignments ``online'' refers to Algebra-II lectures posted on my Spring 2010 Algebra-II webpage
\vskip .1cm
\centerline{\url{http://people.virginia.edu/~mve2x/7752_Spring2010/}}
\vskip .1cm

{\bf Note on hints:} All hints are given at the end of the assignment, each on a separate page.
Problems (or parts of problems) for which hint is available are marked with *.

\vskip .3cm

\bf\centerline{Problems, due by 11:59pm on Friday, February 26th.}\rm
\vskip .1cm
\skv
{\bf Problem 1}. Let $R=\mathbb R[x]$, $F=R^3$ (the standard 3-dimensional $R$-module)
and $N$ the $R$-submodule of $F$
generated by $(1-x,1,0)$, $(-2,4-x,0)$ and $(1,-5,-x)$.
\begin{itemize}
\item[(a)] Find compatible bases for $F$ and $N$, that is, bases satisfying the conclusion
of the compatible bases theorem (AKA submodule structure theorem). {\bf Note:} an algorithm for computing such bases is given in Lecture~8.
\item[(b)] Describe the quotient module $F/N$ in IF and ED forms.
\end{itemize}
\skv
{\bf Problem 2}. Let $R$ be a PID. For an $R$-module $M$ denote by $d(M)$ the minimal number of generators of $M$
(this quantity was called the rank in HW\#3).
\begin{itemize}
\item[(a)] Prove that if $M$ is a finitely generated $R$-module and $N$ is a submodule of $R$, then $d(N)\leq d(M)$
\item[(b)] Let $a\in R$ be a nonzero non-unit. Find (with proof) the number of submodules of $R/aR$ in terms of the prime decomposition of $a$.
\end{itemize}
\skv
{\bf Problem 3}. DF, Problem 4, page 469. Warning: the definition of rank in this problem (and in DF in general is different from HW\#4.
Then give an example showing that the assertion of this problem would be false if we replace $rk(U)$ by $d(U)$ for every module $U$ in the problem.
\skv
{\bf Problem 4}. DF, Problem 6, page 488. 
\skv
{\bf Problem 5}. Given a matrix $A$ we denoted by
$\chi_A(x)$ its characteristic polynomial. Determine the number of possible RCFs of $8\times 8$ matrices $A$ over $\dbQ$ with 
$\chi_A(x)=x^8-x^4$. Explain your argument in detail.
 
\end{document}
