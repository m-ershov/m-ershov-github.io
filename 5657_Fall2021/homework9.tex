\documentclass[12pt]{amsart}

\usepackage{amsmath}
\usepackage{amssymb}
\usepackage{amsthm}
\usepackage{amscd}
\usepackage{url}
\usepackage{hyperref}
\usepackage[all]{xy}
%\usepackage{psfig}

\begin{document}
\baselineskip=16pt
\textheight=8.5in
\parindent=0pt 
\def\Ker {{\rm Ker}}
\def\Im {{\rm Im}}
\def\Tr {{\rm Tr}}
\def\sk {\hskip .3cm}
\def\skv {\vskip .08cm}
\def\cos {\mbox{cos}}
\def\sin {\mbox{sin}}
\def\tan {\mbox{tan}}
\def\intl{\int\limits}
\def\lm{\lim\limits}
\newcommand{\frc}{\displaystyle\frac}
\def\xbf{{\mathbf x}}
\def\fbf{{\mathbf f}}
\def\gbf{{\mathbf g}}

\def\dbA{{\mathbb A}}
\def\dbB{{\mathbb B}}
\def\dbC{{\mathbb C}}
\def\dbD{{\mathbb D}}
\def\dbE{{\mathbb E}}
\def\dbF{{\mathbb F}}
\def\dbG{{\mathbb G}}
\def\dbH{{\mathbb H}}
\def\dbI{{\mathbb I}}
\def\dbJ{{\mathbb J}}
\def\dbK{{\mathbb K}}
\def\dbL{{\mathbb L}}
\def\dbM{{\mathbb M}}
\def\dbN{{\mathbb N}}
\def\dbO{{\mathbb O}}
\def\dbP{{\mathbb P}}
\def\dbQ{{\mathbb Q}}
\def\dbR{{\mathbb R}}
\def\dbS{{\mathbb S}}
\def\dbT{{\mathbb T}}
\def\dbU{{\mathbb U}}
\def\dbV{{\mathbb V}}
\def\dbW{{\mathbb W}}
\def\dbX{{\mathbb X}}
\def\dbY{{\mathbb Y}}
\def\dbZ{{\mathbb Z}}

\def\lam{{\lambda}}
\def\Ker{\mathrm {Ker}}
\def\End{\mathrm {End}}
\def\Hom{\mathrm {Hom}}
\def\Irr{\mathrm {Irr}}
\def\la{{\langle}}
\def\ra{{\rangle}}
\def\summ{{\sum\limits}}
\def\char{{\rm char}}

\bf\centerline{Homework \#9. Due Tuesday, November 23rd}\rm
\vskip .1cm

\bf\centerline{Reading: }\rm
\skv
1. For this homework assignment: online class notes (Lectures 18-21) and Steinberg, parts of 4.2-4.4.
\skv

2. Plan for upcoming classes. Thu, Nov 18: permutation representations (online Lecture 22, Chapter 7 in Steinberg). Tue, Nov 23: TBA.
\skv
\skv


\bf\centerline{Problems: }\rm
\skv
For problems (or their parts) marked with a *, a hint is given later in the assignment. Do not to look at the hint(s) until you seriously tried to solve the problem without it.
\skv
{\bf 1.} Let $p$ be a prime and $G={\rm Heis}(\dbZ_p)$, the Heisenberg group over $\dbZ_p$ defined in HW\#7.2
\begin{itemize}
\item[(a)] Determine the number of conjugacy classes of $G$ and their sizes. As in HW\#8.6, you can work directly with matrices or 
with their expressions in terms of the generators $x,y,z$ introduced in HW\#7.2.
\item[(b)] Let $\omega\neq 1$ be a $p^{\rm th}$ root of unity, that is, $\omega=e^{\frac{2\pi k i}{p}}$ with $1\leq k\leq p-1$.
Let $V$ be a $p$-dimensional complex vector space with basis $e_{[0]}, e_{[1]},\ldots, e_{[p-1]}$ where we think of indices as elements of $\dbZ_p$.
Prove that there exists a representation $(\rho_{\omega},V)$ of $G$ such that
\begin{itemize}
\item $\rho_{\omega}(z) e_{[k]}=\omega e_{[k]}$ for each $k$ (that is, $\rho_{\omega}(z)$ is just the scalar multiplication by $\omega$),
\item $\rho_{\omega}(y) e_{[k]}=e_{[k+1]}$ for each $k$ (that is, $\rho_{\omega}(y)$ cyclically permutes the basis vectors) and finally 
\item $\rho_{\omega}(x) e_{[k]}=\omega^{k} e_{[k]}$ for each $k$.
\end{itemize}
\item[(c)] Prove that every representation in (b) is irreducible (do not do this directly from definition) and every irreducible complex representation of $G$ is either one-dimensional 
or equivalent to $(\rho_{\omega},V)$ for some $\omega$.
\end{itemize}
{\bf 2.} Let $(\rho,V)$ be a representation of a group $G$. Recall that the dual representation $(\rho^*,V^*)$ is defined by
$\rho^*(g)(f)=f\circ \rho(g)^{-1}$ for all $f\in V^*$. Prove parts (1) and (2) of Claim~21.1 from class:
\begin{itemize}
\item[(1)] $(\rho^*,V^*)$ is indeed a representation
\item[(2)] If $\beta$ is any basis of $V$ and $\beta^*$ is the dual basis of $V^*$, then $[\rho^*(g)]_{\beta^*}=([\rho(g)]^{-1}_{\beta})^T$
\end{itemize}
\skv
{\bf Note:} Part (2) has almost nothing to do with representation theory. Recall from Lecture~9 that given an operator $A\in \End(V)$, its adjoint
$A^*\in \End(V^*)$ is defined by $A^*(f)=f\circ A$ (recall that this notion of adjoint is related to but slightly different from adjoints in complex
inner product spaces). What you need to prove is Claim~9.2 from online notes which asserts that $[A^*]_{\beta^*}=([A]_{\beta})^T$ (make sure to explain how (2) follows from this).
\skv
{\bf 3.*} Let $G=S_n$ for some $n\geq 2$ and $\chi$ an irreducible complex character of $G$. Prove that $\chi$ is real-valued, that is,
$\chi(g)\in \dbR$ for all $g\in G$. 
\skv
{\bf 4.} Give an example of two representations $V$ and $W$ of the same group which are not equivalent, but have the same character. Recall that by Corollary~22.2 from class this cannot happen if $G$ is finite and representations are complex.
\skv
{\bf 5.} Let $(\rho,V)$ and $(\rho',V)$ be complex representations of a finite group $G$ (the vector space $V$ is the same for both representations). Suppose that $\rho'(g)$ is conjugate to $\rho(g)$ in ${\rm GL}(V)$ for every $g\in G$. Prove that the representations $(\rho,V)$ and $(\rho',V)$ are equivalent. {\bf Note:} The result is not automatic since the matrix which conjugates
$\rho'(g)$ to $\rho(g)$ may depend on $g$.

\newpage
{\bf Hint for 3:} Use our discussion in Lecture~21 (=online Lecture~18) to find a general condition on a finite group $G$ which guarantees that all of its complex characters are real-valued and then show that this condition holds for $S_n$.
\end{document}
