\documentclass[12pt]{amsart}

\usepackage{amsmath}
\usepackage{amssymb}
\usepackage{amsthm}
\usepackage{amscd}
\usepackage{url}
\usepackage{hyperref}
\usepackage[all]{xy}
%\usepackage{psfig}

\begin{document}
\baselineskip=16pt
\textheight=8.5in
\parindent=0pt 
\def\Ker {{\rm Ker}}
\def\Im {{\rm Im}}
\def\Tr {{\rm Tr}}
\def\GL {{\rm GL}}
\def\diag {{\rm diag}}
\def\sk {\hskip .3cm}
\def\skv {\vskip .08cm}
\def\cos {\mbox{cos}}
\def\sin {\mbox{sin}}
\def\tan {\mbox{tan}}
\def\intl{\int\limits}
\def\lm{\lim\limits}
\newcommand{\frc}{\displaystyle\frac}
\def\xbf{{\mathbf x}}
\def\fbf{{\mathbf f}}
\def\gbf{{\mathbf g}}

\def\dbA{{\mathbb A}}
\def\dbB{{\mathbb B}}
\def\dbC{{\mathbb C}}
\def\dbD{{\mathbb D}}
\def\dbE{{\mathbb E}}
\def\dbF{{\mathbb F}}
\def\dbG{{\mathbb G}}
\def\dbH{{\mathbb H}}
\def\dbI{{\mathbb I}}
\def\dbJ{{\mathbb J}}
\def\dbK{{\mathbb K}}
\def\dbL{{\mathbb L}}
\def\dbM{{\mathbb M}}
\def\dbN{{\mathbb N}}
\def\dbO{{\mathbb O}}
\def\dbP{{\mathbb P}}
\def\dbQ{{\mathbb Q}}
\def\dbR{{\mathbb R}}
\def\dbS{{\mathbb S}}
\def\dbT{{\mathbb T}}
\def\dbU{{\mathbb U}}
\def\dbV{{\mathbb V}}
\def\dbW{{\mathbb W}}
\def\dbX{{\mathbb X}}
\def\dbY{{\mathbb Y}}
\def\dbZ{{\mathbb Z}}

\def\lam{{\lambda}}
\def\Ker{\mathrm {Ker}}
\def\End{\mathrm {End}}
\def\Hom{\mathrm {Hom}}
\def\Irr{\mathrm {Irr}}
\def\la{{\langle}}
\def\ra{{\rangle}}
\def\summ{{\sum\limits}}
\def\char{{\rm char}}

\bf\centerline{Homework \#10. Due Saturday, December 4th}\rm
\vskip .1cm

\bf\centerline{Reading: }\rm
\skv
1. For this homework assignment: online notes (Lectures 20-23), class notes (Lecture 24-26; note that most of the material from Lectures 25 and 26 is not in the online notes) and Steinberg, 7.1 and 7.2. 
\skv

2. The main topic after Thanksgiving break will be Burnside's pq-theorem (Steinberg, Chapter 6).





\bf\centerline{Problems: }\rm
\skv
{\bf 1.} Problem 4 from Midterm\#2.

\skv
{\bf 2.} Recall that in HW\#8.7 we proved that if $(\rho,V)$ is any cyclic representation of a finite group $G$ over an arbitrary field, then 
$\dim(V)\leq |G|$. Now prove that if $(\rho,V)$ is irreducible, then
$\dim(V)\leq |G|-1$.
\skv
{\bf 3.} The goal of this problem is to explicitly decompose the regular representation $(\rho_{reg},\dbC[S_3])$ as a direct sum of irreducible representations of $S_3$. Recall that
$S_3$ has 3 I$\dbC$R's: two one-dimensional (the trivial representation and the sign representation) and one two-dimensional (the standard representation) and that each I$\dbC$R appears in $\dbC[S_3]$ with multiplicity equal to its dimension (Proposition~21.3 in online notes).
\begin{itemize}
\item[(a)] Let $H=\la (1,2,3)\ra$ and consider $(\rho_{reg},\dbC[S_3])$ as a representation of $H$. Prove that $\dbC[S_3]=V_1\oplus V_2$ where both 
$V_1$ and $V_2$ are $H$-invariant and equivalent (as $H$-representations) 
to the regular representation of $H$.
\item[(b)] Since $H\cong \dbZ_3$, online Lecture~21 shows how to explicitly decompose $\dbC[H]$, the regular representation of $H$, into a direct sum of 3 one-dimensional $H$-representations.
Combining this with (a), we get an explicit decomposition
$\dbC[S_3]=\oplus_{i=1}^6 W_i$ where each $W_i$ is a one-dimensional 
and $H$-invariant. 

Show that after a suitable renumbering of $W_1,\ldots, W_6$
the following is true: $W_1\oplus W_2$ and $W_3\oplus W_4$ are both irreducible
subrepresentations of $S_3$ (these are the copies of the standard representation we are supposed to get by Proposition~21.3), while $W_5\oplus W_6$ decomposes (in a different way) into a direct sum of the trivial and the sign representations of $S_3$. 
\end{itemize}
\skv
{\bf 4.} (Steinberg, Exercise 7.5, reformulated). Let $p$ be a prime, and let 
$G$ be the set of all functions from $\dbZ_p$ to $\dbZ_p$ which have the form
$x\mapsto ax+b$ for some $a\in\dbZ_p^{\times}$ and $b\in\dbZ_p$.
\begin{itemize}
\item[(a)] Prove that $G$ is a group (with respect to composition) isomorphic to the group of matrices $\left\{\begin{pmatrix} a&b\\ 0&1\end{pmatrix}:
a\in\dbZ_p^{\times}, b\in\dbZ_p\right\}$. Note that for $p=5$ this is the group
from HW\#8.6.
\item[(b)] The group $G$ has a natural action on $\dbZ_p$ (given by
$g.x=g(x)$ for all $g\in G$ and $x\in\dbZ_p$). Prove that this action is
$2$-transitive (see Steinberg 7.1 for the definition).
\item[(c)] By Lemma~21.2(2) from online notes, the action from (b) yields a homomorphism
$\phi:G\to S_p$. Composing $\phi$ with the standard representation of $S_p$,
we obtain a $(p-1)$-dimensional representation of $G$. Deduce from (b) (and a suitable result from class) that this representation is irreducible. 
\end{itemize}

 


\skv
{\bf 5.} Let $C$ be the cube in $\dbR^3$ whose vertices have coordinates 
$(\pm 1, \pm 1,\pm 1)$. Let $G$ be the group of rotations of $C$, that is rotations in $\dbR^3$ which preserve the cube (you may assume that $G$ is a group without proof). Let $X$ be the set of $4$ main diagonals of $C$ (diagonals
connecting the opposite vertices). Note that $G$ naturally acts on $X$ and therefore we have a homomorphism $\pi:G\to Sym(X)\cong S_4$.
\begin{itemize}
\item[(a)*] Prove that $\pi$ is an isomorphism.
\item[(b)] Note that $G$ is naturally a subgroup of $\GL_3(\dbR)$ and hence
also a subgroup of $\GL_3(\dbC)$, and let $\iota: G\to \GL_3(\dbC)$
be the inclusion map. By (a) we get a representation $\iota\circ \pi^{-1}:
S_4\to\GL_3(\dbC)$. Prove that this representation is equivalent to the tensor product of the standard and sign representations.
\end{itemize}
\newpage
{\bf Hint for 5(a)} First show that $G$ acts transitively on the 8 vertices of $C$. Then show that the stabilizer of a fixed vertex has order $\geq 3$. This implies that $|G|\geq 24=|S_4|$. Finally, show that $\pi$ is injective (since $|G|\geq |S_4|$, this would force $\pi$ to be an isomorphism).
\end{document}
