\documentclass[12pt]{amsart}

\usepackage{amsmath}
\usepackage{amssymb}
\usepackage{amsthm}
%\usepackage{psfig}

\begin{document}
\baselineskip=16pt
\textheight=9.6in
\parindent=0pt 
\def\sk {\hskip .5cm}
\def\skv {\vskip .3cm}
\def\cos {\mbox{cos}}
\def\sin {\mbox{sin}}
\def\tan {\mbox{tan}}
\def\intl{\int\limits}
\def\lm{\lim\limits}
\newcommand{\frc}{\displaystyle\frac}
\def\xbf{{\mathbf x}}
\def\fbf{{\mathbf f}}
\def\gbf{{\mathbf g}}

\def\dbA{{\mathbb A}}
\def\dbB{{\mathbb B}}
\def\dbC{{\mathbb C}}
\def\dbD{{\mathbb D}}
\def\dbE{{\mathbb E}}
\def\dbF{{\mathbb F}}
\def\dbG{{\mathbb G}}
\def\dbH{{\mathbb H}}
\def\dbI{{\mathbb I}}
\def\dbJ{{\mathbb J}}
\def\dbK{{\mathbb K}}
\def\dbL{{\mathbb L}}
\def\dbM{{\mathbb M}}
\def\dbN{{\mathbb N}}
\def\dbO{{\mathbb O}}
\def\dbP{{\mathbb P}}
\def\dbQ{{\mathbb Q}}
\def\dbR{{\mathbb R}}
\def\dbS{{\mathbb S}}
\def\dbT{{\mathbb T}}
\def\dbU{{\mathbb U}}
\def\dbV{{\mathbb V}}
\def\dbW{{\mathbb W}}
\def\dbX{{\mathbb X}}
\def\dbY{{\mathbb Y}}
\def\dbZ{{\mathbb Z}}

\def\la{{\langle}}
\def\ra{{\rangle}}

\def\Aut{{\rm Aut}}
\def\Inn{{\rm Inn}}

\bf\centerline{Homework \#4. }\rm
\vskip .1cm
{\bf Plan for next week: } Simplicity of $A_n$ (4.6), classification of finitely generated abelian groups (5.2).
\vskip .1cm
\centerline{\bf Problems, to be submitted by Thursday, September 26th}

\skv
{\bf 1.} $\empty$
\begin{itemize}
\item[(a)] Prove Observation 8.2 from class: Let $H,K$ be groups, let $\phi$ and $\psi$ be homomorphisms from $K$ to $\Aut(H)$, and assume that there exists $\theta\in \Aut(K)$ such that
$\phi\circ\theta=\psi$. Prove that $H\rtimes_{\phi} K\cong H\rtimes_{\psi} K$.
\item[(b)] DF, Problem 6 on page 184.
\end{itemize}

\skv
{\bf 2.} DF, Problem 7(a)(c)(e) on page 185. Note that we proved (b) in class (Lecture~6).
Clarification for part (c): for each isomorphism class of $S$
you are asked to construct a certain number of non-isomorphic groups
of order $56$ with normal $7$-Sylow and $2$-Sylow isomorphic to $S$.
You are not asked to prove that your groups cover all possible isomorphism classes
(this part is optional and can be done using Problem~2(a) above). If you are using the hint in brackets following part (c), you should prove the statement in the hint.
\skv

{\bf 3.} Let $n\geq 3$ be an integer and let $S_n$ be the symmetric group on $\{1,2,\ldots, n\}$. Let $H$ be a subgroup of $S_n$ with $[S_n:H]=n$. Prove that $$H\cong S_{n-1}.$$ {\bf Hint:} Start by constructing a suitable action of $S_n$ associated to $H$. 
You may use the description of normal subgroups of $S_n$ which will be proved in class next Tuesday:
\begin{itemize}
\item[(i)] If $n\neq 4$, the only normal subgroups of $S_n$ are $S_n$, $A_n$ and $\{1\}$
\item[(ii)] The only normal subgroups of $S_4$ are $S_4$, $A_4$, $V_4$ (the Klein 4-group) and $\{1\}$
\end{itemize}
\skv

{\bf 4.} Let $\Omega$ be an infinite countable set (for simplicity you may assume that $\Omega=\dbZ$,
the integers). Let $S(\Omega)$ be the group of all permutations of $\Omega$.
A permutation $\sigma\in S(\Omega)$ is called {\it finitary} if it moves
only a finite number of points, that is, the set $\{i\in\Omega : \sigma(i)\neq i\}$
is finite. It is easy to see that finitary permutations form a subgroup of $S(\Omega)$
which will be denoted by $S_{fin}(\Omega)$. Finally, let $A_{fin}(\Omega)$ be the subgroup
of even permutations in $S_{fin}(\Omega)$ (note that it makes sense to talk about
even permutations in $S_{fin}(\Omega)$, but not in $S(\Omega)$).
\begin{itemize}
\item[(a)] Prove that the group $A_{fin}(\Omega)$ is simple and that $A_{fin}(\Omega)$
is a subgroup of index two in $S_{fin}(\Omega)$. {\bf Hint:} To prove the first
assertion solve problem 5 in [DF, page 151]. You may use the fact that $A_n$ is simple for $n\geq 5$.
\item[(b)] Prove that $A_{fin}(\Omega)$ and $S_{fin}(\Omega)$ are both normal in $S(\Omega)$.
\item[(c)] Prove that neither of the groups $S(\Omega)$ and $S_{fin}(\Omega)$ is finitely generated. {\bf Hint:} The two groups are not finitely generated for completely
different reasons.
\item[(d)] Construct a finitely generated subgroup $G$ of $S(\Omega)$ which contains
$S_{fin}(\Omega)$. {\bf Note:} This example shows that a subgroup of a finitely
generated group does not have to be finitely generated.
\end{itemize}



\end{document}