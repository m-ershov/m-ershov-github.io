\documentclass[12pt]{amsart}

\usepackage{amsmath}
\usepackage{amssymb}
\usepackage{amsthm}
%\usepackage{psfig}

\begin{document}
\baselineskip=16pt
\textheight=9.2in
\parindent=0pt 
\def\sk {\hskip .5cm}
\def\skv {\vskip .12cm}
\def\cos {\mbox{cos}}
\def\sin {\mbox{sin}}
\def\tan {\mbox{tan}}
\def\intl{\int\limits}
\def\lm{\lim\limits}
\newcommand{\frc}{\displaystyle\frac}
\def\xbf{{\mathbf x}}
\def\fbf{{\mathbf f}}
\def\gbf{{\mathbf g}}

\def\dbA{{\mathbb A}}
\def\dbB{{\mathbb B}}
\def\dbC{{\mathbb C}}
\def\dbD{{\mathbb D}}
\def\dbE{{\mathbb E}}
\def\dbF{{\mathbb F}}
\def\dbG{{\mathbb G}}
\def\dbH{{\mathbb H}}
\def\dbI{{\mathbb I}}
\def\dbJ{{\mathbb J}}
\def\dbK{{\mathbb K}}
\def\dbL{{\mathbb L}}
\def\dbM{{\mathbb M}}
\def\dbN{{\mathbb N}}
\def\dbO{{\mathbb O}}
\def\dbP{{\mathbb P}}
\def\dbQ{{\mathbb Q}}
\def\dbR{{\mathbb R}}
\def\dbS{{\mathbb S}}
\def\dbT{{\mathbb T}}
\def\dbU{{\mathbb U}}
\def\dbV{{\mathbb V}}
\def\dbW{{\mathbb W}}
\def\dbX{{\mathbb X}}
\def\dbY{{\mathbb Y}}
\def\dbZ{{\mathbb Z}}

\def\la{{\langle}}
\def\ra{{\rangle}}

\def\Aut{{\rm Aut}}
\def\Inn{{\rm Inn}}
\def\Sym{{\rm Sym}}
\def\det{{\rm det}}

\bf\centerline{Homework \#2}\rm
\vskip .1cm
\centerline{{\bf Plan for next week:} Sylow theorems (\S~4.5).}
\vskip .1cm
\centerline{\bf Problems, to be submitted by Thursday, September 12th}
\vskip .1cm
{\bf Note:} Problems 1 and 2 below are relevant to the proof of Sylow theorems that we will discuss in class next week.
\skv
\sk Let $R$ be a commutative ring with $1$ and $n\in\dbN$. Recall that $GL_n(R)$ denotes the group of invertible $n\times n$
matrices over $R$ and $SL_n(R)$ the subgroup of $GL_n(R)$ consisting of matrices of determinant $1$. In general, a matrix
$A\in Mat_n(R)$ lies in $GL_n(R)$ if and only if $\det(A)\in R^{\times}$. In particular, if $R$ is a field, then $A\in GL_n(R)$
if and only if $\det(A)\neq 0$.

\skv
{\bf 1.} Let $F$ be a finite field of order $q$.
\begin{itemize}
\item[(a)] Prove that $GL_2(F)$ has order $(q^2-1)(q^2-q)=q(q-1)^2(q+1)$

\item[(b)] State and prove the formula for the order of the groups $GL_n(F)$ for $n>2$.

\item[(c)] Prove that $|SL_n(F)|=\frac{|GL_n(F)|}{q-1}$ for any $n\geq 2$.
\end{itemize}
{\bf Hint: } To determine the order of $GL_n(F)$ use the fact that a square matrix over a field $F$
is invertible if and only if its rows are linearly independent.
\skv
{\bf 2.} Problem 10 on page 117 in Dummit and Foote.
\skv

{\bf 3.} An action of a group $G$ on a set $X$ is called {\it transitive} if it has
just one orbit, that is, for any $x,y\in X$ there exists $g\in G$ with $g. x=y$.
\begin{itemize}


\item[(a)] Let $(G,X,.)$ be a group action. Prove that if $x,y\in X$ lie in the same orbit,
then their stabilizers $Stab_G(x)$ and $Stab_G(y)$ are conjugate, that is, there exists $g\in G$
with $g Stab_G(x) g^{-1} = Stab_G(y)$. 

\item[(b)] Suppose that $(G,X,.)$ is a transitive action and fix $x\in X$. Prove that
the kernel of this action is equal to $\bigcap\limits_{g\in G} g Stab_G(x) g^{-1}$

\item[(c)] Now suppose that $G$ and $X$ are both finite, $(G,X,.)$ is a transitive 
faithful action (where `faithful' means the kernel is trivial) and $G$ is abelian. 
Prove that for any $g\in G\setminus \{1\}$ the fixed set $Fix_X(g)$ is empty. 
Deduce that $|X|=|G|$. {\bf Hint:} Use (b).
\end{itemize}

{\bf 4.} Let $C$ be the cube in $\dbR^3$ whose vertices have coordinates 
$(\pm 1, \pm 1,\pm 1)$. Let $G$ be the group of rotations of $C$, that is rotations in $\dbR^3$ which preserve the cube (you may assume that $G$ is a group without proof). Let $X$ be the set of $4$ main diagonals of $C$ (diagonals
connecting the opposite vertices). Note that $G$ naturally acts on $X$ and therefore we have a homomorphism $\pi:G\to Sym(X)\cong S_4$. Prove that $\pi$ is an isomorphism.

{\bf Hint: } First show that $G$ acts transitively on the 8 vertices of $C$. Then show that the stabilizer of a fixed vertex had order $\geq 3$. This implies that $|G|\geq 24=|S_4|$. Finally, show that $\pi$ is injective (since $|G|\geq |S_4|$, this would force $\pi$ to be an isomorphism).

\skv
{\bf 5.} In Lecture~3 we sketched a proof of the fact that $\Aut(D_8)\cong D_8$. Justify the following parts of the argument from class:
\begin{itemize}
\item[(i)] $|\Aut(D_8)|\geq 8$. {\bf Hint:} first prove that $|\Inn(D_8)|=4$ and think how this helps to prove the desired inequality. Also think how one can construct automorphisms of a group defined by generators and relations (since we have not formally defined presentations by generators and relations yet, I am not expecting a completely rigorous argument here).
\item[(ii)] Let $S$ be the set of 4 reflections in $D_8$. In class we argued that $S$ invariant under any automorphism of $D_8$. Hence we have a natural action of $\Aut(D_8)$ on $S$ and the corresponding permutation representation $T:\Aut(D_8)\to \Sym(S)$. Prove that $T$ is injective.
{\bf Hint:} let $G$ be any group, and let $S$ be an $\Aut(G)$-invariant subset of $G$. Find a simple-to-state sufficient condition on $S$
which guarantees that the the corresponding permutation representation $T:\Aut(G)\to \Sym(S)$ is injective
\end{itemize}

{\bf 6.} Let $n\geq 4$ and $f=(1,2)(3,4)\in S_n$. Prove that 
$|C_{S_n}(f)|=8 (n-4)!$. Then describe elements of this 
centralizer explicitly. {\bf Hint:} What is the conjugacy
class of $f$?
\skv

{\bf 7.} (optional) {\it Necklace-counting problem:}
Suppose that we want to build a necklace using $n$ beads of $k$ possible colors
(we do not have to use all available colors). Two necklaces will be considered equivalent 
if they can be obtained from each other using rotations or reflections. What is the number 
of non-equivalent necklaces one can construct?

{\it Approach using group actions.} Let $X$ be the set of all necklaces with
beads of $k$ possible colors located at the vertices of a regular $n$-gon. The dihedral
group $D_{2n}$ has a natural action on $X$, and the orbits under that action
are precisely equivalence classes of necklaces in the above sense. Use this
interpretation and Burnside's orbit-counting formula to prove that for $n=9$ the number of 
non-equivalent necklaces is
$$\frac{k^9+2k^3+6k+9k^5}{18}.$$
Then try to do the same for general $n$. 


\end{document}
