\documentclass[12pt]{amsart}

\usepackage{amsmath}
\usepackage{amssymb}
\usepackage{amsthm}
\usepackage{amscd}
\usepackage[all]{xy}
\usepackage{url}
\usepackage{hyperref}
%\usepackage{psfig}

\begin{document}
\baselineskip=16pt
\textheight=8.5in
\textwidth=6in
\parindent=0pt
\def\sk {\hskip .5cm}
\def\skv {\vskip .12cm}
\def\cos {\mbox{cos}}
\def\sin {\mbox{sin}}
\def\tan {\mbox{tan}}
\def\intl{\int\limits}
\def\lm{\lim\limits}
\newcommand{\frc}{\displaystyle\frac}
\def\xbf{{\mathbf x}}
\def\fbf{{\mathbf f}}
\def\gbf{{\mathbf g}}

\def\dbA{{\mathbb A}}
\def\dbB{{\mathbb B}}
\def\dbC{{\mathbb C}}
\def\dbD{{\mathbb D}}
\def\dbE{{\mathbb E}}
\def\dbF{{\mathbb F}}
\def\dbG{{\mathbb G}}
\def\dbH{{\mathbb H}}
\def\dbI{{\mathbb I}}
\def\dbJ{{\mathbb J}}
\def\dbK{{\mathbb K}}
\def\dbL{{\mathbb L}}
\def\dbM{{\mathbb M}}
\def\dbN{{\mathbb N}}
\def\dbO{{\mathbb O}}
\def\dbP{{\mathbb P}}
\def\dbQ{{\mathbb Q}}
\def\dbR{{\mathbb R}}
\def\dbS{{\mathbb S}}
\def\dbT{{\mathbb T}}
\def\dbU{{\mathbb U}}
\def\dbV{{\mathbb V}}
\def\dbW{{\mathbb W}}
\def\dbX{{\mathbb X}}
\def\dbY{{\mathbb Y}}
\def\dbZ{{\mathbb Z}}

\def\la{{\langle}}
\def\ra{{\rangle}}
\def\phi{{\varphi}}

\def\Ker{{\rm Ker\,}}
\def\Aut{{\rm Aut}}
\def\Inn{{\rm Inn}}

\bf\centerline{Homework \#10. }\rm
\vskip .1cm
{\bf Plan for next week:} Affine algebraic sets and Nullstellensatz (DF, Chapter 15). For the material of the lecture on Nov 19 see
\skv
\centerline{\url{http://people.virginia.edu/~mve2x/7751_Fall2009/lecture29.pdf}}
\skv
\skv
\centerline{\bf Problems, to be submitted by Friday, Nov 22, by 1pm}
\skv 
Before solving problems 1 and 2 read the online lecture on irreducibility criteria
\skv
\centerline{\url{http://people.virginia.edu/~mve2x/7751_Fall2011/lecture23.pdf}}
\skv
\skv
{\bf 1.} Let $F$ be a field, take $f(x,y)\in F[x,y]$, and write $f(x,y)=\sum_{i=0}^n c_i(y) x^i$
where $c_i(y)\in F[y]$. Suppose that 
\begin{itemize}
\item[(i)] There exists $\alpha\in F$ such that $c_n(\alpha)\neq 0$
\item[(ii)] $gcd(c_0(y), c_1(y),\ldots, c_n(y))=1$ in $F[y]$
\item[(iii)] $f(x,\alpha)$ is an irreducible element of $F[x]$ (where $f(x,\alpha)$ is the polynomial obtained from $f(x,y)$ be substituting $\alpha$ for $y$). 
\end{itemize}
Prove that $f(x,y)$ is irreducible in $F[x,y]$. 
\skv
{\bf 2.} Prove that the following polynomials are irreducible:
\begin{itemize}
\item[(a)] $f(x,y)=y^3+x^2 y^2+x^3y+x^2+x$ in $\dbQ[x,y]$
\item[(b)] $f(x,y)=xy^2+x^2y+2xy+x+y+1$ in $\dbQ[x,y]$
\item[(c)] $f(x)=x^5-3x^2+15x-7$ in $\dbQ[x]$
\end{itemize}
{\bf Hint:} For (a) and (b) -- think of $\dbQ[x,y]$ as $(\dbQ[x])[y]$, the ring of polynomials in one variable $y$ over $R=\dbQ[x]$, or as $(\dbQ[y])[x]$.
For (c): By Gauss Lemma, it is enough to prove irreducibility
of $f(x)$ in $\dbZ[x]$. Consider the reduction map $u(x)\to \overline u(x)$
from $\dbZ[x]$ to $\dbZ_3[x]$, consider possible factorizations of
$\overline f(x)$ and show that none of them can be lifted to a factorization
of $f(x)$ (the general idea is similar to the proof of the Eisenstein criterion).
\skv

{\bf 3.}\begin{itemize} 
\item[(a)] Let $p$ be a prime. Use direct counting argument to find the number
of monic irreducible polynomials of degree $n$ in $\dbF_p[x]$ for $n=2,3,4$ and
check that your answer matches the general formula derived in the online supplement
\skv
\centerline{\url{http://people.virginia.edu/~mve2x/7751_Fall2011/irreducible.pdf}}
\skv
{\bf Hint:} The number of irreducible monic polynomials of degree $n$ equals
the total number of monic polynomials of degree $n$ minus the number
of reducible monic polynomials of degree $n$; the latter can be computed
considering possible factorizations into irreducibles (assuming the number of 
irreducible monic polynomials of degree $m$ for $m<n$ has already been computed).
\item[(b)] Find a monic irreducible polynomial of degree $4$ in $\dbF_2[x]$.
Then use it to construct a field $F$ with $|F|=16$ and {\bf find explicitly}
a generator of the multiplicative group $F^{\times}$
\end{itemize}
\skv
{\bf 4.} Let $R$ be a commutative Noetherian ring with 1. Prove that
the ring $R[[x]]$ of power series over $R$ is also Noetherian.
{\bf Hint:} As you may expect, this can be proved similarly 
to the Hilbert basis theorem (HBT) except that you have to consider
the lowest degree terms, not the highest degree terms (which may not exist). 
In fact, the first part of the proof is even easier than in HBT, but you will 
need some kind of limit argument at the end.


\skv
{\bf 5.} Give an example of a domain $R$ (other than a field or
the zero ring) which has no irreducible elements. {\bf Hint:}
Start with the ring of power series $R=F[[x]]$ where $F$ is a field.
Then up to associates $x$ is the only irreducible element of $R$.
Construct a larger ring $R_1\supseteq R$ s.t. $x$ is reducible in $R_1$, but
$R_1\cong F[[x]]$. Then iterating the process construct
an infinite ascending chain $R\subseteq R_1\subseteq R_2\subseteq \ldots$ and consider its union.
\end{document}
