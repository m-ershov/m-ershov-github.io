\documentclass[12pt]{amsart}

\usepackage{amsmath}
\usepackage{amssymb}
\usepackage{amsthm}
%\usepackage{psfig}

\begin{document}
\baselineskip=16pt
\textheight=9in
\textwidth=6in
\parindent=0pt
\def\sk {\hskip .5cm}
\def\skv {\vskip .12cm}
\def\cos {\mbox{cos}}
\def\sin {\mbox{sin}}
\def\tan {\mbox{tan}}
\def\intl{\int\limits}
\def\lm{\lim\limits}
\newcommand{\frc}{\displaystyle\frac}
\def\xbf{{\mathbf x}}
\def\fbf{{\mathbf f}}
\def\gbf{{\mathbf g}}

\def\dbA{{\mathbb A}}
\def\dbB{{\mathbb B}}
\def\dbC{{\mathbb C}}
\def\dbD{{\mathbb D}}
\def\dbE{{\mathbb E}}
\def\dbF{{\mathbb F}}
\def\dbG{{\mathbb G}}
\def\dbH{{\mathbb H}}
\def\dbI{{\mathbb I}}
\def\dbJ{{\mathbb J}}
\def\dbK{{\mathbb K}}
\def\dbL{{\mathbb L}}
\def\dbM{{\mathbb M}}
\def\dbN{{\mathbb N}}
\def\dbO{{\mathbb O}}
\def\dbP{{\mathbb P}}
\def\dbQ{{\mathbb Q}}
\def\dbR{{\mathbb R}}
\def\dbS{{\mathbb S}}
\def\dbT{{\mathbb T}}
\def\dbU{{\mathbb U}}
\def\dbV{{\mathbb V}}
\def\dbW{{\mathbb W}}
\def\dbX{{\mathbb X}}
\def\dbY{{\mathbb Y}}
\def\dbZ{{\mathbb Z}}

\def\la{{\langle}}
\def\ra{{\rangle}}

\def\Aut{{\rm Aut}}
\def\Inn{{\rm Inn}}
\def\wr{\,{\rm wr}\,}

\bf\centerline{Homework \#6. }\rm
\vskip .1cm
{\bf Plan for next week:} Free groups and presentations of groups by generators and relators (\S~6.3).
\vskip .1cm
\centerline{\bf Problems, to be submitted by Thursday, October 17th}
\skv
{\bf 1.} Let $G_1,\ldots, G_k$ be non-abelian simple groups and let $G=G_1\times\ldots\times G_k$. Prove that every normal subgroup of $G$ is equal to $H_1\times\ldots\times H_k$ where for each $i$ either $H_i=G_i$ or $H_i=\{1\}$. You do not have to turn in this problem if you got full credit for 2(b) on the take-home part of Midterm~1.


{\bf 2.} Let $G$ be a group.
\begin{itemize}
\item[(a)] Prove that if $N$ is a normal subgroup of $G$, then for any $k\in\dbN$ we have
$\gamma_k (G/N)=(\gamma_k G\cdot N)/N$.
\item[(b)] Now assume that $G$ is nilpotent of class $c\geq 1$. Prove that $G/Z(G)$ is nilpotent of class exactly $c-1$ (we proved inequality in one direction in class).
\end{itemize}
\skv
{\bf 3.} (a) Let $R$ be an associative ring with $1$, and let $a,b\in R$ be such that
$1+a$ and $1+b$ are invertible. Prove the following formula 
$$(1+a)^{-1}(1+b)^{-1}(1+a)(1+b)=1+(1+a)^{-1}(1+b)^{-1}(ab-ba).$$

(b) Let $R$ be an associative ring with $1$ and $n\geq 2$ be an integer, and let
$U_n(R)$ be the upper unitriangular subgroup of $GL_n(R)$. Prove that $U_n(R)$
is nilpotent of class $n-1$ (we briefly outlined the proof in class). Note that you
will need to apply (a) not to $R$ itself but to the ring of $n\times n$ matrices over $R$.
\skv
{\bf 4.} Problems 24 and 25 on page 199 of DF.
\skv
{\bf 5.} Problems 31 and 32 on page 200 of DF. {\bf Note:} Problem 31 follows
very easily from the lemma about the structure of minimal normal subgroups proved in class.
\skv
{\bf 6. }(a) Prove that a Sylow $p$-subgroup of $S_{p^2}$ is isomorphic to $\dbZ_p \,wr\, \dbZ_p$.
\skv
(b) Prove that if $A$ and $B$ are solvable groups, then $A\, wr\, B$ is also solvable.
\skv
(c) (bonus) Find all integers $m,n\geq 2$ for which $\dbZ_m \,wr\, \dbZ_n$ is nilpotent.  
\end{document}
