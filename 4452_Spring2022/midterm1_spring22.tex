\documentclass[12pt]{amsart}

\usepackage{amsmath}
\usepackage{amssymb}
\usepackage{amsthm}
%\usepackage{psfig}

\begin{document}
\baselineskip=16pt
%\textheight=9.6in
\parindent=0pt
\def\sk {\hskip .5cm}
\def\skv {\vskip .12cm}
\def\cos {\mbox{cos}}
\def\sin {\mbox{sin}}
\def\tan {\mbox{tan}}
\def\intl{\int\limits}
\def\lm{\lim\limits}
\newcommand{\frc}{\displaystyle\frac}
\def\xbf{{\mathbf x}}
\def\fbf{{\mathbf f}}
\def\gbf{{\mathbf g}}

\def\dbA{{\mathbb A}}
\def\dbB{{\mathbb B}}
\def\dbC{{\mathbb C}}
\def\dbD{{\mathbb D}}
\def\dbE{{\mathbb E}}
\def\dbF{{\mathbb F}}
\def\dbG{{\mathbb G}}
\def\dbH{{\mathbb H}}
\def\dbI{{\mathbb I}}
\def\dbJ{{\mathbb J}}
\def\dbK{{\mathbb K}}
\def\dbL{{\mathbb L}}
\def\dbM{{\mathbb M}}
\def\dbN{{\mathbb N}}
\def\dbO{{\mathbb O}}
\def\dbP{{\mathbb P}}
\def\dbQ{{\mathbb Q}}
\def\dbR{{\mathbb R}}
\def\dbS{{\mathbb S}}
\def\dbT{{\mathbb T}}
\def\dbU{{\mathbb U}}
\def\dbV{{\mathbb V}}
\def\dbW{{\mathbb W}}
\def\dbX{{\mathbb X}}
\def\dbY{{\mathbb Y}}
\def\dbZ{{\mathbb Z}}

\def\la{{\langle}}
\def\ra{{\rangle}}

\def\Aut{{\rm Aut}}
\def\End{{\rm End}}
\def\Inn{{\rm Inn}}
\def\Ker{{\rm Ker}}
\def\Im{{\rm Im\,}}
\def\phi{{\varphi}}

\bf\centerline{Math 4452, Spring 2020. Midterm \#2}\rm
\skv
\bf\centerline{due Friday, March 4th, by 5pm in filedrop}\rm
\vskip .3cm
{\bf Directions: } Provide complete arguments
(do not skip steps). State clearly any result you are referring to. Partial credit for
incorrect solutions, containing steps in the right direction, may be given.
\vskip .1cm

{\bf Rules: } You are not allowed to discuss midterm problems with each other.
You may ask me any questions about the problems (e.g. if the formulation is unclear),
but as a rule I will only provide minor hints. You may freely use class notes (your own notes as well as notes posted on collab),
previous homework assignments, our main textbook ``Coding theory: a first course'' and lectures notes by J. Hall and Y. Lindell. The use of other books or other online resources is prohibited.

\skv
{\bf Scoring:} To be announced by Sunday, Feb 27th.

\skv
{\bf 1. }\rm Let $C$ be a linear code over some field $F$. 
\begin{itemize}
\item[(a)] Suppose that $d(C)=2k$ for some $k\in\dbN$ and there exists a coset $D$ of $C$ such that the {\bf maximum} weight of an element of $D$ is equal to $k$. Prove that all elements of $D$ have weight $k$.
\item[(b)] Give an example of a code $C$ satisfying the hypotheses of (a). You can pick your field, but you are not
allowed to specify $k$ (so you should give a family of examples, one for each $k$).
\end{itemize}
\skv
{\bf 2. }\rm  Let $C$ be the linear code over some finite field $F$ spanned by the following 3 vectors: $10112, 11212, 21021$. Find
\begin{itemize}
\item[(a)] a generator matrix for $C$
\item[(b)] a parity-check matrix for $C$
\item[(c)] $d(C)$, the distance of $C$
\end{itemize}
Include all the computations and justify all the statements (especially your answer for the distance)

{\bf Note:} The answer will depend on the characteristic of $F$. We are not excluding characteristic $2$ (by definition,
$2=1+1$ which makes sense in an arbitrary ring with $1$).
\skv
{\bf 3. }\rm Problem 4.23 from the book.
\skv
{\bf 4. }\rm Problem 4.26 from the book.
\skv
{\bf 5.} 
\begin{itemize}
\item[(a)] Write down the parity-check matrix in standard form for the Hamming code $Ham(5,2)$. {\bf Note:} Recall that Hamming codes are only defined up to equivalence, so the first 26 columns
of the matrix can be ordered arbitrarily.
\item[(b)] Assuming $Ham(5,2)$ is used for encoding, decode $$w=1^{23}0^8$$ using NND decoding. Make sure to prove your answer! (note that your answer will depend on the order of columns you chose in (a)).
\end{itemize}
{\bf Note for (b):} You do not have to use NND decoding as initially defined -- instead you can apply any of the algorithms we discussed that yield the same result. 
\skv
{\bf 6. } In Homework~1 we proved that if $C$ is a binary code of length $n$ and distance $2$, then $|C|\leq 2^{n-1}$; thus, if in addition $C$ is linear,
then $\dim C\leq n-1$. The main goal of this problem (part (b) below) is to show that if $C$ is a binary $[n,n-1,2]$-linear code, then $C$ is the parity-check code. 
\begin{itemize}
\item[(a)] Let $C$ be any binary $[n,n-1]$-linear code. Prove that $d(C)\leq 2$. {\bf Note:} This can be proved in many different ways and the statement actually holds for codes over arbitrary fields.
\item[(b)] Prove that if $C$ is a binary $[n,n-1,2]$-linear code, then $C$ is the parity-check code. 
{\bf Hint:} Use induction on $n$ and Problem~4.27 (for $q=2$). If you need a more detailed hint, see next page.
\item[(c)] Does there exist a non-linear binary $(n,2^{n-1},2)$-code? Give an example or show that such a code does not exist.
\end{itemize}
\newpage
{\bf Hint for 6(b):} For the induction step take an arbitrary binary  $[n,n-1,2]$-linear code $C$, consider the set $C'=\{w\in\dbF_2^{n-1}: w0\in C\}$
(here $w0$ is the concatenation of $w$ and $0$) and show that $C'$ is an $[n-1,n-2,2]$-linear code.
\end{document}
