\documentclass[12pt]{amsart}

\usepackage{amsmath}
\usepackage{amssymb}
\usepackage{amsthm}
%\usepackage{psfig}

\begin{document}
\baselineskip=16pt
\textheight=8.5in
%\parindent=0pt 
\def\sk {\hskip .5cm}
\def\skv {\vskip .08cm}
\def\cos {\mbox{cos}}
\def\sin {\mbox{sin}}
\def\tan {\mbox{tan}}
\def\intl{\int\limits}
\def\lm{\lim\limits}
\newcommand{\frc}{\displaystyle\frac}
\def\xbf{{\mathbf x}}
\def\fbf{{\mathbf f}}
\def\gbf{{\mathbf g}}

\def\dbA{{\mathbb A}}
\def\dbB{{\mathbb B}}
\def\dbC{{\mathbb C}}
\def\dbD{{\mathbb D}}
\def\dbE{{\mathbb E}}
\def\dbF{{\mathbb F}}
\def\dbG{{\mathbb G}}
\def\dbH{{\mathbb H}}
\def\dbI{{\mathbb I}}
\def\dbJ{{\mathbb J}}
\def\dbK{{\mathbb K}}
\def\dbL{{\mathbb L}}
\def\dbM{{\mathbb M}}
\def\dbN{{\mathbb N}}
\def\dbO{{\mathbb O}}
\def\dbP{{\mathbb P}}
\def\dbQ{{\mathbb Q}}
\def\dbR{{\mathbb R}}
\def\dbS{{\mathbb S}}
\def\dbT{{\mathbb T}}
\def\dbU{{\mathbb U}}
\def\dbV{{\mathbb V}}
\def\dbW{{\mathbb W}}
\def\dbX{{\mathbb X}}
\def\dbY{{\mathbb Y}}
\def\dbZ{{\mathbb Z}}

\def\la{{\langle}}
\def\ra{{\rangle}}
\def\summ{{\sum\limits}}

\bf\centerline{Homework \#3. Due Saturday, February 12th, by 11:59pm in filedrop}\rm
\vskip .1cm
All reading assignments and references to exercises, definitions etc. are from our main book `Coding Theory: A First Course' by Ling and Xing 
\vskip .1cm


\bf\centerline{Reading and plan for the next week: }\rm
\skv
1. For this homework assignment read 4.2-4.6
\skv
\skv
2. Next week we will continue with the basic theory of linear codes (4.2-4.6). Then (probably next Wednesday) we will introduce binary Hamming codes (5.3.1, page 84). If there will be time left, we will go back to Chapter 4 and start discussing encoding and decoding for linear codes (4.7-4.8).
\skv

\skv
\bf\centerline{Problems: }\rm
\skv
%For problems (or their parts) marked with a *, a hint is given later in the assignment. Do not to look at the hint(s) %until you seriously tried to solve the problem without it.
%\skv
{\bf 1.} Let $F$ be a finite field, and let $Q$ be the set of all nonzero squares in $F$, that is, all nonzero elements of $F$ representable as $a^2$ for some $a\in F$.
\begin{itemize}
\item[(a)] Assume that $F$ has odd characteristic. Prove that $|Q|=\frac{|F|-1}{2}$.
{\bf Hint:} Prove that for every nonzero $b\in F$ the equation $x^2=b$ has either no solutions (for $x$) in $F$ or exactly two solutions. 
\item[(b)] Now assume that $F$ has characteristic $2$. Prove that $|Q|=|F|-1$, that is, every nonzero element of $F$ is a square. {\bf Hint:}
Since $F$ is finite, it suffices to prove that the map $x\mapsto x^2$ from $F\setminus\{0\}$ to $F\setminus\{0\}$ is injective (one-to-one). The latter is not hard to prove directly (using the assumption ${\rm char\,} F=2$). 
\end{itemize}
For both parts you will likely need to use the fact that fields have no zero divisors (Lemma~3.1.3(ii) from the book).
\skv

{\bf 2.} Problem 4.2, page 66. Ignore the question about the number of bases, but make sure to prove your answer whether the given set is a subspace or not. If the set in question is a subspace, compute its dimension (also with proof).
\skv
{\bf 3.} Problem 4.3. {\bf Hint:} First count the number of ordered $k$-tuples $(v_1,\ldots, v_k)$ such that the vectors $v_1,\ldots, v_k$ are linearly independent. This can be done by using the argument from the proof of Theorem~4.1.15(ii).
\skv

{\bf 4.} Problem 4.15.

{\bf 5.} Problem 4.20 (see 4.3 for the definition of weight)

{\bf 6.} Problem 4.22.

{\bf 7.} Problem 4.31. {\bf Note:} The book describes two algorithms for finding a generator matrix for a code (Algorithms~4.1 and 4.2 in Section 4.4) and one algorithm for finding a parity-check matrix (Algorithm~4.3 in Section 4.4),
but does not provide justifications. For each algorithm we will either discuss in class next week why it works, or I will post an addendum to this assignment with an explanation. If you are using one of these algorithms in your solution to Problem~7, please clearly state which one you are using.
\newpage
{\bf Hint for 2.} If a code $C$ in this problem is linear, you just need to find a linearly independent subset $S$
such that $Span(S)=C$.
\end{document}



