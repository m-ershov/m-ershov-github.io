\documentclass[12pt]{amsart}

\usepackage{amsmath}
\usepackage{amssymb}
\usepackage{amsthm}
%\usepackage{psfig}

\begin{document}
\baselineskip=16pt
\textheight=8.5in
%\parindent=0pt 
\def\sk {\hskip .5cm}
\def\skv {\vskip .08cm}
\def\cos {\mbox{cos}}
\def\sin {\mbox{sin}}
\def\tan {\mbox{tan}}
\def\intl{\int\limits}
\def\lm{\lim\limits}
\newcommand{\frc}{\displaystyle\frac}
\def\xbf{{\mathbf x}}
\def\fbf{{\mathbf f}}
\def\gbf{{\mathbf g}}

\def\dbA{{\mathbb A}}
\def\dbB{{\mathbb B}}
\def\dbC{{\mathbb C}}
\def\dbD{{\mathbb D}}
\def\dbE{{\mathbb E}}
\def\dbF{{\mathbb F}}
\def\dbG{{\mathbb G}}
\def\dbH{{\mathbb H}}
\def\dbI{{\mathbb I}}
\def\dbJ{{\mathbb J}}
\def\dbK{{\mathbb K}}
\def\dbL{{\mathbb L}}
\def\dbM{{\mathbb M}}
\def\dbN{{\mathbb N}}
\def\dbO{{\mathbb O}}
\def\dbP{{\mathbb P}}
\def\dbQ{{\mathbb Q}}
\def\dbR{{\mathbb R}}
\def\dbS{{\mathbb S}}
\def\dbT{{\mathbb T}}
\def\dbU{{\mathbb U}}
\def\dbV{{\mathbb V}}
\def\dbW{{\mathbb W}}
\def\dbX{{\mathbb X}}
\def\dbY{{\mathbb Y}}
\def\dbZ{{\mathbb Z}}

\def\la{{\langle}}
\def\ra{{\rangle}}
\def\summ{{\sum\limits}}

\bf\centerline{Homework \#2. Due Saturday, February 5th, by 11:59pm in filedrop}\rm
\vskip .1cm
All reading assignments and references to exercises, definitions etc. are from our main book `Coding Theory: A First Course' by Ling and Xing 
\vskip .1cm

\bf\centerline{Reading: }\rm
\skv
1. For this homework assignment: Chapter 2, 3.1 and 3.2.
\skv
\skv
2. For the classes next week. Tue, Feb 1: 3.1 and 4.1. Thu, Feb 3: 4.2-4.5. We will definitely not cover all the material from those sections, but I might touch on at least one topic from each section.
\skv

\skv
\bf\centerline{Problems: }\rm
\skv
{\bf 1.} Let $C$ be a code of Hamming distance $d$. Suppose that a codeword $c\in C$ was transmitted, let $w$ be the received word and $k$ the number of transmission errors, that is, $k=d(c,w)$. By Theorem~3.1(b) from class (= Theorem~2.5.10 from the book), if $d>2k$, then NND (nearest neighbor decoding) rule works correctly, that is, $c_w=c$ where $c_w$ is the decoded word (the result of applying NND to $w$).

Now assume that $d=2k$. By Theorem~3.1(b) NND rule may not work correctly in this case
(we did not prove this part of Theorem~3.1(b) in class, but it is proved in the book).
Prove that NND still correctly determines the number of transmission errors, that is, $$d(c_w,w)=d(c,w)=k.$$

\skv
{\bf 2.} Let $k,r\in\dbN$, and let $C=Rep(k,r)$ be the repetition code with parameters $k$ (length of the original message) and $r$ (number of repetitions) over the binary alphabet $A=\{0,1\}$, that is, 
$$C=\{v^r=\underbrace{v\ldots v}_{r\mbox{ times }}: v\in \{0,1\}^k\}.$$  
The goal of this problem is to estimate the probability that NND works correctly for $C$ if we transmit over $BSC(p)$, the binary symmetric channel with crossover probability $p$, with $p<\frac{1}{2}$ (we started discussing this problem at the end
of Lecture~3 on Thu, Jan 27). To simplify some of the formulas below we will assume
that {\it $r$ is even}.

Note that $C$ has length $n=kr$. In parts (a) and (b) below we assume that $k=1$, so $n=r$.
\begin{itemize}
\item[(a)] Assume that $k=1$. For each $0\leq i\leq r$ let $p(i)$ be the probability that exactly $i$ errors occur during the transmission. Prove that 
$p(i)={r\choose i} p^i (1-p)^{r-i}$.
%and deduce that $p(i)$ is strictly decreasing as a function of $i$.
\item[(b)] We still assume that $k=1$. As discussed in Lecture~3, NND works correctly
if and only if the number of transmission errors is $<\frac{r}{2}$. Thus, if
$f(p,r)$ is the probability that NND works correctly, then
$$f(p,r)=1-\sum\limits_{i=r/2}^r p(i)=1-\sum\limits_{i=r/2}^r {r\choose i} p^i (1-p)^{r-i}$$
(recall that $r$ is assumed to be even).
Prove that $$f(p,r)>1-(4p(1-p))^{r/2}$$ and deduce that if $p$ is fixed,
then $f(p,r)\to 1$ as $r\to\infty$. 
\item[(c)] Now assume that $k$ is arbitrary, and let $f(p,r,k)$ be the probability
that NND works correctly. Prove that $f(p,r,k)=f(p,r)^k$ (where $f(p,r)$ is defined as in (b)). Deduce that if $p$ and $k$ are fixed, then $f(p,r,k)\to 1$ as $r\to\infty$.
\end{itemize}
{\bf Hints:} For (b) use the following facts (make sure to prove them first):
\begin{itemize}
\item[(i)] $g(i)=p^i (1-p)^{r-i}$ is decreasing as a function of $i$
\item[(ii)] $\sum\limits_{i=0}^r {r\choose i}=2^r$.
\end{itemize}
For (c) explain why transmitting a codeword $c\in Rep(k,r)$ and then applying NND is essentially equivalent to independently transmitting $k$ codewords from $Rep(1,r)$ and then applying NND to each of them. Then use this observation to deduce the equality in (c).

{\bf 3.} Problem 3.1, page 36

{\bf 4.} Problem 3.2, page 36

{\bf 5.} 
\begin{itemize}
\item[(a)] Use the Euclidean algorithm to find integers $u$ and $v$ such that $127u+35v=1$. If you have not studied this before, see Lecture~4 of my 3354 notes.
\item[(b)] Use your answer in (a) to compute $35^{-1}$ in $\dbZ_{127}$. Make sure to explain your logic.
\end{itemize}

{\bf 6.} Problem 3.4, page 36. {\bf Hint:} (a) can be proved directly from the standard formula for binomial coefficients and
basic divisibility properties. Then use (a) to solve both (b) and (c).

{\bf 7.} Problem 3.5, page 36

\end{document}



