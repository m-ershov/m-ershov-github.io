\documentclass[12pt]{amsart}

\usepackage{amsmath}
\usepackage{amssymb}
\usepackage{amsthm}
%\usepackage{psfig}

\begin{document}
\baselineskip=16pt
\textheight=8.5in
%\parindent=0pt 
\def\sk {\hskip .5cm}
\def\skv {\vskip .08cm}
\def\cos {\mbox{cos}}
\def\sin {\mbox{sin}}
\def\tan {\mbox{tan}}
\def\intl{\int\limits}
\def\lm{\lim\limits}
\newcommand{\frc}{\displaystyle\frac}
\def\xbf{{\mathbf x}}
\def\fbf{{\mathbf f}}
\def\gbf{{\mathbf g}}

\def\dbA{{\mathbb A}}
\def\dbB{{\mathbb B}}
\def\dbC{{\mathbb C}}
\def\dbD{{\mathbb D}}
\def\dbE{{\mathbb E}}
\def\dbF{{\mathbb F}}
\def\dbG{{\mathbb G}}
\def\dbH{{\mathbb H}}
\def\dbI{{\mathbb I}}
\def\dbJ{{\mathbb J}}
\def\dbK{{\mathbb K}}
\def\dbL{{\mathbb L}}
\def\dbM{{\mathbb M}}
\def\dbN{{\mathbb N}}
\def\dbO{{\mathbb O}}
\def\dbP{{\mathbb P}}
\def\dbQ{{\mathbb Q}}
\def\dbR{{\mathbb R}}
\def\dbS{{\mathbb S}}
\def\dbT{{\mathbb T}}
\def\dbU{{\mathbb U}}
\def\dbV{{\mathbb V}}
\def\dbW{{\mathbb W}}
\def\dbX{{\mathbb X}}
\def\dbY{{\mathbb Y}}
\def\dbZ{{\mathbb Z}}

\def\la{{\langle}}
\def\ra{{\rangle}}
\def\summ{{\sum\limits}}

\bf\centerline{Homework \#6. Due Saturday, March 19th, by 11:59pm in filedrop}\rm
\vskip .1cm
All reading assignments and references to exercises, definitions etc. are from our main book `Coding Theory: A First Course' by Ling and Xing 
\vskip .1cm


\bf\centerline{Reading and plan for the next week: }\rm
\skv
1. For this homework assignment read 5.1-5.4 and 5.7.
\skv
\skv
2. Plan for next week: Tue (Mar 15) -- Plotkin bound (5.6) and Gilbert-Varshamov bound (5.2). Thu (Mar 17) -- Polynomial rings and basic theory of finite fields (3.2 and parts of 3.3; see also online Lecture 15 from Spring 2020).
\skv

\skv
\bf\centerline{Problems: }\rm
\skv
{\bf 1.}  Recall that if $C$ is a code of length $n$ and $I$ is a proper subset of $\{1,\ldots,n\}$, the punctured code $C_I$ is a code of length
$n-|I|$ obtained from $C$ by puncturing the $i^{\rm th}$ coordinate for every $i\in I$ (from every $c\in C$). Prove part (b) of Lemma~13.1 from class: $$d(C_I)\geq d(C)-|I|.$$
\skv

{\bf 2.} In parts (a) and (b) of this problem assume that $d\geq 2$.
\begin{itemize}
\item[(a)] Prove that $A_q(n,d)\leq A_q(n,d-1)$. In other words, fix an alphabet $A$ with $|A|=q$ and prove the following: if there exists
an $(n,M,d)$-code $C$ over $A$, there also exists an $(n,M,d-1)$-code $C'$ over $A$. You should give a precise argument; do not try to say this is obvious or something like that.
\item[(b)] Now prove that $A_q(n,d)\leq A_q(n-1,d-1)$. {\bf Hint:} use punctured codes.
\item[(c)] Now use (b) and induction to give another proof of the Singleton bound: $A_q(n,d)\leq q^{n-d+1}$.
\end{itemize}
\skv

{\bf 3.} Problem 5.7. 
\skv

{\bf 4.}
\begin{itemize}
\item[(a)] Show that the all-one vector $(1, 1,\ldots , 1)$ of length $24$ lies in the extended binary Golay code $G_{24}$.
\item[(b)] Assume without proof that $G_{24}$ contains (exactly) $759$ words of weight $8$. Use this fact and (a) to prove that the distribution of weights in $G_{24}$ is given by Table~5.5 on page 109. 
\end{itemize}
\skv


{\bf 5.} This problem deals with the Golay code $G_{23}$.
\begin{itemize}
\item[(a)] Use Problem~5(b) to prove that possible weights of elements of $G_{23}$ are $0,7,8,11,12,15,16$ and $23$. Make sure to prove that each of those numbers actually arises as the weight of some element of $G_{23}$.
\item[(b)] Let $w\in \dbF_2^{23}$ with $wt(w)=4$. Let $c_{w}\in G_{23}$ be the result of applying NND decoding (with respect to $G_{23}$) to $w$.
Use (a) and the fact that $G_{23}$ is perfect (as shown in class) to prove that $d(w,c_w)=3$ and $wt(c_w)=7$. 
\end{itemize}
\skv

{\bf 6.} Problem~5.19. {\bf Note:} Simplex codes $S(r,q)$ are defined at the end of 5.3.2, page 88.
\skv

{\bf 7.} Use the result of Problem~5.19 to show that the simplex codes $S(r,q)$ attain the Griesmer bound.
\skv
\end{document}


