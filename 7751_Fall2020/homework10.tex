\documentclass[12pt]{amsart}

\usepackage{amsmath}
\usepackage{amssymb}
\usepackage{amsthm}
\usepackage{hyperref}
\usepackage{url}
%\usepackage{psfig}

\begin{document}
\baselineskip=16pt
\textheight=9in
\parindent=0pt 
\def\sk {\hskip .5cm}
\def\skv {\vskip .12cm}
\def\cos {\mbox{cos}}
\def\sin {\mbox{sin}}
\def\tan {\mbox{tan}}
\def\intl{\int\limits}
\def\lm{\lim\limits}
\newcommand{\frc}{\displaystyle\frac}
\def\xbf{{\mathbf x}}
\def\fbf{{\mathbf f}}
\def\gbf{{\mathbf g}}

\def\dbA{{\mathbb A}}
\def\dbB{{\mathbb B}}
\def\dbC{{\mathbb C}}
\def\dbD{{\mathbb D}}
\def\dbE{{\mathbb E}}
\def\dbF{{\mathbb F}}
\def\dbG{{\mathbb G}}
\def\dbH{{\mathbb H}}
\def\dbI{{\mathbb I}}
\def\dbJ{{\mathbb J}}
\def\dbK{{\mathbb K}}
\def\dbL{{\mathbb L}}
\def\dbM{{\mathbb M}}
\def\dbN{{\mathbb N}}
\def\dbO{{\mathbb O}}
\def\dbP{{\mathbb P}}
\def\dbQ{{\mathbb Q}}
\def\dbR{{\mathbb R}}
\def\dbS{{\mathbb S}}
\def\dbT{{\mathbb T}}
\def\dbU{{\mathbb U}}
\def\dbV{{\mathbb V}}
\def\dbW{{\mathbb W}}
\def\dbX{{\mathbb X}}
\def\dbY{{\mathbb Y}}
\def\dbZ{{\mathbb Z}}

\def\la{{\langle}}
\def\ra{{\rangle}}
\def\Aut{{\rm Aut}}
\def\Tor{{\rm Tor}}
\def\Inn{{\rm Inn\,}}
\def\Ker{{\rm Ker\,}}
\def\Im{{\rm Im\,}}
\def\phi{{\varphi}}

\bf\centerline{Homework \#10}\rm
\vskip .1cm
{\bf Plan for next week:} Tensor products of modules and algebras (10.4). See also Lectures 3-5 at
\skv
\url{http://people.virginia.edu/~mve2x/7752_Spring2010/}
\skv


\vskip .1cm
\centerline{\bf Problems, to be submitted by 11:59pm on Sat, November 21st}
\vskip .1cm
\skv

{\bf Convention:} All rings below are assumed to have $1$, and all modules are left modules.
\skv

{\bf 1.} Problem~\#4 from Midterm~\#2 (no need to submit if you received full credit on the midterm).
\skv
{\bf 2.} Let $R$ be a ring and let $M$ be an $R$-module.
\begin{itemize}
\item[(a)] Prove that for any $m\in M$, the map $x\mapsto xm$ from $R$ to $M$ is a homomorphism of $R$-modules
(recall that $R$ is an $R$-module with the left multiplication action).
\item[(b)] Assume that $R$ is commutative, and let $M$ be an $R$-module. Prove that $Hom_R(R,M)\cong M$ as $R$-modules. {\bf Note:} For the definition and justification
of the $R$-module structure on the set $Hom_R(M,N)$ (where $R$ is commutative and $M$ and $N$ are $R$-modules) see Proposition~2 on page 346 in DF. {\bf Hint:}
An element of $Hom_R(R,M)$ is uniquely determined by where it maps $1$.
\end{itemize}
\skv
{\bf 3.} An $R$-module $M$ is called {\it simple (or irreducible)} if
$M$ has no submodules besides $\{0\}$ and $M$. An $R$-module $M$ is called
{\it indecomposable} if $M$ is not isomorphic to $N\oplus P$ for nonzero
$R$-modules $N$ and $P$.
\begin{itemize}
\item[(a)] Prove that every simple module is indecomposable
\item[(b)] Describe (with proof) all simple $\dbZ$-modules and all finitely generated
indecomposable $\dbZ$-modules. Deduce that an indecomposable module need not
be simple.
\end{itemize}
\skv
{\bf 4.} An $R$-module $M$ is called {\it cyclic} if $M$ is generated
(as an $R$-module) by one element.
\begin{itemize}
\item[(a)] Prove that cyclic $R$-modules are precisely the ones which are isomorphic to $R/I$
for some left ideal $I$ of $R$.
\item[(b)] Prove that every simple module is cyclic. Then show that simple
$R$-modules are precisely the ones which are isomorphic to $R/I$
for some maximal left ideal $I$ of $R$.
\end{itemize}
\skv
{\bf 5.} Let $R$ be a commutative domain, and let $I$ be a non-principal ideal of $R$. 
Prove that $I$, considered as an $R$-module (with left-multiplication action) is indecomposable but not cyclic. 
{\bf Hint:} One way to prove that $I$ is indecomposable is to show that any two elements of $I$ 
are linearly dependent over $R$. {\bf Note:} As we will prove in Algebra-II, if $R$ is a principal
ideal domain, every finitely generated indecomposable module is cyclic.
\skv
{\bf 6.} Let $R$ be a commutative ring.
An $R$-module $M$ is called {\it torsion} if for any $m\in M$
there exists nonzero $r\in R$ such that $rm=0$. An $R$-module $M$ is called {\it divisible} 
if for any nonzero $r\in R$ we have $rM=M$. In other words, $M$ is divisible if
for any $m\in M$ and nonzero $r\in R$ there exists $x\in M$ such that $rx=m$.

\begin{itemize}
\item[(a)] Suppose that $M$ is a torsion $R$-module and $N$ is a divisible $R$-module.
Prove that $M\otimes_R N =\{0\}$.

\item[(b)] Let $M=\dbQ/\dbZ$ considered as a $\dbZ$-module. Prove that
$M\otimes _{\dbZ} M=\{0\}$.
\end{itemize}

\skv
{\bf 7.} Let $R$ be a commutative ring, and let $M$ be a free 
$R$-module with basis $e_1,\ldots, e_k$. Prove that the element
$e_1\otimes e_2+e_2\otimes e_1\in M\otimes M$ is not representable as a simple tensor
$m\otimes n$ for some $m,n\in M$. {\bf Note:} You may want to start with the case where $R$ is a domain (where the proof is a little bit easier)
and then think how to modify the argument in the general case.
\end{document}