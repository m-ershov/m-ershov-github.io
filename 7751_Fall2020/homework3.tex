\documentclass[12pt]{amsart}

\usepackage{amsmath}
\usepackage{amssymb}
\usepackage{amsthm}
%\usepackage{psfig}

\begin{document}
\baselineskip=16pt
\textheight=9in
\parindent=0pt 
\def\sk {\hskip .5cm}
\def\skv {\vskip .12cm}
\def\cos {\mbox{cos}}
\def\sin {\mbox{sin}}
\def\tan {\mbox{tan}}
\def\intl{\int\limits}
\def\lm{\lim\limits}
\newcommand{\frc}{\displaystyle\frac}
\def\xbf{{\mathbf x}}
\def\fbf{{\mathbf f}}
\def\gbf{{\mathbf g}}

\def\dbA{{\mathbb A}}
\def\dbB{{\mathbb B}}
\def\dbC{{\mathbb C}}
\def\dbD{{\mathbb D}}
\def\dbE{{\mathbb E}}
\def\dbF{{\mathbb F}}
\def\dbG{{\mathbb G}}
\def\dbH{{\mathbb H}}
\def\dbI{{\mathbb I}}
\def\dbJ{{\mathbb J}}
\def\dbK{{\mathbb K}}
\def\dbL{{\mathbb L}}
\def\dbM{{\mathbb M}}
\def\dbN{{\mathbb N}}
\def\dbO{{\mathbb O}}
\def\dbP{{\mathbb P}}
\def\dbQ{{\mathbb Q}}
\def\dbR{{\mathbb R}}
\def\dbS{{\mathbb S}}
\def\dbT{{\mathbb T}}
\def\dbU{{\mathbb U}}
\def\dbV{{\mathbb V}}
\def\dbW{{\mathbb W}}
\def\dbX{{\mathbb X}}
\def\dbY{{\mathbb Y}}
\def\dbZ{{\mathbb Z}}

\def\la{{\langle}}
\def\ra{{\rangle}}
\def\Aut{{\rm Aut}}
\def\phi{{\varphi}}

\bf\centerline{Homework \#3}\rm
\vskip .1cm
{\bf Plan for next week:} Finite fields (\S~9.5), Hilbert's basis Theorem (\S~9.6).
\vskip .1cm
\centerline{\bf Problems, to be submitted by 11:59pm on Thu, September 17th}
\vskip .1cm
\skv
{\bf 1.} Let $R$ and $S$ be rings with $1$, and let $\phi:R\to S$ be a ring homomorphism such that
$\phi(1_R)=1_S$.
\begin{itemize}
\item[(a)] Prove that $\phi(R^{\times})\subseteq S^{\times}$
\item[(b)] Give an example where $\phi$ is surjective, but $\phi(R^{\times})\neq S^{\times}$.
\end{itemize}

\skv
{\bf 2.} Let $m,n\in\dbN$ with $m\mid n$, and define $f:\dbZ/n\dbZ\to \dbZ/m\dbZ$ by
$$f(x+n\dbZ)=x+m\dbZ$$ (note that $f$ is well defined precisely because $m\mid n$). Prove that the associated map of the groups of units $f:(\dbZ/n\dbZ)^{\times}\to (\dbZ/m\dbZ)^{\times}$ is surjective.
\skv
{\bf Hint:} First prove this directly for special values of $n$ and $m$ (using the standard characterization of units in $\dbZ/k\dbZ$
from HW\#1.3). Then use another homework problem to do the general case.
\skv

{\bf 3.} Use HW\#2.5 to find all maximal ideals of $\dbZ[x]$ (the ring of polynomials over $\dbZ$ in one variable) which contain 
$x^2+1$ and $15$. You will need to use standard results about primes (=irreducible) elements in $\dbZ[i]$ -- see the corresponding section in DF (pp. 289-292).
\skv

{\bf 4.} Let $R=\dbZ[\sqrt{5}]=\{a+b\sqrt{5} : a,b\in\dbZ\}$. Find an element of $R$
which is irreducible but not prime and deduce that $R$ is not a unique factorization domain (UFD).
\skv
 {\bf Hint:} Consider the equality $2\cdot 2=(\sqrt{5}+1)(\sqrt{5}-1)$. In order to check
whether some element of $R$ is irreducible it is convenient to use the standard norm function
$N:R\to\dbZ_{\geq 0}$ given by $N(a+b\sqrt{5})=|(a+b\sqrt{5})(a-b\sqrt{5})|=|a^2-5b^2|$ (just as in the $\dbZ[\sqrt{3}]$ example considered in class,
it is easy to check that $N(uv)=N(u)N(v)$).
\skv\skv 
{\bf 5.} Let $R=\dbZ+x\dbQ[x]$, the subring of $\dbQ[x]$ consisting of
polynomials whose constant term is an integer.
\begin{itemize}
\item[(a)] Show that the element $\alpha x$, with $\alpha\in\dbQ$ is NOT irreducible
in $R$. Then show that $x$ cannot be written as a product of irreducibles in $R$.
Note that by Lecture~5, this implies that $R$ is not Noetherian.

\item[(b)] Now prove directly that $R$ is not Noetherian by showing that
$I=x\dbQ[x]$ is an ideal of $R$ which is not finitely generated.

\item[(c)] Give an example of a non-Noetherian domain which is a UFD.
\end{itemize}
\skv
{\bf 6.} Let $R$ be a commutative ring with $1$. 
\begin{itemize}
\item[(a)]   Let $M$ be an ideal of $R$. Prove that the following conditions are equivalent:
\begin{itemize}
\item[(i)] $M$ is the unique maximal ideal of $R$ 
\item[(ii)] every element of $R\setminus M$ is invertible. 
\end{itemize}
Rings satisfying these equivalent conditions are called local.
\item[(b)] Let $F$ be a field and $F[[x]]$ the ring of power series over $F$. Prove that $F[[x]]$ is local.
\item[(c)] Now let $R$ be arbitrary, let $P$ be a prime ideal of $R$ and $S=R\setminus P$. Prove that the localization
$S^{-1}R$ is local and explicitly describe its unique maximal ideal.
\end{itemize}
\end{document}
