\documentclass[12pt]{amsart}

\usepackage{amsmath}
\usepackage{amssymb}
\usepackage{amsthm}
%\usepackage{psfig}

\begin{document}
\baselineskip=16pt
\textheight=9in
\parindent=0pt 
\def\sk {\hskip .5cm}
\def\skv {\vskip .12cm}
\def\cos {\mbox{cos}}
\def\sin {\mbox{sin}}
\def\tan {\mbox{tan}}
\def\intl{\int\limits}
\def\lm{\lim\limits}
\newcommand{\frc}{\displaystyle\frac}
\def\xbf{{\mathbf x}}
\def\fbf{{\mathbf f}}
\def\gbf{{\mathbf g}}

\def\dbA{{\mathbb A}}
\def\dbB{{\mathbb B}}
\def\dbC{{\mathbb C}}
\def\dbD{{\mathbb D}}
\def\dbE{{\mathbb E}}
\def\dbF{{\mathbb F}}
\def\dbG{{\mathbb G}}
\def\dbH{{\mathbb H}}
\def\dbI{{\mathbb I}}
\def\dbJ{{\mathbb J}}
\def\dbK{{\mathbb K}}
\def\dbL{{\mathbb L}}
\def\dbM{{\mathbb M}}
\def\dbN{{\mathbb N}}
\def\dbO{{\mathbb O}}
\def\dbP{{\mathbb P}}
\def\dbQ{{\mathbb Q}}
\def\dbR{{\mathbb R}}
\def\dbS{{\mathbb S}}
\def\dbT{{\mathbb T}}
\def\dbU{{\mathbb U}}
\def\dbV{{\mathbb V}}
\def\dbW{{\mathbb W}}
\def\dbX{{\mathbb X}}
\def\dbY{{\mathbb Y}}
\def\dbZ{{\mathbb Z}}

\def\la{{\langle}}
\def\ra{{\rangle}}
\def\Aut{{\rm Aut}}
\def\phi{{\varphi}}

\bf\centerline{Homework \#2}\rm
\vskip .1cm
{\bf Plan for next week:} PIDs, continued (\S~8.2), UFDs (\S~8.3), polynomial rings over UFDs (\S~9.3).
\vskip .1cm
\centerline{\bf Problems, to be submitted by 11:59pm on Thu, September 10th}
\vskip .1cm

{\bf 1.} Let $R$ be a finitely generated ring (not necessarily with $1$). Use Zorn's lemma to show that $R$ has a  
maximal subring (by definition a maximal subring is a maximal element of the set of  
proper subrings of $R$ partially ordered by inclusion). Give a detailed argument. 
%{\bf Hint:} The key step is to show 
%that  if $\mathcal C$ is a chain of proper subrings of $G$, then the union of subrings in this 
%chain is not the entire $R$.
\skv
{\bf 2.} Let $R$ be a commutative ring with $1$. The {\it nilradical} of $R$ denoted $Nil(R)$ is the set of all nilpotent elements of $R$, that is $$Nil(R)=\{a\in R : a^n=0 \mbox{ for some }n\in\dbN\}.$$ Prove that 
\begin{itemize}
\item[(a)] $Nil(R)$ is an ideal of $R$ 
\item[(b)] $Nil(R)$ is contained in every prime ideal of $R$. Later we will show that $Nil(R)$ is equal to the intersection of all prime ideals.
\end{itemize}
\skv
{\bf 3.} Let $R$ be a commutative ring with $1$, and let $D$ be a subset of $R$ which is closed under multiplication such that 
$1\in D$ and $0\not\in D$.
\begin{itemize}
\item[(a)] Prove that the relation $\sim$ in the definition of rings of fractions $RD^{-1}$ is an equivalence relation
\item[(b)] The operations $+$ and $\cdot$ on $RD^{-1}$ (given by $\frac{r_1}{d_1}+\frac{r_2}{d_2}=\frac{r_1 d_2+r_2 d_1}{d_1 d_2}$
and $\frac{r_1}{d_1}\cdot\frac{r_2}{d_2}=\frac{r_1 r_2}{d_1 d_2}$) are well defined.
\end{itemize}
\skv
{\bf 4.} Let $R=\dbZ_{14}$, $D=\{\bar 1,\bar 2,\bar 4,\bar 8\}$ (note that $D$ is multiplicatively closed
but it does contain zero divisors). Prove that the localization $RD^{-1}$ is isomorphic to $\dbZ_{7}$.
{\bf Hint:} What can you say about the map $\iota:R\to RD^{-1}$ in this case?
\skv
\skv
{\bf 5.} Let $\dbZ[i]=\{a+bi: a,b\in\dbZ\}$ be the ring of Gaussian integers.
\begin{itemize}
\item[(a)] Prove that $\dbZ[i]$ is a Euclidean domain.
\item[(b)] Prove that $\dbZ[i]\cong \dbZ[x]/(x^2+1)$
\end{itemize}
\skv
{\bf 6.} Let $D$ be a positive integer such that $D\equiv 3\mod 4$, and let
$R=\dbZ[\frac{1+\sqrt{-D}}{2}]$, that is, $R$ is the minimal subring of $\dbC$ containing $\dbZ$ and $\frac{1+\sqrt{-D}}{2}$.

\begin{itemize}
\item[(a)] Prove that $R=\{a+ b \frac{1+\sqrt{-D}}{2} : a,b\in\dbZ \}$. It should be clear from your argument where the assumption $D\equiv 3\mod 4$ is used (otherwise the result is simply not true).

\item[(b)] Assume that $D=3, 7$ or $11$. Prove that $R$ is a Euclidean domain. {\bf Hint:} It is important to use the right Euclidean norm $N$. Before claiming that your $N$ is multiplicative, better check it is actually true. Your argument should ``barely'' work for
$D=11$; if it still ``works'' for $D=15$, it is a wrong argument.
\end{itemize} 
\end{document}
