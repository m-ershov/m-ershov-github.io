\documentclass[12pt]{amsart}

\usepackage{amsmath}
\usepackage{amssymb}
\usepackage{amsthm}
%\usepackage{psfig}

\begin{document}
\baselineskip=16pt
\textheight=9in
\parindent=0pt 
\def\sk {\hskip .5cm}
\def\skv {\vskip .12cm}
\def\cos {\mbox{cos}}
\def\sin {\mbox{sin}}
\def\tan {\mbox{tan}}
\def\intl{\int\limits}
\def\lm{\lim\limits}
\newcommand{\frc}{\displaystyle\frac}
\def\xbf{{\mathbf x}}
\def\fbf{{\mathbf f}}
\def\gbf{{\mathbf g}}

\def\dbA{{\mathbb A}}
\def\dbB{{\mathbb B}}
\def\dbC{{\mathbb C}}
\def\dbD{{\mathbb D}}
\def\dbE{{\mathbb E}}
\def\dbF{{\mathbb F}}
\def\dbG{{\mathbb G}}
\def\dbH{{\mathbb H}}
\def\dbI{{\mathbb I}}
\def\dbJ{{\mathbb J}}
\def\dbK{{\mathbb K}}
\def\dbL{{\mathbb L}}
\def\dbM{{\mathbb M}}
\def\dbN{{\mathbb N}}
\def\dbO{{\mathbb O}}
\def\dbP{{\mathbb P}}
\def\dbQ{{\mathbb Q}}
\def\dbR{{\mathbb R}}
\def\dbS{{\mathbb S}}
\def\dbT{{\mathbb T}}
\def\dbU{{\mathbb U}}
\def\dbV{{\mathbb V}}
\def\dbW{{\mathbb W}}
\def\dbX{{\mathbb X}}
\def\dbY{{\mathbb Y}}
\def\dbZ{{\mathbb Z}}

\def\la{{\langle}}
\def\ra{{\rangle}}
\def\Aut{{\rm Aut}}
\def\Inn{{\rm Inn\,}}
\def\Ker{{\rm Ker\,}}
\def\Im{{\rm Im\,}}
\def\phi{{\varphi}}

\bf\centerline{Homework \#4}\rm
\vskip .1cm
{\bf Plan for next week:} Group actions (\S~4.1-4.3). Start direct and semi-direct products (\S~5.4-5.5).

\vskip .1cm
\centerline{\bf Problems, to be submitted by 11:59pm on Thu, October 1st}
\vskip .1cm
\skv
{\bf 1.} Let $k$ be an algebraically closed field.
Let $Y$ be a subset of $k^n$. Let 
$$k[Y]=k[x_1,\ldots, x_n]/I(Y).$$ 
The ring $k[Y]$ can be naturally identified with the ring of polynomial functions from $Y$ to $k$ (with pointwise addition and multiplication). Indeed, every polynomial in $k[x_1,\ldots, x_n]$ naturally defines a function
from $Y$ to $k$. Thus, we have a map $ev_Y: k[x_1,\ldots, x_n]\to Func(Y,k)$. By {\bf polynomial functions from $Y$ to $k$} we mean
exactly the functions in $\Im(ev_Y)$. It is straightforward to check that $ev_Y$ is a ring homomorphism and thus 
$$\Im(ev_Y)\cong k[x_1,\ldots, x_n]/\Ker(ev_Y)$$
Finally, by definition $\Ker(ev_Y)=I(Y)$, and thus $k[Y]\cong \Im(ev_Y)$.

\sk Let $O(Y)$ be the set of all everywhere defined rational functions on $Y$, that is,
all functions $f:Y\to k$ for which there exist polynomials $p,q\in k[x_1,\ldots, x_n]$
s.t. $q$ does not vanish at any point of $Y$ and $f=p/q$ as a function on $Y$. Clearly,
$k[Y]\subseteq O(Y)$.
\begin{itemize}
\item[(a)] Prove that if $Y$ is an algebraic set, then $O(Y)=k[Y]$. {\bf Hint:} Use the weak
Nullstellensatz, version 2.
\item[(b)] Let $Y=k^1\setminus\{0\}$, the affine line with $0$ removed. Prove that
$k[Y]=k[x]$ (polynomials in one variable) while $O(Y)=k[x,1/x]$.
\item[(c)] Find an algebraic subset $Z$ of $k^2$ such that $k[Z]\cong k[x,1/x]$.
How is $Z$ related to $Y$ from part (b)?
\item[(d)] Find a non-algebraic subset $W$ of $k^2$ for which $O(W)=k[W]\cong k[x_1,x_2]$.
\end{itemize}
\skv
{\bf 2.} Let $R$ be a commutative ring with $1$, and let $n\in\dbN$. Let $End(R^n)$ be the set of all endomorphism of the additive group $(R^n,+)$.
\begin{itemize}
\item[(a)] Define $\iota: Mat_n(R)\to End(R^n)$ by $\iota(A)=(v\mapsto Av)$ (or, in more elementary notation,
$(\iota(A)(v))=Av$ for all $v\in R^n$). Here we think of elements of $R^n$ as column vectors. Prove that
$\iota$ is an injective homomorphism of monoids.
\item[(b)] Now define $\Phi: End(R^n)\to Mat_n(R)$ as in Lecture~10 (in Lecture~10 we dealt with the special case $R=\dbZ$,
but definition remains the same -- $e_i$ is still the element of $R^n$ whose $i^{\rm th}$ coordinate is $1$ and other coordinates
are $0$). Prove that $\Phi\circ \iota$ is the identity map on $Mat(R^n)$. Deduce that $\Phi$ is always surjective.
\item[(c)] By (a) and (b) $\Phi$ is an injective $\iff$ $\Phi$ is an isomorphism $\iff$ $\iota$ is an isomorphism.
In Lecture~10 we observed that $\Phi$ is injective for $R=\dbZ$. Determine (with proof) if $\Phi$ is injective 
for each of the following rings: (i) $R=\dbZ_m$ for some $m\in\dbN$, (ii) $R=\dbQ$, (iii) $R=\dbZ[\sqrt{2}]$. 
\end{itemize}
Note that whenever $\Phi$ is an isomorphism, by taking the units on both sides, we get that $\Aut(R^n)\cong GL_n(R)$.

\skv
{\bf 3.} Let $G$ be a group. For each $g\in G$ let $\iota_g: G\to G$
be the conjugation by $g$, that is, $\iota_g(x)=gxg^{-1}$.
Recall that $\iota_g\in \Aut(G)$ for any $g\in G$ and
the mapping $\iota: G\to \Aut(G)$ given by $\iota(g)=\iota_g$
is a homomorphism. Elements of the subgroup $\Inn(G)=\iota(G)$ of $\Aut(G)$
are called inner automorphisms.

\begin{itemize}
\item[(a)] Prove that for any $g\in G$ and $\sigma\in \Aut(G)$
one has $\sigma \iota_g \sigma^{-1}=\iota_{\sigma(g)}$.
Deduce that $\Inn(G)$ is a normal subgroup of $\Aut(G)$.

\item[(b)] Let $H$ be a normal subgroup of $G$. Note that for each $g\in G$, 
the mapping $\iota_g$ restricted to $H$
is an automorphism of $H$. By slight abuse of notation
we denote this automorphism of $H$ by $\iota_g$ as well.
Prove that $\iota_g$ is an inner automorphism of $H$ if and
only if $g\in H\cdot C_{G}(H)$ where $C_G(H)$ is the centralizer
of $H$ in $G$.
\end{itemize}


\skv
{\bf 4.} Find the minimal $n$ for which the symmetric group $S_n$ contains an element of order $15$
(and prove rigorously why your $n$ is indeed minimal). {\it Note: }All you need to know about $S_n$ for this
problem is stated in Section 1.3 of DF (pp.29-32).

\skv
{\bf 5.} Let $G=D_8$, the dihedral group of order $8$ (that is, the group of isometries of a square). Prove that $|[G,G]|=2$ and describe $[G,G]$
explicitly without computing every single commutator. 
\skv
\skv
{\bf Index of a subgroup.} If $G$ is a group and $H$ is a subgroup of $G$, the index of $H$ in $G$, denoted by $[G:H]$, is
defined to be the cardinality of $G/H$, that is, the number of left cosets of $H$ in $G$.
It is not hard to show that the sets $G/H$ (the set of left cosets of $H$) and
$H\setminus G$ (the set of right cosets of $H$) always have the same cardinality, so there is no need to introduce ``left index'' and ``right index''. 

The full statement of Lagrange theorem asserts that if $G$ is a finite group and $H$ is a subgroup of $G$, then $[G:H]=\frac{|G|}{|H|}$ (typically one applies not the full statement but its most useful consequence, namely, that the order of $H$ divides the order of $G$).
\skv




{\bf 6.} Let $G$ be a group and let $H$ and $K$ be subgroups of $G$ of finite index
(note that $G$ is not assumed to be finite).

\begin{itemize}

\item[(a)] Assume that $H\subseteq K$. Prove that $[G:H]=[G:K][K:H]$.

\item[(b)] Let $m=[G:H]$ and $n=[G:K]$. Prove that
$${\rm LCM}(m,n)\leq [G:H\cap K]\leq  mn$$ 
(where ${\rm LCM}$ is the least common multiple).
\end{itemize}


{\bf Hint for (a):} If $A$ is a group and $B$ a subgroup of $A$, a
subset $S$ of $A$ is called a left transversal of $B$ in $A$
if $S$ contains precisely one element from each left coset of $B$
(an alternative name for a transversal is a system of left coset representatives).
Let $\{g_1,\ldots, g_r\}$ be a left transversal of $K$ in $G$
and $\{k_1,\ldots, k_s\}$ a left transversal of $H$ in $K$. Prove
that $\{g_i k_j\}_{1\leq i\leq r, 1\leq j\leq s}$ is a left transversal for $H$ in $G$.
Recall that if $B$ is a subgroup of a group $G$, then $xB=yB$ $\iff$ $x^{-1}y\in B$
for $x,y\in G$.
\end{document}
