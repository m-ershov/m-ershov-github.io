\documentclass[12pt]{amsart}

\usepackage{amsmath}
\usepackage{amssymb}
\usepackage{amsthm}
\usepackage{hyperref}
\usepackage{url}

%\usepackage{psfig}

\newtheorem* {Theorem}    {Theorem}
\newtheorem* {Lemma}    {Lemma}


\begin{document}
 \pagenumbering{gobble}
\baselineskip=16pt
\textheight=8.5in
\textwidth=6.5in
%\parindent=0pt 
\def\sk {\hskip .5cm}
\def\skv {\vskip .08cm}
\def\cos {\mbox{cos}}
\def\sin {\mbox{sin}}
\def\tan {\mbox{tan}}
\def\intl{\int\limits}
\def\lm{\lim\limits}
\newcommand{\frc}{\displaystyle\frac}
\def\xbf{{\mathbf x}}
\def\fbf{{\mathbf f}}
\def\gbf{{\mathbf g}}

\def\dbA{{\mathbb A}}
\def\dbB{{\mathbb B}}
\def\dbC{{\mathbb C}}
\def\dbD{{\mathbb D}}
\def\dbE{{\mathbb E}}
\def\dbF{{\mathbb F}}
\def\dbG{{\mathbb G}}
\def\dbH{{\mathbb H}}
\def\dbI{{\mathbb I}}
\def\dbJ{{\mathbb J}}
\def\dbK{{\mathbb K}}
\def\dbL{{\mathbb L}}
\def\dbM{{\mathbb M}}
\def\dbN{{\mathbb N}}
\def\dbO{{\mathbb O}}
\def\dbP{{\mathbb P}}
\def\dbQ{{\mathbb Q}}
\def\dbR{{\mathbb R}}
\def\dbS{{\mathbb S}}
\def\dbT{{\mathbb T}}
\def\dbU{{\mathbb U}}
\def\dbV{{\mathbb V}}
\def\dbW{{\mathbb W}}
\def\dbX{{\mathbb X}}
\def\dbY{{\mathbb Y}}
\def\dbZ{{\mathbb Z}}

\def\la{{\langle}}
\def\ra{{\rangle}}
\def\Ker{{\rm Ker}}
\def\Aut{{\rm Aut}}
\def\Out{{\rm Out}}
\def\Inn{{\rm Inn}}
\def\IA{{\rm IA}}
\def\rk{{\rm rk}}
\def\summ{{\sum\limits}}
\def\phi{{\varphi}}

\bf\centerline{Math 8851. Homework \#4. To be completed by Thu, Feb 23}\rm
\vskip .1cm

\skv
{\bf 1.} Prove that for any integer $n\geq 2$ the matrices
$\begin{pmatrix} 1 & n \\ 0 & 1
\end{pmatrix}$ and $\begin{pmatrix} 1 & 0 \\ n & 1
\end{pmatrix}$ generate a free group of rank two.
{\bf Hint:} Consider the natural action of $SL_2(\dbZ)$ on $\dbZ^2$
and apply the Ping-Pong Lemma with suitable subsets $X_1$ and $X_2$
(there exist $X_1$ and $X_2$ which work for every $n\geq 2$).
\skv

{\bf 2.} Use the isomorphism $PSL_2(\dbZ)\cong \dbZ/2\dbZ \ast \dbZ/3\dbZ$
established in Lecture~9 to prove that $$SL_2(\dbZ)=\la A,B \mid A^4=1, A^2=(AB)^3\ra$$
where $A=\begin{pmatrix} 0 & -1 \\ 1 & 0
\end{pmatrix}$ and $B=\begin{pmatrix} 1 & 1 \\ 0 & 1
\end{pmatrix}$.
{\bf Hint:} Show that in the group given by the presentation 
$\la a,b \mid a^4=1, a^2=(ab)^3\ra$ the element $a^2$ is central
and has order $2$.
\skv
Deduce that $SL_2(\dbZ)$ decomposes as the amalgam 
$\dbZ/4\dbZ \ast_{\dbZ/2\dbZ} \dbZ/6\dbZ$.
\skv

{\bf 3.} The goal of this problem is to give another proof of the equality
$\IA_2=\Inn(F_2)$ using the result of Problem~2.

Define $\Out^+(F_2)=\Aut^+(F_2)/\Inn(F_2)$. Since $\Aut^+(F_2)$ is generated
by the Nielsen maps $R_{12}, R_{21}, L_{12}, L_{21}$ by HW\#3.3, $\Out^+(F_2)$
is generated by their images in $\Out^+(F_2)$ which will denote by 
the same symbols with bars ($\overline{R_{12}}$ etc.).


\begin{itemize}

\item[(a)] Prove that $\Out^+(F_2)$ is already generated by $\overline{R_{12}}$
and $\overline{R_{21}}$. Deduce that $\Out^+(F_2)$ is generated by
$\overline{\alpha}$ and $\overline{\beta}$ where $\beta=R_{21}$ and 
$\alpha$ is given by $\alpha(x_1)=x_2, \alpha(x_2)=x_1^{-1}$.

\item[(b)] Verify by direct computation that $(\overline \alpha)^4=1$ and $(\overline \alpha)^2=(\overline \alpha\overline \beta)^3$. Deduce from 
Problem~2 that there exists an epimorphism $\phi:SL_2(\dbZ)\to \Out^+(F_2)$
such that $\phi(A)=\overline \alpha$ and $\phi(B)=\overline \beta$.


\item[(c)] Since $\Inn(F_2)\subseteq \IA_2$, there is a natural
projection map from $\Out^+(F_2)=\Aut^+(F_2)/\Inn(F_2)$ 
to $\Aut^+(F_2)/\IA_2$, and as explained in Lecture~8, $\Aut^+(F_2)/\IA_2$
is naturally isomorphic to $SL_2(\dbZ)$. Thus we obtain an epimorphism $\pi:\Out^+(F_2)\to SL_2(\dbZ)$. Check that $\pi(\overline \alpha)=A$
and $\pi(\overline \beta)=B$. Combining this with $\phi$ from (b),
deduce that $\pi$ is an isomorphism and therefore $\IA_2=\Inn(F_2)$.
\end{itemize}

{\bf 4.} Complete the proof (started in Lecture~10) of the fact that the graph $\Gamma$ associated to an amalgamated free product is a tree.

Recall the setup from class: $G=P\ast_{A}Q$. The vertex set of $\Gamma$
is $V(\Gamma)=G/P\sqcup G/Q$, the edge set is $E(\Gamma)=G/A$, and for every
edge $gA$ its initial vertex $\alpha(gA)$ and end vertex $\omega(gA)$ are 
$\alpha(gA)=gP$ and $\omega(gA)=gQ$.

We observed that any cycle in $\Gamma$ must have even length, say $2k$.
Let $v_0,v_1,\ldots, v_{2k}=v_0$ by the vertices in the cycle. WOLOG 
$v_0=gP$ for some $g\in G$. In class we argued that there exist
elements $p_1,\ldots, p_k\in P$ and $q_1,\ldots, q_k\in Q$ such that
$v_{2m}=g\prod\limits_{i=1}^m p_iq_i P$ and $v_{2m-1}=g(\prod\limits_{i=1}^{m-1} p_iq_i)p_m Q$
for all $1\leq m\leq k$. We deduced that $\prod\limits_{i=1}^k p_iq_i=p$ for some
$p\in P$. If none of the elements $p_i,q_i$ lie in $A$, we obtained a contradiction with the
uniqueness of normal forms.

To finish the proof it remains to show that if some $p_i$ or $q_i$ lies in $A$,
then the cycle must have backtracking (that is, two consecutive edges are inverse
to each other). This is what you need to prove in this problem. Most likely, 
you will first need to show that $\Gamma$ does not have multiple edges.


\end{document}



