\documentclass[12pt]{amsart}

\usepackage{amsmath}
\usepackage{amssymb}
\usepackage{amsthm}
\usepackage{hyperref}
\usepackage{url}

%\usepackage{psfig}

\newtheorem* {Theorem}    {Theorem}
\newtheorem* {Lemma}    {Lemma}


\begin{document}
 \pagenumbering{gobble}
\baselineskip=16pt
\textheight=8.5in
\textwidth=6.5in
%\parindent=0pt 
\def\sk {\hskip .5cm}
\def\skv {\vskip .08cm}
\def\cos {\mbox{cos}}
\def\sin {\mbox{sin}}
\def\tan {\mbox{tan}}
\def\intl{\int\limits}
\def\lm{\lim\limits}
\newcommand{\frc}{\displaystyle\frac}
\def\xbf{{\mathbf x}}
\def\fbf{{\mathbf f}}
\def\gbf{{\mathbf g}}

\def\dbA{{\mathbb A}}
\def\dbB{{\mathbb B}}
\def\dbC{{\mathbb C}}
\def\dbD{{\mathbb D}}
\def\dbE{{\mathbb E}}
\def\dbF{{\mathbb F}}
\def\dbG{{\mathbb G}}
\def\dbH{{\mathbb H}}
\def\dbI{{\mathbb I}}
\def\dbJ{{\mathbb J}}
\def\dbK{{\mathbb K}}
\def\dbL{{\mathbb L}}
\def\dbM{{\mathbb M}}
\def\dbN{{\mathbb N}}
\def\dbO{{\mathbb O}}
\def\dbP{{\mathbb P}}
\def\dbQ{{\mathbb Q}}
\def\dbR{{\mathbb R}}
\def\dbS{{\mathbb S}}
\def\dbT{{\mathbb T}}
\def\dbU{{\mathbb U}}
\def\dbV{{\mathbb V}}
\def\dbW{{\mathbb W}}
\def\dbX{{\mathbb X}}
\def\dbY{{\mathbb Y}}
\def\dbZ{{\mathbb Z}}

\def\la{{\langle}}
\def\ra{{\rangle}}
\def\Ker{{\rm Ker\,}}
\def\Aut{{\rm Aut}}
\def\Out{{\rm Out}}
\def\Inn{{\rm Inn}}
\def\IA{{\rm IA}}
\def\rk{{\rm rk}}
\def\summ{{\sum\limits}}
\def\phi{{\varphi}}

\bf\centerline{Math 8851. Homework \#5. To be completed by Thu, Mar 2}\rm
\vskip .1cm

\skv
{\bf 1.} Let $G$ be a group. Prove that the following are equivalent:
\begin{itemize}
\item[(a)] $G$ arises as an HNN-extension, that is, there exists a group $H$, isomorphic subgroups
$A$ and $B$ of $H$ and an isomorphism $\phi:A\to B$ such that $G\cong \la H,t\mid t^{-1}at=\phi(a)
\mbox{ for all } a\in A\ra$.
\item[(b)] There exists an epimorphism $\pi:G\to \dbZ$.
\end{itemize}
{\bf Hint:} For (b)$\Rightarrow$(a) use the fact that $\dbZ$ is a free group (of rank $1$).
\skv
{\bf 2.} Prove that if $G$ is an HNN extension, then $G$ acts on the associated tree $T$ without
edge inversions. {\bf Note:} This is related to Problem~1. I am not sure you can use the result of 
Problem~1, but solution to 1 should definitely help.

\skv 
{\bf 3.} Let $A_1,\ldots, A_k$ be a finite collection of groups, and consider the natural
epimorphism $\pi$ from the free product $\ast_{i=1}^k A_i= A_1\ast A_2\ast\ldots \ast A_k$ to the direct product $\prod_{i=1}^k A_i$. The group $C=\Ker\pi$ is called the {\it Cartesian subgroup} of $\ast_{i=1}^k A_i$. The Cartesian subgroup $C$ is always free -- this is an immediate consequence of the Kurosh Subgroup Theorem which we will discuss next week (see the statement at the end of the problem). Note that if $A_1,\ldots, A_k$ are finite, then $G=\ast_{i=1}^k A_i$ is finitely generated and $C$ has finite index in $G$ and hence $C$ is also finitely generated (by the Schreier formula). The goal of this problem is to compute the rank of $C$ in this special case.

The rank can be computed using the main theorem of Bass-Serre theory (which we will also discuss next week),
but in this exercise we take a different approach based on the notion of rational Euler characterstic which is introduced below.
\skv
Let $\Omega$ be the smallest class of groups such that
\begin{itemize}
\item[(i)] $\Omega$ contains the trivial group $\{1\}$ and $\dbZ$
\item[(ii)] $\Omega$ is closed under finite direct products
\item[(iii)] $\Omega$ is closed under finite free products
\item[(iv)] $\Omega$ is closed under taking finite index subgroups
and finite index supergroups
\end{itemize}
For each group $G\in\Omega$ one can uniquely define the {\bf rational Euler characteristic}
$\chi(G)\in\dbQ$ such that the following properties hold:
\begin{itemize}
\item[(a)] $\chi(\{1\})=1$ and $\chi(\dbZ)=0$
\item[(b)] $\chi(G\ast H)=\chi(G)+ \chi(H)-1$ for any $G,H\in\Omega$ 
\item[(c)] $\chi(G\times H)=\chi(G)\chi(H)$ for any $G,H\in\Omega$
\item[(d)] If $G\in\Omega$ and $H$ is a subgroup of index $n$ in $G$,
then $\chi(H)=n\chi(G)$.
\end{itemize}
The basic idea is that if $G$ is a group which has a finite CW-complex $X$
as its classifying space, then one should have $\chi(G)=\chi(X)$, but
then the definition of Euler characteristic has to be extended
to a larger class of groups. An explanation of how this can be done
is given in the following paper: 

C.T.C. Wall, {\it Rational Euler characteristics.} 
Proc. Cambridge Philos. Soc. 57 1961 182--184.
\skv
Now the actual problem begins. Let $A_1,\ldots, A_k$ be finite groups, set $n_i=|A_i|$,
and let $F_r$ be a free group of rank $r$. Let  $G=A_1\ast A_2\ast\ldots\ast A_k\ast F_r$, 
and let $C$ be the kernel of the natural epimorphism $G\to A_1\times A_2\times\ldots\times A_k$
which sends $F_r$ to $\{1\}$ (if $r=0$, then $G=A_1\ast A_2\ast\ldots\ast A_k$ and
$C$ is exactly the Cartesian subgroup).
\begin{itemize}
\item[(a)] Prove that $C$ is free (using the Kurosh Subgroup Theorem, see the statement below).
\item[(b)] Use Euler characteristic to prove that $$rk(C)=\prod_{i=1}^k n_i (r+k-1-\sum_{i=1}^k\frac{1}{n_i})+1.$$
%\item[(c)] Now assume that $k=2$ and $r=0$. Prove that $C$ is freely generated
%by the set $\{[a,b]: a\in A_1\setminus\{1\},b\in A_2\setminus\{1\}\}$.
\end{itemize}

\begin{Theorem}[Kurosh Subgroup Theorem] Let $G=\ast_{i\in I} G_{i}$ be the free product of some
family of groups $\{G_{i}: i\in I\}$ (here $I$ may be infinite). Then any subgroup of
$G$ can be decomposed as a free product $F\ast(\ast_{\alpha\in J} H_{\alpha})$ where $F$ is free and
each $H_{\alpha}$ is conjugate (in $G$) to a subgroup of $G_{i}$ for some $i$. Note that several
$H_{\alpha}$ may be conjugate to (distinct) subgroups of the same $G_i$.
\end{Theorem}


{\bf 4.} %(a) Explain why for any $n\in\dbN$ there are only finitely many isomorphism
%classes of groups of order $n$.
Let $G$ be a finitely generated group.
\begin{itemize}
\item[(a)] Prove that for any finite group $H$ there are only finitely many homomorphisms 
from $G$ to $H$. {\bf Hint:} A homomorphism from $G$ is completely determined by its values 
on generators.
\item[(b)] Prove that for any $n\in\dbN$ there are only finitely many normal subgroups 
of index $n$ in $G$. Then deduce that $G$ has only finitely many subgroups of index $n$.
\end{itemize}

\end{document}



