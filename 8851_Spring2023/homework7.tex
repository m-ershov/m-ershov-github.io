\documentclass[12pt]{amsart}

\usepackage{amsmath}
\usepackage{amssymb}
\usepackage{amsthm}
\usepackage{hyperref}
\usepackage{url}

%\usepackage{psfig}

\newtheorem* {Theorem}    {Theorem}
\newtheorem* {Lemma}    {Lemma}
\newtheorem* {Remark}	{\bf{Remark}}

\begin{document}
 \pagenumbering{gobble}
\baselineskip=16pt
\textheight=8.5in
\textwidth=6.5in
%\parindent=0pt 
\def\sk {\hskip .5cm}
\def\skv {\vskip .08cm}
\def\cos {\mbox{cos}}
\def\sin {\mbox{sin}}
\def\tan {\mbox{tan}}
\def\intl{\int\limits}
\def\lm{\lim\limits}
\newcommand{\frc}{\displaystyle\frac}
\def\xbf{{\mathbf x}}
\def\fbf{{\mathbf f}}
\def\gbf{{\mathbf g}}

\def\dbA{{\mathbb A}}
\def\dbB{{\mathbb B}}
\def\dbC{{\mathbb C}}
\def\dbD{{\mathbb D}}
\def\dbE{{\mathbb E}}
\def\dbF{{\mathbb F}}
\def\dbG{{\mathbb G}}
\def\dbH{{\mathbb H}}
\def\dbI{{\mathbb I}}
\def\dbJ{{\mathbb J}}
\def\dbK{{\mathbb K}}
\def\dbL{{\mathbb L}}
\def\dbM{{\mathbb M}}
\def\dbN{{\mathbb N}}
\def\dbO{{\mathbb O}}
\def\dbP{{\mathbb P}}
\def\dbQ{{\mathbb Q}}
\def\dbR{{\mathbb R}}
\def\dbS{{\mathbb S}}
\def\dbT{{\mathbb T}}
\def\dbU{{\mathbb U}}
\def\dbV{{\mathbb V}}
\def\dbW{{\mathbb W}}
\def\dbX{{\mathbb X}}
\def\dbY{{\mathbb Y}}
\def\dbZ{{\mathbb Z}}

\def\la{{\langle}}
\def\ra{{\rangle}}
\def\Ker{{\rm Ker\,}}
\def\Aut{{\rm Aut}}
\def\Out{{\rm Out}}
\def\Inn{{\rm Inn}}
\def\IA{{\rm IA}}
\def\rk{{\rm rk}}
\def\summ{{\sum\limits}}
\def\phi{{\varphi}}

\bf\centerline{Math 8851. Homework \#7. To be completed by Thu, Mar 30}\rm
\vskip .1cm

\skv
{\bf 1.} Let $\Omega$ be the set of all functions $f:\dbN\to \dbR_{\geq 0}$
which are non-decreasing (that is, $f(n)\leq f(m)$ whenever $n\leq m$) -- note that any growth function
$b_{G,S}$ lies in $\Omega$. It is not hard to show that the restriction of the relation $\preceq$ to $\Omega$
can be defined by the following simpler condition (this is not part of the problem):
$$f\preceq g \iff \mbox{ there exists }C\in\dbN \mbox{ such that }f(x)\leq Cg(Cx) \mbox{ for all }n\in\dbN.$$
As before, define $f\sim g$ for $f,g\in\Omega$ if $f\preceq g$ and $g\preceq f$.

Prove that the relation $\preceq$ on $\Omega$ is transitive and that $\sim$ is an equivalence relation.
\skv

{\bf 2.} Let $S_1$ and $S_2$ be finite generating sets for the same group $G$. Prove that $b_{G,S_1}\sim b_{G,S_2}$.
Recall that
$$B_{G,S}(n)=\{g\in G: l_{S}(g)\leq n\}\mbox{ and }b_{G,S}(n)=|B_{G,S}(n)|.$$
\skv

{\bf 3.} Let $H$ be a subgroup of a group $G$.
\begin{itemize}
\item[(a)] Prove that $b_H\preceq b_G$. Recall that for a group $\Gamma$, $b_{\Gamma}$ is the equivalece class of the functions $b_{\Gamma,S}$ where $S$ is a finite generating set for $\Gamma$ (all such functions are indeed equivalent by Problem~2).

\item[(b)] Prove that if $H$ has finite index in $G$, then $b_H\sim b_G$.
\end{itemize}
\skv

{\bf 4.} Let $G$ be an infinite group and $S$ a finite generating set for $G$. Prove that $b_{G,S}(n)\geq n$ for all $n$. 

\skv
{\bf 5.} Let $G$ be a group and $S$ a finite generating set for $G$. As before, for $n\in\dbZ_{\geq 0}$
let $\Sigma_{G,S}(n)=\{g\in G: l_{S}(g)= n\}$ (the sphere of radius $n$) and $\sigma_{G,S}(n)=|\Sigma_{G,S}(n)|$. 
The formal power series $f_{G,S}(t)=\sum\limits_{n=0}^{\infty} \sigma_{G,S}(n) t^n$ is called the {\it spherical growth series}
of $G$ with respect to $S$.
\begin{itemize}
\item[(1)] Suppose $G=\la S\ra$ and $H=\la T\ra$ where $S,T$ are finite. Prove that
$$f_{G\times H,S\cup T}(t)=f_{G,S}(t)\cdot f_{H,T}(t).$$
\item[(2)] Let $G,H,S$ and $T$ be as in (1). Prove that
$$f_{G* H,S\cup T}(t)=\frac{f_{G,S}(t)\cdot f_{H,T}(t)}{f_{G,S}(t)+f_{H,T}(t)-f_{G,S}(t)\cdot f_{H,T}(t)}.$$
{\bf Hint:} Let $a=a(t)=f_{G,S}(t)-1$ and $b=b(t)=f_{H,T}(t)-1$. First argue that
$f_{G* H,S\cup T}(t)=1+a+b+ab+ba+aba+bab+\ldots$ (this follows from the normal form of elements of a free product).

\item[(3)] Let $G=\dbZ^d$ and $S=\{e_1,\ldots,e_d\}$ (the standard basis). Use (1) to compute precisely
$f_{G,S}(t)$ and $\sigma_{G,S}(n)$.

\item[(4)] Now let $X=\{x_1,\ldots, x_d\}$ and $F=F(X)$. Prove that $$f_{G,X}(t)=\frac{1+t}{1-(2d-1)t}$$
in 2 different ways: using the formula $\sigma_{G,S}(n)=2d(2d-1)^n$ for $n\geq 1$ derived in class and then using (2).
\end{itemize}
\end{document}