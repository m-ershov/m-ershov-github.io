\documentclass[12pt]{article}

\usepackage{amsmath}
\usepackage{amssymb}
\usepackage{amsthm}
%\usepackage{psfig}

\begin{document}
\baselineskip=16pt
\textheight=9in
\parindent=0pt 
\def\sk {\hskip .5cm}
\def\skv {\vskip .12cm}
\def\cos {\mbox{cos}}
\def\sin {\mbox{sin}}
\def\tan {\mbox{tan}}
\def\intl{\int\limits}
\def\lm{\lim\limits}
\newcommand{\frc}{\displaystyle\frac}
\def\xbf{{\mathbf x}}
\def\fbf{{\mathbf f}}
\def\gbf{{\mathbf g}}

\def\dbA{{\mathbb A}}
\def\dbB{{\mathbb B}}
\def\dbC{{\mathbb C}}
\def\dbD{{\mathbb D}}
\def\dbE{{\mathbb E}}
\def\dbF{{\mathbb F}}
\def\dbG{{\mathbb G}}
\def\dbH{{\mathbb H}}
\def\dbI{{\mathbb I}}
\def\dbJ{{\mathbb J}}
\def\dbK{{\mathbb K}}
\def\dbL{{\mathbb L}}
\def\dbM{{\mathbb M}}
\def\dbN{{\mathbb N}}
\def\dbO{{\mathbb O}}
\def\dbP{{\mathbb P}}
\def\dbQ{{\mathbb Q}}
\def\dbR{{\mathbb R}}
\def\dbS{{\mathbb S}}
\def\dbT{{\mathbb T}}
\def\dbU{{\mathbb U}}
\def\dbV{{\mathbb V}}
\def\dbW{{\mathbb W}}
\def\dbX{{\mathbb X}}
\def\dbY{{\mathbb Y}}
\def\dbZ{{\mathbb Z}}

\def\la{{\langle}}
\def\ra{{\rangle}}

\bf\centerline{Homework \#2, to be submitted by Thursday, September, 8th}\rm
\vskip .1cm

\skv
{\bf 1.} (a) Let $G$ be a cyclic group of order $n<\infty$. Prove that if $k\in\dbZ$,
then the mapping $\phi:G\to G$ defined by $\phi(x)=x^k$ is bijective if and
only if $k$ is relatively prime to $n$.

(b) Let $G$ be an arbitrary finite group of order $n<\infty$. Prove that if $k\in\dbZ$
is relatively prime to $n$, then the mapping $\phi:G\to G$ defined by $\phi(x)=x^k$ is bijective. 
{\bf Hint: } Use the main corollary of Lagrange theorem.
\skv

{\bf 2.} Let $F$ be a finite field of order $q$.

(a) Prove that the group $GL_2(F)$ has order $q(q-1)^2(q+1)$

(b) Prove that the group $SL_2(F)$ has order $q(q-1)(q+1)$

(c) State and prove the formula for the order of the groups $GL_n(F)$
and $SL_n(F)$ for $n>2$.

{\bf Hint: } To determine the order of $GL_n(F)$ use the fact that a square matrix over a field $F$
is invertible if and only if its rows are linearly independent. 
Use this to count the number of possible choices for the first row, then the 
number of choices for the second row, once the first one has been fixed etc.
\skv

{\bf 3.} (a) Let $n\geq 2$ be an integer. Prove that the quotient group $\dbZ/n\dbZ$ is isomorphic to $\dbZ_n$.

(b) Prove that the quotient group $\dbR/\dbZ$ is isomorphic to the group
of rotations of a circle.

(c) Prove that every element of the group $\dbQ/\dbZ$ has finite order.
\skv

{\bf 4 (practice).} Let $G$ be a group and $H$ a normal subgroup of $G$, and let
$\pi: G\to G/H$ be the canonical projection. Let $\mathcal S$
be the set of all subgroups of $G$ containing $H$ and $\mathcal T$
the set of all subgroups of $G/H$. Define the mappings $f: \mathcal S\to \mathcal T$
and $g: \mathcal T\to \mathcal S$ by $f(X)=\pi(X)$ and $g(Y)=\pi^{-1}(Y)$.

(a) Prove that $f$ and $g$ are mutually inverse, and thus there is a natural
bijection between the sets $\mathcal S$ and $\mathcal T$.

(b) Prove any two (your choice) of parts (1)-(5) of the lattice isomorphism theorem
(Theorem 20 on page 99, DF).
\skv

{\bf 5.} Let $G$ be a group and let $H$ and $K$ be subgroups of $G$ of finite index
(note that $G$ is not assumed to be finite).

(a) Assume that $H\subseteq K$. Prove that $[G:H]=[G:K][K:H]$ (recall that $[A:B]$
denotes the index of a subgroup $B$ in a group $A$).

(b) (independent of (a)) Let $m=[G:H]$ and $n=[G:K]$. Prove that
${\rm LCM}(m,n)\leq [G:H\cap K]\leq  mn$ (where ${\rm LCM}$ is the least common multiple).

{\bf Hint for (a):} If $A$ is a group and $B$ a subgroup of $A$, a
subset $S$ of $A$ is called a (left) transversal of $B$ in $A$
if $S$ contains precisely one element from each left coset of $B$
(an alternative name for a transversal is a system of left coset representatives).
Let $\{g_1,\ldots, g_r\}$ be a left transversal of $K$ in $G$
and $\{k_1,\ldots, k_s\}$ a left transversal of $H$ in $K$. Prove
that $\{g_i k_j\}_{1\leq i\leq r, 1\leq j\leq s}$ is a left transversal for $H$ in $G$.
Recall that if $B$ is a subgroup of a group $G$, then $xB=yB$ $\iff$ $x^{-1}y\in B$
for $x,y\in G$.


\skv
{\bf 6.} (a) Let $(G,X,.)$ be a group action. Recall that for a subset $S$ of $X$
we put
$PStab_G(S)=\{g\in G:\, g . s=s\mbox{ for all }s\in S\}$ (the poinwise stabilizer of $S$) and 
$Stab_G(S)=\{g\in G:\, g. S=S\}$ (the stablizer of $S$). 
Prove that $PStab_G(S)$ is a normal subgroup of $Stab_G(S)$.

(b) Let $R$ be a commutative ring with $1$ and $n\geq 2$ an integer.
Let $UT_{n}(R)$ be the subgroup of $GL_n(R)$ consisting of upper-triangular
matrices and $U_{n}(R)$ the subgroup of $GL_n(R)$ consisting of upper-unitriangular
matrices (upper-triangular matrices with $1$'s on the diagonal). Use (a) to
prove that $U_{n}(R)$ is normal in $UT_{n}(R)$.
\end{document}
