\documentclass[12pt]{article}

\usepackage{amsmath}
\usepackage{amssymb}
\usepackage{amsthm}
%\usepackage{psfig}

\begin{document}
\baselineskip=16pt
\textheight=9.3in
\parindent=0pt 
\def\sk {\hskip .5cm}
\def\skv {\vskip .12cm}
\def\cos {\mbox{cos}}
\def\sin {\mbox{sin}}
\def\tan {\mbox{tan}}
\def\intl{\int\limits}
\def\lm{\lim\limits}
\newcommand{\frc}{\displaystyle\frac}
\def\xbf{{\mathbf x}}
\def\fbf{{\mathbf f}}
\def\gbf{{\mathbf g}}

\def\dbA{{\mathbb A}}
\def\dbB{{\mathbb B}}
\def\dbC{{\mathbb C}}
\def\dbD{{\mathbb D}}
\def\dbE{{\mathbb E}}
\def\dbF{{\mathbb F}}
\def\dbG{{\mathbb G}}
\def\dbH{{\mathbb H}}
\def\dbI{{\mathbb I}}
\def\dbJ{{\mathbb J}}
\def\dbK{{\mathbb K}}
\def\dbL{{\mathbb L}}
\def\dbM{{\mathbb M}}
\def\dbN{{\mathbb N}}
\def\dbO{{\mathbb O}}
\def\dbP{{\mathbb P}}
\def\dbQ{{\mathbb Q}}
\def\dbR{{\mathbb R}}
\def\dbS{{\mathbb S}}
\def\dbT{{\mathbb T}}
\def\dbU{{\mathbb U}}
\def\dbV{{\mathbb V}}
\def\dbW{{\mathbb W}}
\def\dbX{{\mathbb X}}
\def\dbY{{\mathbb Y}}
\def\dbZ{{\mathbb Z}}

\def\la{{\langle}}
\def\ra{{\rangle}}
\def\phi{{\varphi}}

\bf\centerline{Homework \#1, to be submitted by Thursday, September, 1st}\rm
\vskip .1cm

\skv
{\bf 1.} Let $A$ be a set and let $f:A\to A$ and $g:A\to A$ be mappings such that
$f\circ g=id_A$ (where $id_A$ is the identity mapping on $A$).

(a) Prove that $f$ is surjective and $g$ is injective

(b) Show by example that $f$ need not be injective and $g$ need not be surjective
\skv

{\bf 2.} (a) Let $G$ be a group, and define the relation $\sim$ on $G$: $x\sim y$
if $y$ is conjugate to $x$ in $G$, that is, there exists $g\in G$ such that
$y=gxg^{-1}$. Prove that $\sim$ is an equivalence relation.

(b) Let $G$ be a group. An equivalence relation $\sim$ on $G$ is called a {\bf congruence}
if for any $a,b,c,d\in G$ such that $a\sim b$ and $c\sim d$ we have $ac\sim bd$.

Now let $H$ be a subgroup of $G$ and define the relation $\sim_H$ on $G$ by
$$a\sim_H b \iff a^{-1}b\in H.$$ Prove that $\sim_H$ is a congruence
if and only if $H$ is normal in $G$.

\skv
{\bf 3.} Let $G$ be a group.
(a) Define $\phi: G\to G$ by $\phi(g)=g^2$. Prove that $\phi$ is a homomorphism
if and only if $G$ is abelian.

(b) Assume that $x^2=1$ for any $x\in G$. Prove that $G$ is abelian.
\skv

{\bf 4.} Find the minimal $n$ for which the symmetric group $S_n$ contains an element of order $15$
(and explain why your $n$ is indeed minimal). {\it Note: }All you need to know about $S_n$ for this
problem is stated in Section 1.3 of DF (pp.29-32).
\skv

{\bf 5.} Prove that an element $\bar a\in\dbZ_n$ is invertible if and only if $gcd(a,n)=1$
where $gcd$ is the greatest common divisor. You may use any standard theorem about integers
(e.g. unique factorization), but do not use any theorems about $\dbZ_n$.

{\bf Hint: } The forward direction is easy. For the opposite direction
either use the theorem about representation of $gcd(a,n)$ as an integral linear 
combination of $a$ and $n$ or, alternatively, show that the mapping $\phi_n:\dbZ_n\to\dbZ_n$
given by  $\phi_n(\bar x)=\bar x\bar a$ is injective whenever $gcd(a,n)=1$.
\skv
{\bf 6.} (a) Prove that every finite group is finitely generated. 

(b) Let $\dbQ$ be the group of rational numbers with addition. Prove that $\dbQ$
is not finitely generated. 

(c) Prove that any finitely generated subgroup of $\dbQ$ is cyclic.
\skv
{\bf 7.} Let $G$ be a group and $H$ a subgroup of $G$. Let $G/H$ (resp. $H\backslash G$)
be the set of left (resp. right) cosets of $H$ in $G$. Construct an explicit
bijection between $G/H$ and $H\backslash G$. {\bf Remark:} This result explains
why we can talk about the index $[G:H]$ of a subgroup $H$ of $G$ instead
of talking about left and right indices.
\end{document}