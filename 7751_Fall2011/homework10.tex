\documentclass[12pt]{article}

\usepackage{amsmath}
\usepackage{amssymb}
\usepackage{amsthm}
\usepackage{amscd}
\usepackage[all]{xy}
%\usepackage{psfig}

\begin{document}
\baselineskip=16pt
\textheight=8.5in
\textwidth=6in
\parindent=0pt
\def\sk {\hskip .5cm}
\def\skv {\vskip .12cm}
\def\cos {\mbox{cos}}
\def\sin {\mbox{sin}}
\def\tan {\mbox{tan}}
\def\intl{\int\limits}
\def\lm{\lim\limits}
\newcommand{\frc}{\displaystyle\frac}
\def\xbf{{\mathbf x}}
\def\fbf{{\mathbf f}}
\def\gbf{{\mathbf g}}

\def\dbA{{\mathbb A}}
\def\dbB{{\mathbb B}}
\def\dbC{{\mathbb C}}
\def\dbD{{\mathbb D}}
\def\dbE{{\mathbb E}}
\def\dbF{{\mathbb F}}
\def\dbG{{\mathbb G}}
\def\dbH{{\mathbb H}}
\def\dbI{{\mathbb I}}
\def\dbJ{{\mathbb J}}
\def\dbK{{\mathbb K}}
\def\dbL{{\mathbb L}}
\def\dbM{{\mathbb M}}
\def\dbN{{\mathbb N}}
\def\dbO{{\mathbb O}}
\def\dbP{{\mathbb P}}
\def\dbQ{{\mathbb Q}}
\def\dbR{{\mathbb R}}
\def\dbS{{\mathbb S}}
\def\dbT{{\mathbb T}}
\def\dbU{{\mathbb U}}
\def\dbV{{\mathbb V}}
\def\dbW{{\mathbb W}}
\def\dbX{{\mathbb X}}
\def\dbY{{\mathbb Y}}
\def\dbZ{{\mathbb Z}}

\def\la{{\langle}}
\def\ra{{\rangle}}

\def\Ker{{\rm Ker\,}}
\def\Aut{{\rm Aut}}
\def\Inn{{\rm Inn}}

\bf\centerline{Homework \#10. }\rm
\vskip .1cm
{\bf Plan for next week:} Finite fields (9.5 + some other stuff), Hilbert basis
theorem (9.6).

\centerline{\bf Problems, to be submitted by Tuesday, November, 22nd}
\skv 
{\bf 1.} Let $R=\dbZ+x\dbQ[x]$, the subring of $\dbQ[x]$ consisting of
polynomials whose constant term is an integer.

(a) Show that the element $\alpha x$, with $\alpha\in\dbQ$ is NOT irreducible
in $R$. Deduce that $x$ cannot be written as a product of irreducibles in $R$.
Note that by Proposition~21.4 this implies that $R$ is not Noetherian.

(b) Now prove directly that $R$ is not Noetherian by showing that
$I=x\dbQ[x]$ is an ideal of $R$ which is not finitely generated.

(c) Give an example of a non-Noetherian domain which is a UFD.


\skv
{\bf 2.} Give an example of a domain $R$ (other than a field or
the zero ring) which has no irreducible elements. {\bf Hint:}
Start with the ring of power series $R=F[[x]]$ where $F$ is a field.
Then up to associates $x$ is the only irreducible element of $R$.
Construct a larger ring $R_1\supseteq R$ s.t. $x$ is reducible in $R_1$, but
$R_1\cong F[[x]]$. Then iterating the process construct
an infinite ascending chain $R\subseteq R_1\subseteq R_2\subseteq \ldots$ and consider its union.
\skv

{\bf 3.} (a) Let $R$ be a domain and let $f\in R$. Prove that $f$ is irreducible in $R$ if and only if $f$ is irreducible in $R[x]$.
\skv

(b) Recall the main theorem of Lecture 22: {\it If $R$ is a UFD, then $R[x]$ is a UFD.}
This exercises provides an  alternative proof for the uniqueness of factorization in $R[x]$.

So, assume that $R$ is a UFD. Recall that by Proposition 21.5 factorization into irreducibles
in a commutative domain $S$ with $1$ is at most unique whenever every irreducible element
of $S$ is prime. Thus, it is enough to show that every irreducible element of $R[x]$ is prime in $R[x]$.
So, let $p$ be an irreducible element of $R[x]$.
Consider two cases:

{\it Case 1:} $p$ is a constant polynomial, that is $p\in R$. Show that $R[x]/ p R[x]\cong R/pR$
and use this isomorphism to prove that $p$ is prime in $R[x]$.


{\it Case 2:} $p$ is a non-constant polynomial. In this case one can prove that $p$ is prime in $R[x]$
via the following chain of implications, where $F$ denotes the field of fractions of $R$:

$f$ is irreducible in $R[x]$ $\Rightarrow$ $p$ is irreducible in $F[x]$ $\Rightarrow$ $p$ is prime in $F[x]$  
$\Rightarrow$ $p$ is prime in $R[x]$

The first two of these implications easily follow from things we proved in class. The third one can be proved
similarly to Gauss lemma.


\skv
{\bf 4.} Let $F$ be a field, take $f(x,y)\in F[x,y]$, and write $f(x,y)=\sum_{i=0}^n c_i(y) x^i$
where $c_i(y)\in F[y]$. Suppose that 

\sk (i) There exists $\alpha\in F$ such that $c_n(\alpha)\neq 0$

\sk (ii) $gcd(c_0(y), c_1(y),\ldots, c_n(y))=1$ in $F[y]$

\sk (iii) $f(x,\alpha)$ is an irreducible element of $F[x]$ (where $f(x,\alpha)$ is the polynomial
obtained from $f(x,y)$ be substituting $\alpha$ for $y$). 

Prove that $f(x,y)$ is irreducible in $F[x,y]$.
\skv
{\bf 5.} Prove that the following polynomials are irreducible:

(a) $f(x,y)=y^3+x^2 y^2+x^3y+x^2+x$ in $\dbQ[x,y]$

(b) $f(x,y)=xy^2+x^2y+2xy+x+y+1$ in $\dbQ[x,y]$

(c) $f(x)=x^5-3x^2+15x-7$ in $\dbQ[x]$
\skv
{\bf Hint for (c):} By Gauss Lemma, it is enough to prove irreducibility
of $f(x)$ in $\dbZ[x]$. Consider the reduction map $u(x)\to \overline u(x)$
from $\dbZ[x]$ to $\dbZ_3[x]$, consider possible factorizations of
$\overline f(x)$ and show that none of them can be lifted to a factorization
of $f(x)$ (the general idea is similar to the proof of the Eisenstein criterion).
\skv
{\bf 6.} Let $p$ be a prime. Use direct counting argument to find the number
of monic irreducible polynomials of degree $n$ in $\dbF_p[x]$ for $n=2,3,4$ and
check that your answer matches the general formula derived in the online supplement
(to be posted).
{\bf Hint:} The number of irreducible monic polynomials of degree $n$ equals
the total number of monic polynomials of degree $n$ minus the number
of reducible monic polynomials of degree $n$; the latter can be computed
consdering possible factorizations into irreducibles (assuming the number of 
irreducible monic polynomials of degree $m$ for $m<n$ has already been computed).
\end{document}
