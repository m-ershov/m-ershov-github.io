\documentclass[12pt]{article}

\usepackage{amsmath}
\usepackage{amssymb}
\usepackage{amsthm}
%\usepackage{psfig}

\begin{document}
\baselineskip=15pt
\textheight=9in
\parindent=0pt
\def\sk {\hskip .5cm}
\def\skv {\vskip .12cm}
\def\cos {\mbox{cos}}
\def\sin {\mbox{sin}}
\def\tan {\mbox{tan}}
\def\intl{\int\limits}
\def\lm{\lim\limits}
\newcommand{\frc}{\displaystyle\frac}
\def\xbf{{\mathbf x}}
\def\fbf{{\mathbf f}}
\def\gbf{{\mathbf g}}

\def\Ker{{\rm Ker\,}}
\def\phi{\varphi}

\def\dbA{{\mathbb A}}
\def\dbB{{\mathbb B}}
\def\dbC{{\mathbb C}}
\def\dbD{{\mathbb D}}
\def\dbE{{\mathbb E}}
\def\dbF{{\mathbb F}}
\def\dbG{{\mathbb G}}
\def\dbH{{\mathbb H}}
\def\dbI{{\mathbb I}}
\def\dbJ{{\mathbb J}}
\def\dbK{{\mathbb K}}
\def\dbL{{\mathbb L}}
\def\dbM{{\mathbb M}}
\def\dbN{{\mathbb N}}
\def\dbO{{\mathbb O}}
\def\dbP{{\mathbb P}}
\def\dbQ{{\mathbb Q}}
\def\dbR{{\mathbb R}}
\def\dbS{{\mathbb S}}
\def\dbT{{\mathbb T}}
\def\dbU{{\mathbb U}}
\def\dbV{{\mathbb V}}
\def\dbW{{\mathbb W}}
\def\dbX{{\mathbb X}}
\def\dbY{{\mathbb Y}}
\def\dbZ{{\mathbb Z}}

\def\Aut{{\rm Aut}}
\def\lam{{\lambda}}


\def\la{{\langle}}
\def\ra{{\rangle}}

\bf\centerline{Homework Assignment \# 6. }\rm
\skv

{\bf Plan for next week:} Algebraic closures, splitting fields and normal extensions (online lectures 15,16,17 and \S~13.4 in DF).

\vskip .1cm

\bf\centerline{Problems, to be submitted by Thursday, March 7th. }\rm
\vskip .1cm

{\bf Problem 1:} Let $F$ be an algebraically closed field of characteristic zero, $V$ a vector space over $F$ and $T:V\to V$ a linear map with $T^n=I$ for some $n$.
\begin{itemize}
\item[(a)] Assume that $\dim(V)<\infty$. Use JCF to prove that $T$ is diagonalizable
\item[(b)] Now do not assume that $\dim(V)<\infty$. Prove that $T$ is still diagonalizable (by definition this means that $V$ has a basis consisting of eigenvectors for $T$). {\bf Hint:} All you need to use is that there is a polynomial $p(x)\in F[x]$ such that $p(T)=0$
and $p$ is separable, that is, $p$ is a product of distinct linear factors.
\end{itemize}
\skv





{\bf Problem 2:} \rm 

\begin{itemize}

\item[(a)] Let $K=\dbQ(\sqrt{2},\sqrt{3})$. Prove that $[K:\dbQ]=4$.

\item[(b)] Let $L=\dbQ(\sqrt{2}+\sqrt{3})$. Prove that $L=K$ and hence $[L:\dbQ]=4$ by (a).
\end{itemize}

{\bf Problem 3:} \rm Let $S=\{n_1,\ldots, n_k\}$ be a finite set of positive integers $\geq 2$
and let $K=\dbQ(\sqrt{n_1},\ldots, \sqrt{n_k})$.
\begin{itemize}
\item[(a)] Prove that $[K:\dbQ]=2^m$ for some $m\leq k$ and the set $P(S)=\{1\}\cup \{\sqrt{n} : n \mbox{ is a product of distinct elements of }S\}$ spans $K$ over $\dbQ$.

\item[(b)] For each $0\leq j\leq k$ let $\dbQ_j=\dbQ(\sqrt{n_1},\ldots, \sqrt{n_j})$ (we set $\dbQ_0=\dbQ$).
Prove that $[K:\dbQ]<2^k$ if and only if $n_1$ is a complete square or
there exists $2\leq i\leq k$ s.t. $\sqrt{n_i}=a+b\sqrt{n_{i-1}}$ for some $a,b\in\dbQ_{i-2}$.

\item[(c)] Assume that the elements of $S$ are pairwise relatively prime
and none of them is a complete square.
Prove that $[K:\dbQ]=2^k$. {\bf Hint:} Use (b) and induction on $k=|S|$.
\end{itemize}

{\bf Problem 4:} \rm Let $F$ be a field, and let $\alpha$ be an algebraic element of odd degree
over $F$ (recall that the degree of $\alpha$ over $F$ is equal to $[F(\alpha):F]$). Prove that
$F(\alpha^2)=F(\alpha)$.
\skv
\skv
{\bf Problem 5:} Let $K/F$ be a finite field extension, $n=[K:F]$, and fix some basis
$\Omega=\{\alpha_1,\ldots, \alpha_n\}$
for $K$ over $F$. For any $\alpha\in K$
define $T_{\alpha}:K\to K$ by $T_{\alpha}(\beta)=\alpha\beta$. Note that $T_{\alpha}\in  End_F(K)$.
Let $A_{\alpha}=[T_{\alpha}]_{\Omega}\in Mat_n(F)$ be the matrix of $T_{\alpha}$ with respect to $\Omega$.
\begin{itemize}


\item[(a)] (practice) Prove that the map $K\to Mat_n(F)$ given by $\alpha\mapsto A_{\alpha}$ is an injective ring homomorphism.
\item[(b)] Prove that the minimal polynomial of $\alpha$ over $F$ and the minimal polynomial of $A_{\alpha}$
coincide.

\item[(c)] Find the minimal polynomial of the matrix
$$A=\begin{pmatrix}  0&3& 5&0\\
1&0& 0&5\\
1&0& 0&3\\
0&1& 1&0\\
\end{pmatrix}$$
without doing extensive computations.
{\bf Hint:} Find an extension $K/\dbQ$ of degree $4$, a basis for $K$ over $\dbQ$
and an element $\alpha\in K$ such that $A_{\alpha}=A$ in the notations of part (a).
\end{itemize}


{\bf Problem 6:} Before doing this problem read about composite fields on pp. 528-529.
Let $E/F$ be a field extension, let $K_1$ and $K_2$ be subfields of $E$ containing $F$,
and assume that the extensions $K_1/F$ and $K_2/F$ are finite. Let $K_1 K_2$ be the composite
field of $K_1$ and $K_2$. Prove that the $F$-algebra $K_1\otimes_F K_2$ is a field if and only if
$[K_1 K_2: F]=[K_1: F][K_2: F]$. {\bf Hint:} First show that there exists an $F$-linear map
$\Phi: K_1\otimes_F K_2\to K_1 K_2$ such that $\Phi(a\otimes b)=ab$ for any $a\in K_1$ and $b\in K_2$.
Then explain why $\Phi$ is always surjective.
\end{document} 