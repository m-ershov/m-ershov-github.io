\documentclass[12pt]{article}

\usepackage{amsmath}
\usepackage{amssymb}
\usepackage{amsthm}
%\usepackage{psfig}

\begin{document}
\baselineskip=15pt
\textheight=9in
\parindent=0pt
\def\sk {\hskip .5cm}
\def\skv {\vskip .07cm}
\def\cos {\mbox{cos}}
\def\sin {\mbox{sin}}
\def\tan {\mbox{tan}}
\def\intl{\int\limits}
\def\lm{\lim\limits}
\newcommand{\frc}{\displaystyle\frac}
\def\xbf{{\mathbf x}}
\def\fbf{{\mathbf f}}
\def\gbf{{\mathbf g}}

\def\Ker{{\rm Ker\,}}
\def\Gal{{\rm Gal\,}}
\def\phi{\varphi}

\def\dbA{{\mathbb A}}
\def\dbB{{\mathbb B}}
\def\dbC{{\mathbb C}}
\def\dbD{{\mathbb D}}
\def\dbE{{\mathbb E}}
\def\dbF{{\mathbb F}}
\def\dbG{{\mathbb G}}
\def\dbH{{\mathbb H}}
\def\dbI{{\mathbb I}}
\def\dbJ{{\mathbb J}}
\def\dbK{{\mathbb K}}
\def\dbL{{\mathbb L}}
\def\dbM{{\mathbb M}}
\def\dbN{{\mathbb N}}
\def\dbO{{\mathbb O}}
\def\dbP{{\mathbb P}}
\def\dbQ{{\mathbb Q}}
\def\dbR{{\mathbb R}}
\def\dbS{{\mathbb S}}
\def\dbT{{\mathbb T}}
\def\dbU{{\mathbb U}}
\def\dbV{{\mathbb V}}
\def\dbW{{\mathbb W}}
\def\dbX{{\mathbb X}}
\def\dbY{{\mathbb Y}}
\def\dbZ{{\mathbb Z}}

\def\Aut{{\rm Aut}}
\def\deg{{\rm deg}}

\def\la{{\langle}}
\def\ra{{\rangle}}

\bf\centerline{Homework Assignment \# 9. }\rm
\skv
\skv
{\bf Plan for the next week:} Galois correspondence, continued (parts of lectures 20 and 21 we have not discussed so far),
finite fields (online lecture 22, the end of 13.5 and 14.3 in DF) and maybe start cyclic Galois extensions (online lecture 23, beginning of 14.7 in DF)
\skv
\bf\centerline{Problems, to be submitted by Thu, April 4th. }\rm
\skv
{\bf Problem 1:} Let $f(x)\in \dbQ[x]$ be an irreducible polynomial
of degree $n$, and let $K$ be a splitting field of $f(x)$.
Label the roots of $f(x)$ by $1,\ldots, n$ (in some order),
and let $\iota: \Gal(K/\dbQ)\to S_n$ be the associated embedding.

\begin{itemize}
\item[(a)] Assume $f(x)$ has at least one non-real root. Prove
that the complex conjugation is an element of $\Gal(K/\dbQ)$
of order $2$.

\item[(b)] Assume that $f(x)$ has precisely two non-real roots. Prove
that the image of the complex conjugation under the embedding $\iota$
is a transposition.

\item[(c)] Suppose that $n=\deg(f)$ is prime and again assume that $f(x)$ has precisely
two non-real roots. Prove that $\Gal(K/\dbQ)$ is isomorphic to $S_n$.
{\bf Hint:} $\Gal(K/\dbQ)$ must contain an element of order $n$ (why?)

\end{itemize}

{\bf Problem 2:} Let $K\subset\dbC$ be the splitting field of $f(x)=x^4-2$ over $\dbQ$.

\begin{itemize}
\item[(a)] Choose an order on the set of roots of $x^4-2$ and describe the associated embedding
of $\Gal(K/\dbQ)$ to $S_4$.

\item[(b)] Describe all subgroups of $\Gal(K/\dbQ)$ and the corresponding subfields of $K$.

\end{itemize}

\skv
{\bf Problem 3:} Let $p$ and $q$ be distinct primes with $q>p$,
and let $K/F$ be a Galois extension of degree $pq$. Prove that
\begin{itemize}
\item[(a)] There exists a field $L$ with $F\subseteq L\subseteq K$ and $[L:F]=q$

\item[(b)] There exists a unique field $M$ with $F\subseteq M\subseteq K$ and $[M:F]=p$.
\end{itemize}
\skv

{\bf Problem 4:} DF, Problem 17 on pages 582-583. 
\skv
{\bf Problem 5:} Let $K/F$ and $L/F$ be Galois extensions.

\begin{itemize}
\item[(a)] Prove that the extension $KL/F$ is also Galois and there is a natural
embedding $\iota:\Gal(KL/F)\to \Gal(K/F)\times \Gal(L/F)$.

\item[(b)]
Assume now that $K/F$ and $L/F$ are both finite. Prove that the map $\iota$ in (a)
is an isomorphism if and only if $K\cap L=F$.
\end{itemize}

{\bf Problem 6:} \rm Before doing this problem, read the first half
of Section 14.4 in DF (pp. 591-593).

{\bf Definition~1:} Let $L/F$ be a finite separable extension
and let $\overline F$ be an algebraic closure of $F$ containing $L$.
A subfield $L'$ of $\overline F$ is called
{\bf conjugate to $L$ over $F$} if $L'=\sigma(L)$ for some $F$-embedding of
$\sigma$ into $\overline F$. Note that $L/F$ is Galois if and
only if $L$ does not have any $F$-conjugates besides $L$ itself.
\skv
{\bf Definition~2:} A finite extension $K/F$ is called
a {\bf $p$-extension} if $K/F$ is Galois and $Gal(K/F)$ is a $p$-group.
\skv
\begin{itemize}
\item[(a)] Let $L/F$ be a separable extension of degree $n$, and
let $K$ be the Galois closure of $L$ over $F$ (see p. 594 in DF or the end of online Lecture~18 for the definition). 
Prove that $K$ can be written as a compositum $L_1L_2\ldots L_n$
where $L_1,\ldots L_n$ are (not necessarily distinct) conjugates of
$L$ over $F$.

\item[(b)] Let $K/F$ and $L/F$ be finite $p$-extensions. Prove
that $KL/F$ is also a $p$-extension.

\item[(c)] Suppose $K/L$ and $L/F$ are both $p$-extensions, and
let $M$ be the Galois closure of $K$ over $F$ (note: we do
not know whether $K/F$ is Galois or not). Prove that
$M/F$ is also a $p$-extension. {\bf Hint:} first use (b) to show that
$M/L$ is a $p$-extension.

\item[(d)] Now assume only that $L/F$ is a separable extension
with $[L:F]$ a power of $p$, and let $M$ be the Galois closure
of $L$ over $F$. Prove that $[M:F]$ need not be a power of $p$.
\end{itemize}
\end{document} 